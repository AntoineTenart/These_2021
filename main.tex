\documentclass[a4paper, 11pt,twoside,openright]{book}

\input{structure}



\title{Momentum space correlations in the depletion of weakly interacting lattice Bose gases}

\author{Antoine Ténart}

\institute{Institut d'Optique Graduate School}
\doctoralschool{Ondes et Matière}{572}{EDOM}
\specialty{Physique}
\date{TBD}
\NNT{TBD}

%\jurymember{1}{Prénom NOM}{Titre, \'etablissement}{Président}
%% Faut-il mettre le grade des personnes ?)
% \jurymember{4}{Jean-François CLEMENT}{Maître de conférences au Laboratoire de Physique des Lasers, Atomes et Molécules - Université de Lille}{Examinateur}
% \jurymember{3}{Robin KAISER}{Directeur de recherche à l'Institut de Physique de Nice - Université Côte d'Azur}{Rapporteur}
% \jurymember{2}{Dominique DELANDE}{Directeur de recherche au Laboratoire Kastler Brossel - Sorbonne Université}{Rapporteur}
% \jurymember{1}{Laurence PRUVOST}{Directrice de recherche au Laboratoire Aimé Cotton - Université Paris Saclay}{Présidente}
% \jurymember{6}{Alain ASPECT}{Directeur de recherche émérite au Laboratoire Charles Fabry - Institut d'Optique Graduate School - Université Paris Saclay}{Membre invité}
% \jurymember{5}{Vincent JOSSE}{Maître de conférences au Laboratoire Charles Fabry - Institut d'Optique Graduate School - Université Paris Saclay}{Directeur de thèse}

\frtitle{Corrélations en impulsion dans la déplétion de gaz de Bose faiblement interagissant}

\entitle{Momentum space correlations in the depletion of weakly interacting lattice Bose gases}

\frabstract{Ce travail de thèse est centré sur l'étude d'un exemple emblématique de système quantique en interaction, le condensat de Bose-Einstein faiblement interagissant. A température nulle, la théorie de Bogoliubov prévoit que l'effet conjoint des interactions et des fluctuations quantiques retire une fraction des atomes du condensat. Cette fraction est nommée déplétion quantique. De manière intéressante, la déplétion quantique est exclusivement constituée de paires d'atomes de vitesses opposées et donc corrélés en impulsion. Bien que cette prédiction date de 60 ans, ce signal de corrélation n'avait jamais été observé auparavant. Ce manuscrit rapporte la première observation de corrélations entre impulsions opposées dans la déplétion d'un condensat de Bose-Einstein faiblement interagissant. Pour ce faire, nous produisons des condensats de Bose-Einstein d'Hélium-4 métastable chargés dans des réseaux optiques. L'utilisation d'Hélium métastable rend possible la détection d'atomes individuels en 3D après un long temps de vol, aspect essentiel à la mesure d'un signal de corrélation entre particules individuelles. Après avoir exposé plusieurs résultats visant à démontrer que notre technique de détection mesure fidèlement l'impulsion d'atomes individuels, nous présentons nos signaux de corrélations dont nous étudions les principales caractéristiques (amplitude, largeur, évolution avec la température) pour mettre en valeur leur nature quantique. Au delà de confirmer une prédiction théorique, ces résultats constituent un premier pas dans l'étude des systèmes quantiques à N corps interagissant à l'équilibre au travers des corrélations en impulsion et ouvrent la voie à des études de corrélations plus complexes. Pour finir, nous présentons des résultats préliminaires sur un sujet différent, la mesure du contact de Tan via la densité en impulsion dans des gaz de Bose unidimensionnels. 



}

\enabstract{This thesis is focused on the study of the emblematic example of an interacting quantum system, the weakly-interacting Bose-Einstein condensate. At zero temperature, a fraction of the atoms is removed from the condensate through the interplay between the interactions and the quantum fluctuations. This fraction is called the quantum depletion. Interestingly, the quantum depletion is entirely made of pairs of atoms with opposite speeds, put in another way momentum correlated pairs. While this prediction is 60 years old, this correlation signal had never been observed before. This manuscript reports the first observation of opposite momentum correlations in the depletion of a weakly-interacting Bose-Einstein condensate. To do so, we produce metastable Helium-4 condensates loaded in optical lattices. Using metastable Helium makes possible the detection of individual atoms in 3D after a long time-of-flight, an essential aspect to measure a correlation signal between individual particles. After exposing several results proving that our detection technique faithfully measures the the momentum of individual atoms, we present our correlation signals of which we study the main characteristics (amplitude, width, evolution with temperature) to highlight their quantum nature. More than confirming a theoretical prediction, these results constitute a first step in the study of interacting many-body quantum systems at equilibrium and open the way to the study of more complex correlation patterns. In addition, we also present some preliminary results on a different project, namely the measure of Tan's contact using the momentum density in 1D Bose gases.


}

\frkeywords{Détection d'atomes uniques, Réseau optique, Corrélation en vitesse, Gaz quantique, Problème à N corps, Déplétion quantique}
\enkeywords{Single atom detection, Optical lattice, Momentum correlation, Quantum gas, Many-body problem, Quantum depletion}



\begin{document}
\maketitle{}

\usechapterimagetrue
\setlength{\parskip}{\parskipnew}

{\hypersetup{linkcolor=black}
\tableofcontents}
\setlength{\parskip}{\parskipnew}





\chapter*{Introduction}

\label{chap:intro}
\addcontentsline{toc}{chapter}{\nameref{chap:intro}}

\fancyhead[LO]{\sffamily\bfseries Introduction} % Print the nearest section name on the left side of odd pages
\fancyhead[RE]{\sffamily\bfseries Introduction} % Print the current chapter name on the right side of even pages




In our quest to describe and understand the world with physics, our intuition tells us to start with the small and the simple before working our way up to larger scales and more complex problems. This is exactly what we do when we are first taught about atomic physics and told to start with the smallest existing atom, the Hydrogen atom. Even before we know about quantum mechanics, we are often taught how to use simple Newtonian mechanics to calculate the circular movement of the Hydrogen electron around the nucleus in the historical Rutherford model. If we now wish to do the same for larger atoms, let us say Carbon for instance, we need to consider the electrostatic forces exerted by the nucleus on each of the 6 electrons as well as between each of the electrons, only to be quickly overcome by a feeling of helplessness at the sight of the equations we would have to solve.


This kind of problems is obviously not restricted to this small example and is actually found in many areas of physics. Would we want to study an ensemble of celestial objects orbiting around a star, electrons in a copper wire, molecules in a gas, atoms in a solid or even how a crowd behaves, a thorough description of these systems would require to account for the motions of all the individual bodies and the interactions between each one of them, leaving us with an absurd amounts of degrees of freedom and equations to solve. This is even more so true as the number of particles can get very large in these problems: a good order of magnitude is the Avogadro number $\mathcal{N}_A = 6.02 \times 10^{23}$, giving the number of carbon 12 atoms in only $12\rm{g}$ of carbon! These problems are regrouped under the denomination ``\textbf{many-body} problems''. 


Actually, many-body problems are not entirely impossible to approach theoretically. To do so, we need to go against our intuition to decompose the system into its elementary components to rather see it as whole to study its \textbf{collective} behavior. This idea is for instance at the core of the field of Thermodynamics which aims to describe ensemble properties of large numbers of particles, such as its temperature, pressure, entropy etc. while not taking any interest in the individual components of the ensemble.

\section*{Quantum many-body physics}

When we study many-body problems where the individual constituents are the (almost) smallest brick of matter, namely electrons and atoms, we enter the realm of \textbf{quantum mechanics}. The key concept to understand when a system requires a quantum treatment is the De Broglie wavelength. In 1923 \cite{debroglie:tel-00006807}, the french physicist Louis de Broglie took the hypothesis of M. Planck and A. Einstein that light could have a corpuscular aspect and turned it around by postulating that matter could behave as a wave with a wavelength $\lambda_{\rm{DB}}$ equal to:

\begin{equation}
    \lambda_{\rm{DB}} = \frac{h}{m\bm{v}}
\end{equation}

\noindent where $m\bm{v}$ is the momentum of the particle that also writes $\hbar \bm{k}$ in quantum mechanics. Strictly speaking, $\bm{k}$ designs a wave vector but we will identify it to the momentum in the rest of this manuscript. Translating this concept to many-body physics, when taking an ensemble of particles at temperature $T$, we can define the average De Broglie wavelength, also known as thermal De Broglie wavelength as:

\begin{equation}
    \lambda_{\rm{DB}} = \frac{h}{\sqrt{2 \pi m \kB T}}
\end{equation}

\noindent If the typical inter-particle density in the many-body ensemble is much larger than the thermal De Broglie wavelength, \ie $\lambda_{\rm{DB}}^3 n \ll 1$, the wave character of the particles plays no role as the different matter waves do not overlap and the system can be properly described using classical physics. On the other hand, when $\lambda_{\rm{DB}}^3 n \sim 1$, the system starts showing quantum behavior. This regime is known as the \textbf{quantum degeneracy} regime. Importantly, this condition is most often met when considering the physics of electrons in condensed matter systems even at room temperature, due to their very small mass and the high densities of particles.

\section*{Interactions in quantum systems}

The key and most essential point of quantum many-body physics is the presence of interactions between the particles. Without interactions, the system is essentially a collection of single particles that we know how to describe and the name ``many-body'' then does not make much sense. Actually, one way of studying the many-body problem is to reduce it to an effective single body problem by using the \textbf{mean-field approximation}. The idea is to approximate the action of every particle of the system on a single one as an averaged single effect, \ie consider the system as a whole and ignore the fact that there are many individual components as suggested before.

While this approach have shown to be very successful, physicist have recently tried to look beyond the mean-field approximation to consider effects between individual particles. As a result, the modern day term ``many-body physics" refers to beyond mean-field approaches that account for the presence of \textbf{correlations} between the individual components of the system. The characterization of these correlations that emerge from the interplay between the inter-particle interactions and the quantum fluctuations is the principal goal of quantum many-body physics and will also be the main point of focus of this thesis. This field of physics indeed remains to this day a largely open field with a lot of unresolved questions concerning systems ranging from solid state physics to neutron stars. We can mention the notable example of low temperature superconductivity studied by Bardeen-Cooper-Schrieffer \cite{bardeen1957theory} (BCS) in 1957 that described the superconducting current as a superfluid of Cooper pairs \cite{cooper1956bound}, where the Cooper pair describes a pair of electrons created by the presence of an interaction effect, in this case the exchange of phonons. The existence of high temperature superconductors remains however unexplained to this day and constitutes a particularly interesting question of many-body physics.

\section*{Cold atoms and quantum simulation}

Even though we have understood that exact analytical approaches are almost always impossible to study quantum many-body systems, we could however think of using numerical techniques and the calculation power of modern-day super computers. Nevertheless, if we wish to consider all kinds of correlations between the particles, the size of the associated Hilbert space grows exponentially with the number of particles considerably limiting the number of particles that can be simulated, roughly a dozen with modern day computers. In a famous paper of 1982 \cite{Feynman1982Simulating}, R. Feynman introduced the concept of \textbf{quantum simulation} by suggesting that quantum phenomena could be simulated using actual quantum components instead of classical computers. The idea is to simulate the system or Hamiltonian of interest with a quantum platform on which we can (1) precisely control all the relevant parameters and (2) easily measure the observable of interest. The technological developments of the past years have made Feynman's idea come to life with increasingly more precise and efficient simulators implemented on a variety of platforms such as ions, superconducting qbits or ultracold gases on which we will focus in this manuscript. 

Contrary to condensed matter systems, ultracold gases are a dilute state of matter in the sense that they typically contain $10^5-10^7$ atoms which is way less than solids, and in larger volumes, resulting in much lower densities. As a result, at room temperature, these gases are far from the quantum degeneracy condition. The idea is then to cool the gas down to very low temperature $\sim \mu \rm{K}$ to increase $\lambda_{\rm{DB}}$ until the system reaches the quantum degenerate regime. When the atoms are indistinguishable  bosons, under a critical temperature, a macroscopic number of atoms occupy the lowest energy state of the system, forming a new state of matter called the \textbf{Bose-Einstein Condensate (BEC)}. Importantly, all the condensed atoms then form a single, \textbf{coherent} matter-wave. 

The field of ultracold atoms was born thanks to the discovery of laser cooling techniques \cite{chu1985three,dalibard1989laser,phillips1982laser} that allowed to reach such low temperatures and led to the observations of the first BECs by the teams of E. Cornell \cite{anderson1995observation} and W. Ketterle \cite{davis1995bose} in 1995. From this day, ultracold atoms and Bose-Einstein Condensates have been the subject of many experiments and brought a large variety of important results and several Nobel prizes.

Ultracold atoms are actually perfectly suited to study condensed matter as it is relatively easy to create all sorts of potentials to trap the atoms using laser light, with the notable example of optical lattices \cite{bloch2005ultracold} that reproduces the crystalline structure of condensed matter. Another interesting properties of ultracold atoms is that the strength of the interactions can be tuned using Feshbach resonance \cite{chin2010feshbach,feshbach1958unified}. We thus have a system in which we can control the number of particles, the properties of the crystal-like potential such as its geometry and the distance between the sites and finally the strength of interactions, thus perfectly fulfilling the first condition for it to be a good quantum simulator of famous condensed matter models such as the Fermi- and Bose-Hubbard models for fermions and bosons respectively, or the Ising model. 

\section*{Time-Of-Flight measurements and the momentum distribution}

This level of control on the parameters of the simulated Hamiltonian would of course be useless if it were impossible to measure the properties of the system. One other significant asset of ultracold atoms in optical lattices systems is that they are actually easy to probe. Recent experiments typically fall into two categories, depending on what quantity the aim at measuring:

\begin{itemize}
 \item The position of the atoms, \ie the distribution of the atoms into the different lattice sites. This is done using quantum gases microscopes \cite{bakr2009quantum}, capable of detecting the fluorescence of individual atoms trapped in the lattice.
 \item The momentum of the atoms. This can be done using Bragg spectroscopy \cite{stenger1999} that we will not detail here, or Time-Of-Flight (TOF) techniques. The experiment that we will describe in this manuscript belongs to this last category.
\end{itemize}

The concept of Time-Of-Flight measurements is actually quite simple. The idea is to abruptly turn off the trapping potential to let the atoms fall under the effect of gravity and measure their positions $\bm{r}$ after a given TOF $t_{\rm{TOF}}$. In a very simple picture with classical particles that do not interact during the TOF, this position gives information about the in-trap momentum of the particle through the simple \textbf{ballistic relation}:

\begin{equation}
    \hbar \bm{k} = \frac{m \bm{r}}{t_{\rm{TOF}}}
\end{equation}

\noindent The situation is in reality more complex when accounting for the wave nature of the particles. This will be discussed in great details throughout this manuscript, but we can keep for now this simple image.

There are in fact many motivations to measure the momentum distribution to study ultracold gases and many-body physics as a whole. To begin, the momentum distribution was historically used to detect the presence of Bose-Einstein Condensation as illustrated on Fig.-\ref{fig:1st_BEC}. The BEC manifest itself by the presence of sharp peak around zero momentum with a small width set by Heisenberg uncertainty principle as the atoms are spatially localized. The momentum distribution can also be used to determine the \textbf{coherence} properties of the system, that are themselves greatly influenced by the interactions and the induced correlations between the particles. This can be understood with an analogy with optics in which far-field interference experiments are used to characterize the spatial and temporal coherence properties of a light source. With ultracold atoms, the light waves are replaced by matter-waves and the far-field regime of observation analog to the momentum distribution of the gas. The momentum distribution is also perfectly suited to study excitations signaled in momentum space by the dispersion relation from which a variety of information can be obtained \NOTE{refs}.


\begin{figure}
    \centering
    \includegraphics[width=0.7\textwidth]{Fig/Intro/BEC.png}
    \caption[Momentum distribution across Bose-Einstein Condensation]{Bose-Einstein Condensation. As temperature diminishes, the momentum distribution gets increasingly peaked around zero momentum.}
    \label{fig:1st_BEC}
\end{figure}


Last but not least, the momentum distribution also contains signatures of more complex interaction-induced correlation patterns between several individual particles. One of the most simple and famous example of such correlations are opposite momentum pairing effects. This is notably the case for the two electrons in Cooper pair as discussed earlier, as well as for the \textbf{quantum depletion} of a BEC. The quantum depletion designates the fraction of atom removed from the condensate by the effect of interactions and quantum fluctuations and is then a classic and conceptually simple example of a many-body effect. While the quantum depletion has already been observed before \cite{chang2016,lopes2017,xu2006}, there have been no direct observation of the opposite momentum pairing of quantum depleted atoms. Obtaining this result will be the main point of focus of this manuscript.

\section*{Metastable Helium and electronic detection}

Ideally, we would like to characterize all kinds of correlations present in the system, \ie correlations involving an arbitrary amount of particles. This is only achievable if a method is available to measure the momentum of each of the individual atoms of the gas. This is usually not the case in most ultracold experiments where the optical imaging techniques only allow to measure the \textbf{momentum density} of the gas rather than the full \textbf{momentum distribution}. In the early 2000s, the team lead by D. Boiron, C. Westbrook and A. Aspect at Institut d'Optique pioneered a new electronic detection technique exploiting the properties of the metastable state of the Helium atom that they managed to bring to quantum degeneracy in 2001 \cite{robert2001bose}. This detection technique has the amazing advantage of having a \textbf{single-atom} sensitivity, making it perfectly suited for the measurement of correlations in momentum space.

A second metastable Helium experiment was then built at Institut d'Optique under the direction of David Clément starting in 2011, with the observation Bose-Einstein condensation in 2015 \cite{bouton2015fast}. This new experiment implemented a new cooling sequence allowing to produce a BEC in $\sim 6\rm{s}$ instead of $\sim 30\rm{s}$ on the historical experiment, thus significantly speeding up the data acquisition time for momentum correlation measurements.

\section*{Optical lattices and the superfluid-to-Mott insulator transition}

The other specificity of this experiment is the use of optical lattices, from which the name of the team ``Helium Lattice'' derives. Optical lattices are particularly suited to study many-body, strongly interacting systems as the lattice potential locally increases the density and in turn the interactions, making phenomenon like quantum depletion even more pronounced than in regular harmonic traps. In addition, the Bose-Hubbard model predicts the existence of a phase transition from a superfluid phase to an insulating phase when the depth of the lattice potential increases known as the superfluid-to Mott insulator transition, first observed with cold atoms by I. Bloch team in 2002 \cite{greiner2002quantum}. Studying momentum correlations all across the superfluid-to Mott insulator transition sets the general frame of the work presented in this manuscript conducted during my time as an intern and then PhD student in the Helium Lattice team that I the chance to join in 2018.

\section*{Outline of the manuscript}

This manuscript is organized in five chapters. All chapters but the final one are centered around the common topic of the \kmk correlations in the quantum depletion of weakly-interacting lattice Bose gas.

\begin{itemize}
    \item The first chapter is dedicated to presenting the proper formalism to study quantum correlations. The concept of correlation functions is first introduced in the context of Optics and then extended to atomic physics. We then present the main lines of the Bogoliubov theory of the homogeneous weakly-interacting Bose gas and show what the quantum depletion is and where does the \kmk pairing comes from. Finally, we discuss some recent numerical calculations \cite{butera2020} of the correlations in the Bogoliubov theory for trapped systems, before presenting the essential experimental ingredients to observe the \kmk pairs.
    \item The second chapter is also a theoretical one and discusses the Bose-Hubbard model of bosons trapped in a 3D optical lattice. We explain what the superfluid-to-Mott insulator transition is and discuss the conditions under which the in-trap momentum distribution of the gas can be properly measured using a TOF technique, as well a the observability of the \kmk pairs of the quantum depletion in this system.
    \item The third chapter describes our experimental apparatus, namely the sequence used to produce a BEC of metastable Helium and the detection technique. In a second time, we present two experimental measurements aimed at proving the points raised in Chapter \ref{sec:chapter_2}, one proving that we are able to adiabatically prepare an arbitrary state of the Bose-Hubbard model and a second one measuring beyond-mean field two-body collision effects happening during the TOF to prove that they are negligible in usual experimental conditions and therefore not detrimental to our measurement of the momentum distribution. 
    \item The fourth chapter details our experimental observation of the \kmk pairs of the quantum depletion. We describe the numerical procedure to analyze the data and study the characteristics of the experimental correlation signals in light of Bogoliubov theory: width, amplitude and dependency to temperature. We then perform complementary analysis of the data to obtain results leading towards showing the presence of entanglement in our system: we observe a relative number squeezing measurement between modes $\bm{k}$ and $-\bm{k}$, as well as a violation of the Cauchy-Schwarz inequality. Finally, we discuss some preliminary results on the evolution of the correlation signals with momentum $k$.
    \item The fifth and last chapter is separate from the rest of this manuscript and concerns a different project that was lead during this thesis, the measurement of Tan's contact in 1D gases. We first present what Tan's contact is and present some main results of a recent theoretical study \cite{yao2018tan} of the evolution of contact with temperature and interaction strength for trapped 1D bosons, before showing how our experimental apparatus can be adapted for this kind of measurements. We then present the procedure used to extract the contact from the raw experimental data and discuss the first preliminary results and their discrepancy with theory. We conclude by giving a few possible explanations for these discrepancies and discussing what we plan on doing next.
\end{itemize}











\chapter{Quantum correlations in the weakly-interacting Bose gas}

\NOTE{TOUT CECI EST MAL ECRIT}

One of the key and on-going challenges of quantum mechanics is to understand macroscopic systems containing a large number of particles $N$, commonly referred as many-body physics. Trying to consider all the possible degrees of freedom of each individual particle and interactions effects would result in an incredibly complex problem impossible to solve theoretically. Studying such systems thus require to use approximations ...

Physicists were able to describe a gas of a large number of bosonic particles with increasing complexity throughout history. The first step was the development of statistical physics, aiming to build a bridge between the microscopic properties of individual atoms or molecules and macroscopic properties of bulk materials described by thermodynamics. This approach culminated in the theory of the ideal Bose gas, ideal meaning here that all particles are non-interacting. This theory found great success with the notable prediction of a new state of matter, the Bose-Einstein condensate. 

The next step was then to increase the complexity of the problem by adding interactions between the particles. ...

While also greatly successful, the mean-field approach neglects by essence interaction phenomena between individual particles. To characterize such effects, we thus need to go beyond the mean-field approximation. \NOTE{c'est bien nul je laisse pour l'instant}




\section{Correlation functions}

\subsection{First order correlation function of light}

Let us begin our journey with correlation functions with the simple example of the classical description of light. Correlation functions of light were developed in strong connection with the notion of \textbf{coherence} characterizing the possibility for waves to interfere. A light field is said to be coherent when there is a fixed phase relationship for the electric field at different positions (spatial coherence) and different times (time coherence). 


\begin{figure}
    \centering
    \includegraphics[width=0.55\textwidth]{Fig/Chapter1/michelson.png}
    \caption{Principle of the Michelson interferometer.}
    \label{fig:michelson}
\end{figure}


To illustrate where correlation functions come from, let us begin by taking a look at time coherence in the emblematic Michelson interferometer (see Fig.-\ref{fig:michelson}). We send at the input of the interferometer a complex light field $E(t)$. The intensity measured by the detector writes:

\begin{equation}
    I=\mean{\abs{E(t)+E(t-\tau)}^2}
    \label{eq:i_michelson}
\end{equation}

where $\mean{...}$ denotes the time average made by the detector and $\tau=\frac{2x}{c}$ the delay between the two interfering waves induced by the optical path difference between the two arms of the interferometer. Developing equation \ref{eq:i_michelson} we get:

\begin{equation}
    I=\mean{\abs{E(t)}^2} + \mean{\abs{E(t-\tau)}^2} + 2 {\rm{Re}} \mean{E(t) E^*(t-\tau)}
\end{equation}

For simplicity sake, let's assume that the source is stationary to write $\mean{\abs{E(t)}^2} = \mean{\abs{E(t-\tau)}^2} = I_0$, we then obtain :

\begin{equation}
    I= 2 I_0 (1 + {\rm{Re}} (g^{(1)} (\tau))) \ , \ g^{(1)} (\tau) = \frac{ \mean{E(t) E^*(t-\tau)}}{\mean{\abs{E}^2}}
\end{equation}

We have introduced the normalized \textbf{first-order correlation function} $g^{(1)}$ that characterizes the interference term. If $E(t)$ and $E(t-\tau)$ are independent and thus uncorrelated, $\mean{E(t) E^*(t-\tau)} = \mean{E(t)} \mean{ E^*(t-\tau)}=0$ and interference cannot be observed \NOTE{corriger}. On the other hand, if there is a \textbf{correlation} between these two quantities, an interference phenomenon can be observed. The same reasoning can be conducted for the spatial coherence for instance with the non less famous Young double slit experiment. 

We will not detail here all the intricacies of the study of first-order correlation functions for different light sources. The main point to remember is that the first order correlation function is a natural way to characterize the coherence properties of a light source, and thus its ability to produce interferences. 





\subsection{Classical example : HBT with chaotic light}

\subsection{HBT with the perfect Bose gas}

\section{Bogoliubov theory of the weakly-interacting gas}

Thus far, we have familiarized ourselves with correlation functions and seen through a few examples the kind of information they contain. We will now try to use this tool to study a many-body problem, the weakly-interacting homogeneous Bose gas, {\it i.e.} an ensemble of bosonic particles with weak contact interactions in a box of volume $V$. The theoretical description of this system has been developed by Nikolay Bogoliubov in his famous 1947 article \cite{bogoliubov1947}. In this section, we will remind the main lines of Bogoliubov's approach and see what it tells us in terms of correlation functions.

\subsection{Bogoliubov approximation}

\NOTE{Etapes avant?}

The Hamiltonian of the weakly-interacting Bose gas with contact interactions writes:

\begin{equation}
    \hat{H}=\sum_{\bm{k}}\frac{\hbar^2 k^2}{2m} \hat{a}^{\dagger}_{\bm{k}}  \hat{a}_{\bm{k}} +  \frac{g}{2V} \sum \hat{a}^{\dagger}_{\bm{k_1}+\bm{k_3}} \hat{a}^{\dagger}_{\bm{k_2}-\bm{k_3}} \hat{a}_{\bm{k_1}} \hat{a}_{\bm{k_2}} 
\end{equation}

\noindent where $g=\dfrac{4 \pi \hbar^2 a_s}{m}$ is the strength of the interactions with $a_s$ the s-wave sacttering length. \NOTE{DILUTENESS?} In order to simplify this Hamiltonian, we use the Bogoliubov approximation that relies on two points:

\begin{itemize}
    \item Since the interactions are weak, we assume that the number of atoms outside of the BEC is small. We therefore only consider interaction processes removing two particles from the BEC or bringing back two particles into the BEC. Mathematically speaking, we drop all terms higher than quadratic in $\hat{a}_{\bm{k}}$ and $\hat{a}^{\dagger}_{\bm{k}}$.
    \item The number of atoms $\NBEC$ is assumed to be very large. Therefore, we replace $\hat{a}_{\bm{0}}$ and $\hat{a}^{\dagger}_{\bm{0}}$ by $\sqrt{\NBEC}$. 
\end{itemize}

With these approximation, the simplified Hamiltonian writes:

\begin{equation}
    H_{\rm{bogo}}=\sum_{\bm{k}}\frac{\hbar^2 k^2}{2m} a^{\dagger}_{\bm{k}}  a_{\bm{k}} +  \frac{gn}{2} \sum_{\bm{k}} (a^{\dagger}_{\bm{k}} a^{\dagger}_{-\bm{k}} +a_{\bm{k}} a_{-\bm{k}})+\frac{gn\NBEC}{2}
\end{equation}

What we now want to is diagonalize the Hamiltonian. This is achieved through the linear Bogoliubov transformation where we introduce a new operator:

\begin{equation}
    \hat{b}_{\bm{k}}=u_{\bm{k}} \hat{a}_{\bm{k}} + v_{-\bm{k}} \hat{a}^{\dagger}_{-\bm{k}}
\end{equation}

To determine the expression of the coefficients $u_{\bm{k}}$ and $v_{\bm{k}}$, we impose that the new operator $\hat{b}_{\bm{k}}$ follows the bosonic operator commutation rule:

\begin{equation}
    [\hat{b}_{\bm{k}},\hat{b}^{\dagger}_{\bm{k}'}]= \delta_{\bm{k},\bm{k}'}
\end{equation}

This gives $u_{\bm{k}}^2 -  v_{-\bm{k}}^2 =1$. We can therefore write $u_{\bm{k}}={\rm cosh}(\alpha_{\bm{k}})$ and $v_{-\bm{k}}={\rm sinh}(\alpha_{\bm{k}})$ and look to determine $\alpha_{\bm{k}}$. This value must be chosen so that the coefficients of the terms in $\hat{b}^{\dagger}_{\bm{k}} \hat{b}^{\dagger}_{-\bm{k}}$ and $\hat{b}_{\bm{k}} \hat{b}_{-\bm{k}}$ vanish. We obtain an additional equation:

\begin{equation}
    \frac{g n}{2}\left(u_{\bm{k}}^{2}+v_{-\bm{k}}^{2}\right)+\left(\frac{k^{2}}{2 m}+g n\right) u_{\bm{k}} v_{-\bm{k}}=0
\end{equation}

from which we finally obtain after a few calculations using the properties of hyperbolic functions:

\begin{equation}
    u_{\bm{k}}, v_{-\bm{k}}=\pm\left(\frac{\hbar^2k^{2} / 2 m+g n}{2 \varepsilon(k)} \pm \frac{1}{2}\right)^{1 / 2}
\end{equation}

with 

\begin{equation}
    \varepsilon(k)=\sqrt{\frac{\hbar^2 k^2}{2m}(\frac{\hbar^2 k^2}{2m}+2gn)}
\end{equation}

the famous Bogoliubov dispersion relation. The Hamiltonian has now been diagonalized and writes:

\begin{equation}
    \hat{H}_B = \sum_{\bm{k}}\varepsilon(k) b^{\dagger}_{\bm{k}}  b_{\bm{k}}+E_0
\end{equation}

The system of interacting particles has thus been transformed into a system of non-interacting Bogoliubov quasi-particles associated to creation and annihilation operators $\hat{b}^{\dagger}_{\bm{k}}$ and $\hat{b}_{\bm{k}}$ with a dispersion relation $\varepsilon(k)$. The prediction of this excitation spectrum is one of the key results of Bogoliubov theory that we will now discuss in further details.

\subsection{Spectrum of excitations}

The Bogoliubov dispersion relation has two clear asymptotic trends for small and high momentum values. For low values of $k$, using $\frac{\hbar^2 k^2}{2m} \ll 2gn$, we obtain:

\begin{equation}
    \varepsilon(k) \underrel{k \to 0}{=} \hbar k \sqrt{\frac{gn}{m}}
\end{equation}

The dispersion relation takes a phonon-like linear dispersion form where the sound velocity is $c=\sqrt{\dfrac{gn}{m}}$. In this regime, the Bogoliubov quasi-particles are thus phonons that can be sen as a coherent superposition of a forward and backward propagating real particles $\hat{b}_{\bm{k}}=u_{\bm{k}} \hat{a}_{\bm{k}} + v_{-\bm{k}} \hat{a}^{\dagger}_{-\bm{k}}$ with $\abs{u_{\bm{k}}} \simeq \abs{v_{-\bm{k}}}$.

On the other hand, at high values of $k$, the dispersion relation becomes the one of free particles:

\begin{equation}
    \varepsilon(k) \underrel{k \to +\infty}{=} \frac{\hbar^2 k^2}{2m}
\end{equation}

In terms of operators, $v_k \underrel[c]{k \to +\infty}{=} 0$ and $u_k \underrel[c]{k \to +\infty}{=} 1$ so $\hat{b}_{\bm{k}}\underrel[c]{k \to +\infty}{=}\hat{a}_{\bm{k}}$, a quasi-particle is equivalent to a real particle. 

The transition between the two regimes occurs when $\frac{\hbar^2 k^2}{2m} \simeq gn$, it thus convenient to define a characteristic length associated to this momentum range:

\begin{equation}
    \xi = \sqrt{\frac{\hbar^2}{2mgn}}
\end{equation}

This length is called the \textbf{healing length} \NOTE{COMPLETER?}. We will use it later when discussing our experimental results to characterize the region of the Bogoliubov spectrum we are probing.

The Bogoliubov spectrum of excitation has been a very successful theoretical prediction observed experimentally in a large variety of systems \cite{miller1962, steinhauer2002excitation, ozeri2005, fontaine2018, stepanov2019}. However, the Bogoliubov theory also gives an additional prediction for the \textbf{ground-state} of the system.

\begin{figure}
    \centering
    \includegraphics[width=0.65\textwidth]{Fig/Chapter1/bogo_steinhauer.png}
    \caption{Experimental observation of the Bogoliubov excitation spectrum (Steinhauer {\it et al.} \cite{steinhauer2002excitation}). The phononic and free particle parts are clearly identifiable. The inset shows a zoom on the linear part of the spectrum and the dashed the free particle spectrum. $\xi$ is the healing length of the condensate.}
    \label{fig:my_label}
\end{figure}

\subsection{The quantum depletion}

As we have just seen, the Bogoliubov approach describes the weakly-interacting Bose gas excitations as non-interacting quasi-particles. They therefore behave as ideal bosons and follow the Bose distribution:

\begin{equation}
    \langle b^{\dagger}_{\bm{k}}  b_{\bm{k}} \rangle=\frac{1}{e^{-\varepsilon(k)/(k_B T)}-1} 
\end{equation}

Naturally, at $T=0$, the population of quasi-particles is null: $\langle b^{\dagger}_{\bm{k}}  b_{\bm{k}} \rangle_{T=0}=0$. That being said, let us now write the population of real particles for a given momentum $k \neq 0$:

\begin{equation}
    \langle a^{\dagger}_{\bm{k}}  a_{\bm{k}} \rangle=\textcolor{blue}{(|u_k|^2+|v_k|^2)\langle b^{\dagger}_{\bm{k}}  b_{\bm{k}} \rangle} + \textcolor{green}{|v_k|^2}
\end{equation}

The blue term corresponds to the Bogoliubov excitations populated by temperature. We will call the fraction of particles removed from the condensate this way the \textcolor{blue}{\textbf{thermal depletion}}. This fraction vanishes at $T=0$. Very interestingly, we see the apparition of an additional term \textcolor{green}{$|v_k|^2$} that results from the non commutation of the bosonic creation and annihilation operators, the signature of an essentially quantum phenomenon. This term tells us that $\langle a^{\dagger}_{\bm{k}}  a_{\bm{k}} \rangle_{T=0} \neq 0$, meaning that there are some atoms outside of the BEC with a non zero momentum in the ground state! Under the interplay between interactions and quantum fluctuations, some atoms are removed from the BEC and obtain a non zero momentum. The fraction of these atoms is called the \textcolor{green}{\textbf{quantum depletion}}. 

We are thus looking at a system which seems to fall into our general area of interest described in the introduction of this thesis, namely many-body systems with interactions displaying quantum behaviors. The weakly-interacting Bose gas shows the advantage to be one of the conceptually simplest many-body systems for which a theory can be derived as we just have shown. As explained in the beginning of this chapter, many-body systems are often efficiently characterized by correlation functions. Let us now discuss what are the relevant correlation functions to study for the weakly-interacting Bose gas.

\section{Hanbury Brown and Twiss effect with interacting atoms}


\section{k/-k correlations in the many-body ground-state}

Let us now come back to the ground-state of the weakly-interacting Bose gas and the quantum depletion. We will now try to build a microscopic, physically meaningful picture of how the quantum depletion emerges. We remind that a key ingredient of the Bogoliubov theory is that the only considered interaction processes are the ones involving two particles of the BEC or two particles outside of the BEC being brought into it. From this consideration, we understand that the atoms belonging to the quantum depletion were initially in the BEC and were removed from it after undergoing a two-body interaction process. The interaction process populating the quantum depletion thus involves two atoms in the BEC with a momentum $k \simeq 0$. To conserve the overall momentum, the two atoms leaving the BEC then have opposite momenta $\bm{k}$ and $-\bm{k}$ and form a momentum correlated pair. This falls exactly into the kind of signal we are interested in, namely correlations between several particles, here two, caused by a quantum, interaction-induced effect.

The common factor with quantum effects is that they usually defy our intuition built on our observation of the everyday world, well described by classical physics. In this case, the "quantum weirdness" comes from the fact this process seems to violate the conservation of energy: the two at atoms initially at rest each have an extra kinetic energy $\frac{\hbar^2 k^2}{2m}$ after the interaction process giving them the $\bm{k}$ and $-\bm{k}$ momenta. Naturally, the conservation of energy is still well respected here. The apparent contradiction comes from the fact that it is conceptually wrong to isolate two atoms in the BEC. The ground-state must rather be understood as a single many-body wave-function describing indistinguishable atoms, with a non zero component for momentum values $k \neq 0$. The energy of the many-body ground state of the weakly-interacting Bose gas contains a small correction that corresponds to the presence of the \kmk pairs of the quantum depletion. This small correction is called the Lee-Huang-Yang correction, named after the authors of the seminal 1957 article \cite{lee1957} that first predicted the presence of the \kmk pairs.

Although being more than 60 years old, the prediction of the presence \kmk correlations in the ground-state of a weakly interacting Bose gas still lacks an experimental observation of the microscopic \kmk pairing. This will be the main objective and central theme of this thesis. \NOTE{RAJOUTER UN TRUC SUR HAWKING TOUSSA ?}

Observing such a correlation signal requires some key experimental ingredients. The principal one is to have an experiment capable of measuring the momentum of individual atoms in momentum space and not only the momentum density as in most cold atoms experiment. This will be the subject of Chapter 3. In addition, there are several key features of the \kmk correlation signal that we need to properly understand to design an experimental scheme where the \kmk correlation signal can be properly detected.

\subsection{Momentum spread of the quantum depletion}

\subsection{Finite temperature effects}



\chapter{Optical lattices and the Bose-Hubbard model}

\label{sec:chapter_2}

Quantum gases loaded in optical lattices (lattice gases for short) are one of the paramount examples of Quantum Simulation systems. The periodical trapping potential indeed well reproduces the crystal structure of condensed matter system and allows to study relatively simple Hamiltonians such as the Bose-Hubbard Hamiltonian or the Ising model \cite{bloch2005ultracold} that however account for interactions and shows strong correlations effects. The main advantages of this experimental platform is that the different parameters of the Hamiltonians are easily set and controlled, while the information about the system is accessible through simple procedures like quantum microscopes \cite{bakr2009quantum} for the in-situ positions of the atoms or Time-Of-Flight experiments to access the momentum distribution as it is the case in this manuscript. The experimental observation of the Superfluid to Mott Insulator transition in 2002 \cite{greiner2002quantum} sparked interest in the community and lead to the development of the field from the early 2000s up until this day.

In this chapter, we will detail the main elements of the Bose-Hubbard theory of lattice gases and briefly study the Superfluid to Mott insulator transition. We will then show how and under which conditions the in-trap momentum distribution of the gas can be accessed through Time-Of-Flight measurements, before drawing the connection with the Bogoliubov theory exposed in Chapter \ref{sec:chapter_1}. The goal of this chapter is to give all the essential points necessary to obtain a system in which the \kmk pairs of the quantum depletion can be experimentally observed, rather than providing a detailled description of Bose-Hubbard physics. For a more thorough study of the Superfluid to Mott Insulator transition with our experimental apparatus, we refer the reader to the manuscript of Cécile Carcy \cite{carcy_these}.




\section{The Bose-Hubbard Model}

We consider a 3D lattice potential with cubic symmetry and spacing $d$:

\begin{equation}
    V(\bm{r})=V_{0}\left[\sin ^{2}\left(\frac{k_{d}}{2} x\right)+\sin ^{2}\left(\frac{k_{d}}{2} y\right)+\sin ^{2}\left(\frac{k_{d}}{2} z\right)\right] + V_{\mathrm{ext}} (x,y,z)
\end{equation}

\noindent where $k_d=\frac{2 \pi}{d}$ is the associated wavevector, $V_0$ the lattice depth. For convenience, $V_0$ is usually expressed in units of recoil energy $V_0 = s E_{\rm{r}}$ with $E_r=h^2/ 8 m d^2$. The term $V_{\mathrm{ext}} (x,y,z)$ denotes a harmonic potential related to the Gaussian shape of the lattice beams. For simplicity of calculations, we will first treat here the homogeneous case $V_{\mathrm{ext}} (x,y,z)=0$.

We consider an ensemble of $N$ atoms that interact with one another with the potential $U_{\rm{int}} ( \bm{r}_1, \bm{r}_2)$ loaded in the lattice potential $V(\bm{r})$. The Hamiltonian of the system writes:

\begin{equation}
    \hat{H}=\sum_{i=1}^{N} \frac{\bm{p}_{i}^{2}}{2 m}+\sum_{i=1}^{N} V\left(\bm{r}_{i}\right) + \sum_{i}^{N} \sum_{j>i}^{N} U_{\text{int}}\left(\bm{r}_{i}, \bm{r}_{j}\right)
    \label{eq:H_lattice_full}
\end{equation}

\subsubsection{Non-interacting lattice gas}

To begin, we will consider that the atoms are non-interacting and study the simplified Hamiltonian:

\begin{equation}
    \hat{H}_0=\sum_{i=1}^{N} \frac{\bm{p}_{i}^{2}}{2 m}+\sum_{i=1}^{N} V\left(\bm{r}_{i}\right)
\end{equation}

\noindent As the Hamiltonian is separable along the 3 directions of space and the gas of atoms is non-interacting, we can simply work with the one-dimensional, single particle Hamiltonian:

\begin{equation}
    \hat{H}_{\rm{1D}} = \frac{p_x^2}{2m} + \sin^2 \Big(\frac{k_d}{2} x \Big)
\end{equation}

\noindent To find the eigenstates of this Hamiltonian, we use the Bloch's theorem \cite{ashcroft1976solid}:

\begin{tcolorbox}[colback=red!5!white,colframe=red!75!black,title=\textbf{Bloch's theorem}]
\label{sec:bloch}
The eigenstates of a Hamiltonian corresponding to a spatially periodic potential $V(\bm{r})$ on a lattice $\mathcal{B}$ are Bloch waves $\psi_{\bm{q}}(\bm{r})$, product of a plane-wave $e^{i \bm{r}.\bm{q}}$ and a periodical function on $\mathcal{B}$, $u_{\bm{q}} (\bm{r})$.
\end{tcolorbox}

We therefore look for eigenstates of the form:

\begin{equation}
    \psi_{n,q} (x)= e^{iqx} u_{n,q} (x)
    \label{eq:bloch_wave}
\end{equation}

\noindent with $n \in \N$ and $q \in \R$ the \textbf{quasi-impulsion}. In order to determine the functions $u_{n,q} (x)$ and the energy $E_n (q)$, we inject equation \ref{eq:bloch_wave} in the eigenvalue equation to find that they must verify:

\begin{equation}
    \left[\frac{\left(p_{x}+\hbar q\right)^{2}}{2 m}+V_{0} \sin ^{2}\left(\frac{k_{a}}{2} x\right)\right] u_{n, q}(x)=E_{n}(q) u_{n, q}(x)
\end{equation}

\noindent As $E_n (q)$ is periodic $E_n (q+k_d)= E_n(q) \ \forall (n,q)$, we can restric the definition interval of $q$ to $[-k_d/2, k_d/2]$ which is called the \textbf{first Brillouin zone}. This equation can be easily numerically solved to obtain $u_{n, q}(x)$ and $E_n (q)$. We plot on Fig.-\ref{fig:bloch_bands} the first five energy bands as a function of $q$ in the first Brillouin zone for various values of the lattice amplitude $V_0$. Interestingly, we see that a gap appears between the different bands as we increase $V_0$. For a 3D lattice, the total energy is the sum of the energies along each direction of the lattice. The first excited band then corresponds to two 1D lowest energy bands and one 1D excited band. In order for the gap to appear, the lattice amplitude must be above $V_0 \simeq 2.2 \ E_{\mathrm{r}}$, whereas it is present at all values of $V_0$ in the 1D case.

\begin{figure}
    \centering
    \includegraphics[width=\textwidth]{Fig/Chapter2/bloch_bands.png}
    \caption{First five Bloch energy bands for various lattice amplitudes $V_0$. The gap between the first bands increases as $V_0$ increases.}
    \label{fig:bloch_bands}
\end{figure}

In addition to the Bloch waves, it is possible to define a new kind of functions called the\textbf{ Wannier functions} \cite{wannier1937structure} that are localized near the lattice sites. They are defined from the Bloch waves by:

\begin{equation}
    w_{n, j}(x)=\sqrt{\frac{d}{2 \pi}} \int_{\mathrm{BZ}} \psi_{n, q}(x) e^{-i j q d} \mathrm{~d} q \quad, j \in \mathbb{Z}
    \label{eq:wannier_functions}
\end{equation}

\noindent with BZ denoting an integration over the first Brillouin zone and where $j$ can be interpreted as the index of a lattice site. Actually, we have from equation \ref{eq:wannier_functions} the simple relation:

\begin{equation}
    w_{n, 0}(x-j d)=w_{n, j}(x)
\end{equation}

The Bloch waves can then be re-written with the definition of the Wannier functions and write:

\begin{equation}
    \psi_{n, q}(x)=\left(\frac{d}{2 \pi}\right)^{1 / 2} \sum_{j} w_{n, j}(x) e^{-i j d q}
    \label{eq:bloch_as_wannier}
\end{equation}


\begin{figure}
    \centering
    \includegraphics[width=\textwidth]{Fig/Chapter2/bloch_wannier.png}
    \caption{Real parts of the Bloch and Wannier functions for various lattice depths. When the lattice depth increases, the Bloch function is increasingly peaked around the lattice sites and the Wannier function gets narrower.}
    \label{fig:bloch_wannier}
\end{figure}

\noindent The Bloch waves are the sum of the localized Wannier function $w_{n, j}$ that can be interpreted as the wave-function of a particle located in lattice site $j$. The Bloch and Wannier functions for various lattice depths are represented on Fig.-\ref{fig:bloch_wannier}.

We can now re-write the Hamiltonian of the system with the newly introduced Wannier functions. To do so, we start by writing it in the Bloch waves basis with the second quantification formalism, introducing the operator $\hat{c}_{n,q}$ that destroys a particle in the Bloch wave $\psi_{n,q}$.

\begin{equation}
    \hat{H}_{\rm{1D}}=\sum_{n} \int_{\mathrm{BZ}} E_{n}(q) \hat{c}_{n, q}^{\dagger} \hat{c}_{n, q} \mathrm{~d} q
    \label{eq:H_bloch}
\end{equation}

\noindent To change into the Wannier function basis as defined in \ref{eq:bloch_as_wannier}, we introduce the operator $\hat{b}_{n,j}$ destroying a particle in the Wannier function $w_{n,j}$ and defined such as:

\begin{equation}
    \hat{c}_{n}(q)=\sqrt{\frac{d}{2 \pi}} \sum_{j} \hat{b}_{n, j} e^{i j d q}
    \label{eq:a_wannier}
\end{equation}

\noindent Injecting equation \ref{eq:a_wannier} in equation \ref{eq:H_bloch}, we get:

\begin{equation}
    H_{\rm{1D}}=\sum_{n} \sum_{j, j^{\prime}} J_{n}\left(j-j^{\prime}\right) \hat{b}_{n, j^{\prime}}^{\dagger} \hat{b}_{n,j}
    \label{eq:H_Jn}
\end{equation}

\noindent This Hamiltonian has a nice physical meaning: it describes the tunneling process by which a particle in site $j$ can ``hop'' to another lattice site $j'$ with the tunneling amplitude $J_n (j-j')$ that writes:

\begin{equation}
    J_{n}\left(j-j^{\prime}\right)=\frac{d}{2 \pi} \int_{\rm{BZ}} e^{i\left(j-j^{\prime}\right) q d} E_{n}(q) \mathrm{d} q
\end{equation}

\noindent This expression tells that the probability for a particle to tunnel from lattice site $j$ to $j'$ is reduced as the distance between the two sites $j$ and $j'$ increases and as the potential barrier, \ie the lattice depth, increases.

As for the rest of this thesis, we will focus ourselves on the ground-state properties of the system and therefore assume that the lowest energy band is the only one populated. This assumption is valid as long as $V_0 \geq 2.2 \ E_{\rm{r}}$ at which the gap is opening. In addition, for $V_0 \geq 3 \ E_{\rm{r}}$, we can use the tight-binding approximation for which only the tunneling events between adjacent sites are non-negligible. We thus simplify the Hamiltonian \ref{eq:H_Jn} by replacing  $J_{n}\left(j-j^{\prime}\right)$ by a constant $J$ denoting the probability to tunnel between adjacent lattice sites that we define as:

\begin{equation}
    J=-J_0(1)
\end{equation}

\noindent so that $J$ is positive. Finally, we obtain the first term of the Bose-Hubbard Hamiltonian:

\begin{equation}
    \hat{H}_{\rm{1D}} = -J \sum_{\mean{i,j}} \hat{b}^{\dagger}_i \hat{b}_j
    \label{eq:H_band}
\end{equation}

\noindent where $\mean{i,j}$ denotes the ensemble of all adjacent lattice sites $i$ and $j$. In the 3D case, the expression of the Hamiltonian remains the same.

\subsubsection{Interaction term}
We now turn to studying the interaction term that we had left out in the full Hamiltonian of equation \ref{eq:H_lattice_full}. In the formalism of second quantification, the short-range, $s$-wave, 3D interaction Hamiltonian writes:

\begin{equation}
    \hat{H}_{\mathrm{int}}=\frac{1}{2} \int d x \int d x' U_{\mathrm{int}}\left(x, x'\right) \hat{\Psi}^{\dagger}(x) \hat{\Psi}^{\dagger}\left(x'\right) \hat{\Psi}\left(x^{\prime}\right) \hat{\Psi}(x)
\end{equation}

\noindent with $\hat{\Psi}(x)$ the operator destroying a particle at position $x$ that we write in terms of Wannier functions as:

\begin{equation}
    \hat{\Psi}(x)=\sum_{j} w_{j}(x) \hat{b}_{j} = \sum_{j} w_{0}(x-x_j) \hat{b}_{j} 
    \label{eq:atom_operator_lattice}
\end{equation}

\noindent Note that we dropped the energy band number $n$ as we are studying the ground-state properties and thus only the lowest energy band. We approximate the interactions to be contact, repulsive interactions so that:

\begin{equation}
    U_{\rm{int}}= g \delta (x_1 - x_2)
\end{equation}

\noindent with $g$ the strength of the interactions. The interaction Hamiltonian can then be re-written:

\begin{equation}
    \hat{H}_{\mathrm{int}}=\frac{g}{2} \sum_{j_{1}} \sum_{j_{2}} \sum_{j_{3}} \sum_{j_{4}} \hat{b}_{j_{4}}^{\dagger} \hat{b}_{j_{3}}^{\dagger} \hat{b}_{j_{2}} \hat{b}_{j_{1}} \int w_{j_{4}}^{*}(x) w_{j_{3}}^{*}(x) w_{j_{2}}(x) w_{j_{1}}(x) \mathrm{~d} x
    \label{eq:h_int_intermediate}
\end{equation}

\noindent which is still a fairly complicated expression. We can however greatly simplify it by considering that the Wannier functions become narrower as the lattice depth increases. The overlap between the Wannier functions of the different lattice sites then becomes increasingly negligible. This means that the integral of equation \ref{eq:h_int_intermediate} is non zero only if $j_1=j_2=j_3=j_4$, \ie if we only consider on-site interactions. In the end, the interaction Hamiltonian writes:

\begin{equation}
    \hat{H}_{\rm{int}}=\frac{U_{\mathrm{1D}}}{2} \sum_j \hat{n}_j (\hat{n}_j -1)
\end{equation}

\noindent where we have introduced the on-site energy $U_{\mathrm{1D}}=g \int\left|w_{0,0}(x)\right|^{4} \mathrm{d}x$, easily generalized to the 3D case with $U=g \int(\left|w_{0,0}(x)\right|^{4} \mathrm{d}x)^3$. 

Combining this Hamiltonian to the non-interacting Hamiltonian of equation \ref{eq:H_band}, we reach the form of the famous \textbf{Bose-Hubbard Hamiltonian}:

\begin{equation}
    \hat{H}_{\mathrm{BH}}= -J \sum_{\mean{i,j}} \hat{b}^{\dagger}_i \hat{b}_j + \frac{U}{2} \sum_j \hat{n}_j (\hat{n}_j -1)
\end{equation}

\noindent Interestingly, the physics of the homogeneous ground-state depends only from the two parameters $J$ and $U$ as we will see in the next paragraph. In particular, the ratio $U/J$ depends from the lattice depth $V_0$ as illustrated on Fig-\ref{fig:U_J_vs_s}. This parameter is easily controllable in an experiment, for instance by changing the power of the laser beams used to produce the lattice potential. 

\begin{figure}
    \centering
    \includegraphics[width=0.6\textwidth]{Fig/Chapter2/illu_bose_hubbard.png}
    \caption{Representation of the Bose-Hubbard model. The physics of the system are set by the tunneling coefficient $J$ and the on-site interaction energy $U$.}
    \label{fig:my_label}
\end{figure}

\begin{figure}
    \centering
    \includegraphics[width=\textwidth]{Fig/Chapter2/U_J_vs_s.png}
    \caption{Evolution of $U$, $J$ and the ratio $U/J$ as a function of the lattice depth in log-scale.}
    \label{fig:U_J_vs_s}
\end{figure}

\section{The superfluid to Mott insulator transition}

We discuss in this section the properties of the Bose-Hubbard Hamiltonian ground-state for $N$ particles spread over $M$ sites with filling $\bar{n}=N/M$. To begin, we describe the extreme cases $U/J \to 0$ and $U/J \to \infty$, which are the only cases for which the Hamiltonian can be analytically solved. 

\subsection{Extreme cases}

\subsubsection{Perfect superfluid (SF) phase $\bm{U/J \to 0}$}

In this case, the particles are non-interacting. In these conditions, the ground-state $\ket{\Psi_0}$ of the N particles system is simply the product of the single particle ground state wave-functions, \ie the Bloch wave-functions for $q=0$ \cite{bloch2008many}:

\begin{equation}
    \ket{\Psi_0}_{\mathrm{SF}} = \frac{1}{\sqrt{N!}} (\hat{c}^{\dagger}(\bm{q}=0))^N \ket{0} = \frac{1}{\sqrt{N !}}\left(\frac{1}{\sqrt{M}} \sum_{j=1}^{M} \hat{b}_{j}^{\dagger}\right)^{N} \ket{0}
\end{equation}

\noindent The ground-state is then an ideal Bose-Einstein condensate with a condensed fraction equal to 1. In the thermodynamic limit with $N \to \infty$, $M \to \infty$, it is possible to show at the price of a few lines of complex calculations \cite{gerbier_notes} that the probability to find $n_i$ atoms at a given site $i$ is:

\begin{equation}
    p\left(n_{i}\right) \approx e^{-\bar{n}} \frac{\bar{n}^{n_{i}}}{n_{i} !}
\end{equation}

\noindent We recognize the same Poissonian distribution that we would obtain for a bosonic coherent state as described in Chapter \ref{sec:chapter_1}. We can therefore write:

\begin{equation}
    \ket{\Psi_0}_{\mathrm{SF}} \approx |\Psi\rangle_{\mathrm{coh}}=\mathcal{N} e^{\sqrt{N} \hat{c}^{\dagger}(\bm{q}=0)} \ket{0} = \mathcal{N} \prod_{i} e^{\sqrt{\bar{n}} \hat{b}_{i}^{\dagger}} \ket{0} = \prod_{i} \mathcal{N}_{i} \sum_{n_{i}=0}^{\infty} \frac{\alpha_{i}^{n_{i}}}{\sqrt{n_{i} !}}\left|n_{i}\right\rangle_{i}
    \label{eq:ground-state_superfluid}
\end{equation}

\noindent with $\alpha_i=\sqrt{\bar{n}} \ \forall i \in \Z$ and the normalization factor $\mathcal{N}_{i}=e^{-\left|\alpha_{i}\right|^{2} / 2}$. We thus find that the ground state can be described as a product of local coherent states associated to the different lattice sites.

As we did in Chapter \ref{sec:chapter_1}, we can write the first-order correlation function between two different lattice sites $i$ and $j$ to characterize the coherence properties of the ground-state:

\begin{equation}
    G^{(1)}(i,j)= \mean{\hat{b}^{\dagger}_i \hat{b}_j}
\end{equation}

\noindent In the limit $U \to 0$, $G^{(1)}(i,j)$ is easy to calculate and writes:

\begin{equation}
    G^{(1)}(i,j)= {}_{\mathrm{SF}} \bra{\Psi_0} \hat{b}^{\dagger}_i \hat{b}_j \ket{\Psi_0}_{\mathrm{SF}} = \alpha^*_i \alpha_j = \bar{n}
\end{equation}

\noindent We see that the result does not depend from the chosen lattice sites $i$ and $j$ and thus from the distance between them, indicating an infinite range coherence.


\subsubsection{Perfect Mott Insulator (MI) phase $\bm{U/J \to \infty}$}

In the opposite extreme limit, the tunneling probability goes to zero $J=0$ so that each of the lattice sites are independent from one another. The Hamiltonian reduces to:

\begin{equation}
    \hat{H}_{\mathrm{BH}} = \frac{U}{2} \sum_j \hat{n}_j (\hat{n}_j -1)
\end{equation}

Because of the strong repulsive interactions, the particle localize on the lattice sites and cannot hop from site to site as $J=0$. The ground-state is then reached by distributing the particles among the different sites of the lattice so that the number of particles per site is as low as possible to minimize the interaction energy. This corresponds to putting $\bar{n}=N/M$ particles in each of the $M$ available lattice sites. For simplicity sake, we assume here that the filling is commensurate, \ie $\bar{n}$ is an integer. The ground-state then has the simple expression of a Fock state:

\begin{equation}
    \left|\Psi_{0}\right\rangle_{\mathrm{MI}}=\frac{1}{\sqrt{N !}} \prod_{j=1}^{M}\left(\hat{b}_{j}^{\dagger}\right)^{\bar{n}}|0\rangle
    \label{eq:ground-state_MI}
\end{equation}

\noindent This state is called the \textbf{Mott insulator} state \cite{fisher1989boson}. The first-order correlation function now writes

\begin{equation}
     G^{(1)}(i,j)= {}_{\mathrm{MI}} \bra{\Psi_0} \hat{b}^{\dagger}_i \hat{b}_j \ket{\Psi_0}_{\mathrm{MI}} = \delta_{i,j}
\end{equation}

\noindent and is zero when we consider any pair of different lattice sites with $i \neq j$. In the limit $J \to 0$, the system is therefore incoherent.



\subsection{The zero-temperature Mott phase transition}

What can then be said from intermediates values of $U/J$? If we start from the case $J=0$ and progressively start increasing $J$, it becomes possible for the atoms to hop from site to site. When an atom hops to an adjacent site, the occupancy is increased, increasing the energy by $U$. When the gain in kinetic energy $J$ is smaller than $U$, this process is unfavorable and the atoms remain localized on the lattice sites. However, when $J$ is larger than $U$, the gain in kinetic energy outweighs the effect of the interactions and the atoms hop through the different sites of the lattice. The ground-state of the Bose-Hubbard then undergoes a phase transition as $U/J$ varies from an \textbf{insulating phase} to a \textbf{superfluid phase} with very different properties.

\begin{minipage}[t]{0.45\textwidth}
    \noindent \textbf{Superfluid phase}
    \begin{itemize}
        \item The atoms are delocalized over the entire lattice.
        \item The condensed fraction is non zero.
        \item The system shows long range coherence, \ie the phase is fixed.
        \item The on-site number of atoms is fluctuating.
        \item For low values of $U/J$ far from the transition, the effect of interaction is small, so that we can use the Bogoliubov approximation as we will detail later in this chapter. We find that the excitation spectrum is gapless and phonon-like at low $q$.
    \end{itemize}
\end{minipage}
\begin{minipage}[t]{0.48\textwidth}
    \noindent \textbf{Insulating phase}
    \begin{itemize}
        \item The atoms are well localized on the lattice sites.
        \item The condensed fraction is zero.
        \item The system is incoherent, the phase is fluctuating.
        \item The on-site number of atoms is well defined and non fluctuating.
        \item The excitation spectrum is gapped, with the excitations consisting of particle-hole modes that can restore short-range coherence. 
    \end{itemize}
\end{minipage}

\begin{figure}
    \centering
    \includegraphics[width=0.95\textwidth]{Fig/Chapter2/schema_superfluid_mott.png}
    \caption{Schematic of the superfluid to Mott insulator transition. In the superfluid phase, the atoms are delocalized and the system shows long range coherence, contrary to the Mott insulator phase where the atoms are well localized on the lattice sites, making the system incoherent.}
    \label{fig:my_label}
\end{figure}

\subsubsection{Phase diagram}

In the homogeneous case, the properties of the system thus change dramatically when $U/J$ crosses the \textbf{Quantum Critical Point} (QCP) $(U/J)_c$. To discuss the value of $(U/J)_c$, we slightly complexify our model where we had only considered commensurate fillings, thus fixing the chemical potential that we now set to be a free parameter. We plot on Fig.-\ref{fig:mott_lobes} the full phase diagram function of $\mu/U$, set by the filling in the Mott insulator phase $\bar{n}$ and $J/U$. As the system is homogeneous, the filling $\bar{n}$ is independent of $U/J$. The grey dashed lines correspond to iso-filling lines for a given value of $\bar{n}$. For commensurate fillings, the iso-filling lines cross the Mott stability lobes at the critical ratio $(U/J)_c$ that increases ad the filling increases. If the filling is however incommensurate (line $n=1+\varepsilon$), we notice that the system remains in the superfluid phase as long as $J \neq 0$. This is due to the fact that a small fraction of the atoms can delocalize over the whole lattice without being blocked by the interactions $U$ as there will never be two of these particles in the same site as a consequence of the fact that the filling is incommensurate.

\begin{figure}
    \centering
    \includegraphics[width=0.9\textwidth]{Fig/Chapter2/mott_lobes.png}
    \caption{a) Homogeneous phase diagram as a function of $\mu/U$ and $J/U$. The dashed lines are iso-filling lines. We observe a Superfluid to Mott Insulator phase transition for commensurate fillings. b) Wedding-cake structure for the trapped gas. The red arrow illustrate how $\mu_{\rm{eff}}$ varies and the corresponding phases as the distance from the center of the trap increases. Taken from \cite{bloch2008many}.}
    \label{fig:mott_lobes}
\end{figure}


\subsection{Trapping effects}

\label{sec:ch2_trapping_effects}

In practice, the system is not homogeneous as we require an external harmonic potential $V_{\rm{ext}}(\bm{r})$ to trap the atoms for experiments. The properties of the trapped system can be linked to the properties of the homogeneous system by applying the Local Density Approximation\footnote{Valid as long as the trapping potential varies slowly from site to site and the system is at thermal equilibrium. This approximation may however fail at the quantum critical point \cite{pollet2012recent}.} \cite{bergkvist2004local} and replacing the chemical potential by an effective one:

\begin{equation}
    \mu_{\rm{eff}}= \mu - V_{\rm{ext}}(\bm{r})
\end{equation}

\noindent This means that the chemical potential, and thus the lattice filling, varies with the distance from the center of the trap. A typical situation is illustrated by the red arrow on the phase diagram of Fig.-\ref{fig:mott_lobes} where $J/U$ is small enough for Mott phases to exist and the center of the trap corresponds to a filling $\bar{n}=2$. As we get away from the center of the trap towards regions of low $\mu_{\rm{eff}}$ following the red arrow, we exit the first Mott region $\bar{n}=2$ to enter a superfluid region where the filling decreases continuously up to a second Mott region with filling $\bar{n}=1$ and finally reach a last superfluid region at the edge of the trap at vanishing values of $\mu_{\rm{eff}}$. The Mott phases are incompressible, meaning that the density remains constant even though the external trapping potential is rising, differentiating them from the superfluid region. This results in the famous ``wedding-cake'' density profile as illustrated on panel b) of Fig.-\ref{fig:mott_lobes}.

\noindent For clarity sake, the system will be said to be in the Mott Insulator phase as long as a Mott plateau exists, \ie when $U/J > (U/J)_c$ where $(U/J)_c$ is the critical point for the corresponding homogeneous sytem.

\subsection{Finite temperature effects}

If we finally consider the effect of temperature, we obtain the complete phase diagram of Fig.-\ref{fig:phase_diagram} function of $T/J$ and $U/J$ with the apparition of an additional phase, the normal (thermal) gas. For $U/J \leq (U/J)_c$, the superfluid phase undergoes a condensation transition with temperature similar to the well-known BEC transition. This transition is induced by the thermal fluctuations and is then called classical, in opposition to the Mott transition which is driven by a variations of physical parameters of the Hamiltonian and therefore is a quantum transition. On the other hand, the Mott insulator phase also goes to the normal gas phase as the temperature increases, but with a smooth crossover. Interestingly, the Superfluid to Mott Insulator transition subsists at low temperature. The green area indicates the Quantum Critical Region for which we expect the critical quantum effects of the Mott transition to survive in spite of the temperature. While the physics of the Quantum Critical Point are a very interesting and trending topic, they fall out of the scope of this thesis and we once again refer the reader to \cite{carcy_these} for more informations on this aspect.


\begin{figure}
    \centering
    \includegraphics[width=0.78\textwidth]{Fig/Chapter2/phase_diagram.png}
    \caption{Bose-Hubbard phase diagram function of $T/J$ and $U/J$. We identify three phases, the Superfluid, Mott Insulator, and Normal Gas. The green area indicates the region in which critical point quantum effects could be observable in spite of the finite temperature.}
    \label{fig:phase_diagram}
\end{figure}


% The theoretical description of the Bose-Hubbard ground-state at an arbitrary value of $U/J$ is a very challenging task. It however exists a few approximate approaches from which we can obtain meaningful information as we will detail now.

\subsection{The Gutzwiller method}

To conclude this section, we present an approximate theoretical approach to treat the Bose-Hubbard Hamiltonian, the Gutzwiller method. This method takes its name from its author who introduced it in \cite{gutzwiller1963effect} to study strongly correlated Fermi systems. It was later adapted to characterize the ground-state of the Bose-Hubbard Hamiltonian in \cite{rokhsar1991gutzwiller}. This method is particularly useful to evaluate the density profile and therefore the size of the lattice gas with good accuracy all across the Mott transition, apart from the region close to the critical point. Although this is a ground-state, $T=0$ method that does not faithfully represent the reality of experiments, its predictions will be quite valuable to understand the correlation signals that we will present in thesis.

The method revolves around the Gutzwiller ansatz that consists in writing the many-body ground-state as a product of on-site wave-functions $\ket{\phi_i}$:

\begin{equation}
    \ket{\Psi_G} = \prod_i^{\text{sites}} \ket{\phi_i}
\end{equation}

\noindent The on-site wave-functions are then developed on the Fock-state basis:

\begin{equation}
    \ket{\phi_i}= \sum_{n_j=0}^{\infty} f(n_j) \ket{n_j}
\end{equation}

This ansatz is motivated by the fact that it matches the exact ground-state for the extreme cases:

\begin{itemize}
    \item For $U/J \to 0$, the ground-state is a coherent state (see equation \ref{eq:ground-state_superfluid}) so that $f(n_j) = \mathcal{N}_{j}  (\alpha_{j}^{n_{j}})(\sqrt{n_{j} !})$ with $\alpha_j=\sqrt{\bar{n}}$ and $\mathcal{N}_{i}=e^{-\left|\alpha_{i}\right|^{2} / 2}$.
    \item For $U/J \to \infty$, the ground-state is already a Fock state (see equation \ref{eq:ground-state_MI}) so that $f(n_j) = \delta_{n_j,\bar{n}}$.
\end{itemize}

The Gutzwiller method is a \textbf{variational} approach, meaning that the ground-state is determined by finding the coefficients $f(n_j)$ that minimizes the free energy defined as:

\begin{equation}
    G = \mean{H_{\mathrm{BH}}}_{\ket{\Psi_G}} - \mu \mean{N}_{\ket{\Psi_G}} = -J \sum_{\langle i, j\rangle} \alpha_{i}^{*} \alpha_{j}+\sum_{j} \sum_{n_{j}=0}^{\infty}\left[\frac{U}{2} n_{j}\left(n_{j}-1\right)-\mu n_{j}\right]\left|f\left(n_{j}\right)\right|^{2}
\end{equation}

\noindent with the condition $\left\langle n_{j}\right\rangle=\sum_{n_{j}=0}^{\infty}\left|f_{j}\left(n_{j}\right)\right|^{2} n_{j}=\bar{n}$. The coefficients $f(n_j)$ can be found through numerical calculations. Interestingly, following the arguments of \ref{sec:ch2_trapping_effects}, $\mu$ can be replaced by the effective $\mu_{\rm{eff}}$ to account for the effect of the external trapping potential present in our experiment.


\section{Accessing the in-trap momentum distribution in a Time-Of-Flight experiment}

Now that we have laid down the main elements of lattice gases physics, we need to determine the proper experimental tools to characterize the system. As developed in Chapter \ref{sec:chapter_1}, the main focus of this thesis will be on the \textbf{momentum space} correlations, requiring us to devise a technique to effectively measure the momentum distribution of a lattice gas. The most natural idea for momentum space measurements is to use the well-known and widely used \textbf{Time-Of-Flight} (TOF) technique. This technique consists in suddenly turning off the trapping potential to let the atoms fall under the effect of gravity and measure their positions $\bm{r}$ after a given TOF $t_{\rm{TOF}}$. In a very simple picture with classical and non-interacting particles, this position gives information about the in-trap momentum of the particle through the simple \textbf{ballistic relation}:

\begin{equation}
    \hbar \bm{k} = \frac{m \bm{r}}{t_{\rm{TOF}}}
\end{equation}

The validity of this simple relation is however far from being obvious for the quantum gases released from optical lattice. We will then describe in this section the TOF dynamics of an atomic gas released from an optical lattice to identify the conditions under which a TOF measurement can be used to properly measure the in-trap momentum of the gas.

% \subsection{The momentum distribution}

% The momentum distribution of a gas of atoms on a homogeneous lattice writes \cite{gerbier2008expansion}:

% \begin{equation}
%     n(\mathbf{k}) \propto |\tilde{w}(\mathbf{k})|^{2} \sum_{i,j} e^{i \mathbf{k} \cdot (\bm{r}_i - \bm{r}_j)} G^{(1)}(i,j)
% \end{equation}

% \noindent where $\tilde{w}(\mathbf{k})$ denotes the Fourier transform of the single site Wannier function. The momentum distribution is thus strongly dependent from the lattice depth. For $U/J \to 0$, $G^{(1)}(i,j) = \bar{n}$ meaning that the momentum distribution consists of sharp peaks located at $k = j k_d$ with $j \in \Z$. In the other limit $U/J \to \infty$, $G^{(1)}(i,j) = \delta_{i,j}$ so that the momentum distribution reduces to $|\tilde{w}(\mathbf{k})|^{2}$, \ie a Gaussian-like function.

\subsection{Expansion from the lattice and Far Field regime}

We start our calculations with the simplified case for which we neglect the effects of all interactions during the TOF. In this configuration, the problem is very similar to the diffraction of a light wave by a grating in optics, in which a diffraction interference pattern results from the coherent sum of the contribution of many source points associated to each of the diffraction grating holes. For the lattice gas, the source points correspond to the lattice sites associated to Wannier functions that will be able to interfere if the system is coherent. For simplicity sake, we will only consider the 1D case for which the 3D case can be easily obtained as the non-interacting Hamiltonian is separable.

At time $t=0$, right before the lattice potential is turned off, the atomic field operator writes (see \ref{eq:atom_operator_lattice})

\begin{equation}
    \hat{\Psi}(x)= \sum_{j} w_{0}(x-x_j) \hat{b}_{j} 
    \label{eq:field_operator}
\end{equation}



The expression of the Wannier functions is rather complex and very hard to handle in calculations. However, we can approximate the lattice potential near a minimum to its second-order Taylor expansion, \ie approximate it to a harmonic potential of frequency $\omega_L=2 \sqrt{s} (E_{\rm{r}}/\hbar)$ \cite{toth2008theory}. As illustrated in \NOTE{faire figure}, this means that the Wannier function can be well approximated by the Gaussian wave-function of the harmonic oscillator ground state:

\begin{equation}
    w_0(x) \simeq \frac{1}{\pi^{1 / 4} \sqrt{x_{0}}} \exp \left(\frac{-x^{2}}{2 x_{0}^{2}}\right)
\end{equation}

\noindent with $x_{0}=\sqrt{\hbar / m \omega_{L}}$.

When the lattice potential is turned off, each of the lattice sites wave-functions expand freely following the harmonic oscillator dynamics \cite{toth2008theory}:

\begin{equation}
    w\left(x-x_{j}, t\right)=\frac{1}{\pi^{1 / 4} \sqrt{W(t)}} \exp \left(-\frac{\left(x-x_{j}\right)^{2}}{2 W(t)^{2}}\right) \exp \left(-i \frac{\left(x-x_{j}\right)^{2}}{2 W(t)^{2}} \frac{h t}{m x_{0}^{2}}\right)
    \label{eq:time_dependent_wannier}
\end{equation}

\noindent with $W(t)=x_{0} \sqrt{1+\left(\hbar t / m x_{0}^{2}\right)^{2}}$ the width of the Gaussian envelope.

\subsubsection{The Far-Field regime}

In practice, as $\omega_L$ is high ($\sim 10^5-10^6 \ \rm{Hz}$), $W(t)$ increases very quickly. For instance, with $s=5$, $W(t)$ is multiplied by $\sim 600$ after $1 \ \rm{ms}$ of expansion and is thus much larger than the size of the lattice $L$. For $t>1 \ \rm{ms}$, we can make the approximation $W(t) \simeq \hbar t/m x_0$ as well as neglect the dependency on the initial site $x_j$ in the amplitude term as long as $|x| \ll W(t)$ so that we can write:

\begin{equation}
    \exp \left(-\frac{\left(x-x_{j}\right)^{2}}{2 W(t)^{2}}\right) \simeq \exp \left(-\frac{x^{2}}{2 W(t)^{2}}\right)
\end{equation}

\noindent This is equivalent to the paraxial approximation of the Fraunhofer diffraction regime in Optics.

Building up on the diffraction analogy, we would like to define an analog Fraunhofer distance where the dependency of the phase factor on the quadratic analog Fresnel term in $x_j$ can be neglected. Using $W(t) \simeq \hbar t/m x_0$, we obtain $\frac{\left(x-x_{j}\right)^{2}}{2 W(t)^{2}} \frac{h t}{m x_{0}^{2}} \simeq \frac{\left(x-x_{j}\right)^{2}}{2 x_0 W(t)}$ from which we derive the condition $\frac{x_j^2}{2 x_0 W(t)} \ll 1, \forall j$ that we rewrite \cite{gerbier2008expansion,toth2008theory}:

\begin{equation}
    t \gg t_{\rm{FF}} = \frac{mL^2}{2 \hbar}
    \label{eq:far_field}
\end{equation}

\noindent This condition defines the \textbf{Far-Field regime} after which the interference pattern is well developed, in analogy to the Fraunhofer regime of diffraction. Combining the different approximations, we simplify \ref{eq:time_dependent_wannier} to:

\begin{equation}
    w\left(x-x_{j}, t\right)=\sqrt{\frac{m}{\hbar t}} \tilde{w_0}[Q(x,t)] \exp\left(-i \frac{\hbar Q(x,t)^{2}}{2 m} \right) \exp\left(i Q(x,t) x_{j}\right)
\end{equation}

\noindent with $Q(x,t)=\frac{m x}{\hbar t}$ and $\tilde{w_0}$ the Fourier transform of the Wannier function. 

Now that we have the general expression of $w\left(x-x_{j}, t\right)$, we generalize it to the 3D case and inject it in equation \ref{eq:field_operator} to obtain the sum of the contribution of each site:

\begin{equation}
    \hat{\Psi} (\bm{r},t) = \left(\sqrt{\frac{m}{\hbar t}} \right)^3 \tilde{w}[\bm{Q}(\bm{r},t)] \exp\left(-i \frac{\hbar \bm{Q}(\bm{r},t)^{2}}{2 m} \right) \sum_j e^{i \bm{Q}(\bm{r},t). \bm{r}_{j}} \hat{b}_j
\end{equation}

\noindent From this expression, we finally obtain the atomic density $\rho_{\rm{TOF}}(\bm{r},t) = \mean{\hat{\Psi}^{\dagger} (\bm{r},t) \hat{\Psi} (\bm{r},t)}$ at position $\bm{r}$ and a long TOF $t$:

\begin{equation}
    \rho_{\rm{TOF}}(\bm{r},t) = \left(\frac{m}{\hbar t} \right)^3 |w_0(\bm{Q}(\bm{r},t))|^2 \sum_{i,j} e^{i \bm{Q}(\bm{r},t).(\bm{r}_j - \bm{r}_{i})} \mean{\hat{b}^{\dagger}_i \hat{b}_j }
    \label{eq:rho_tof}
\end{equation}

\noindent The density $\rho_{\rm{TOF}}(\bm{r},t)$ then consists of a smooth envelope $|w_0(\bm{Q}(\bm{r},t))|^2$ set by the Fourier transform of the Wannier function and an interference term $\sum_{i,j} e^{i \bm{Q}(\bm{r},t).(\bm{r}_j - \bm{r}_{i})} \mean{\hat{b}^{\dagger}_i \hat{b}_j }$ that characterizes the coherence properties of the system. A numerical simulation of $\rho_{\rm{TOF}}(\bm{r},t)$ is plotted on Fig.-\ref{fig:sim_expansion} at various expansion time for $s=5$ corresponding to the superfluid phase, illustrating how the interference pattern develops in time.

\begin{figure}
    \centering
    \includegraphics[width=0.7\textwidth]{Fig/Chapter2/TOF_expansion.png}
    \caption{Numerical simulation of the atomic density $\rho_{\rm{TOF}}(x,t)$ after various expansion times from a 1D lattice of 50 sites with $s=5$.}
    \label{fig:sim_expansion}
\end{figure}

\subsubsection{Relation to the momentum distribution}

The main purpose of the TOF technique is to obtain information on the momentum distribution of the gas. To this end, we must find the relation between the measured quantity $\rho_{\rm{TOF}}(\bm{r},t)$ and the in-trap momentum distribution $\rho(\bm{k})$. To do so, we introduce the operator $\hat{a}_{\bm{k}}$ destroying a particle in mode $\bm{k}$:

\begin{equation}
    \hat{a}_{\bm{k}}=\frac{1}{\sqrt{V}} \sum_{j} e^{i \bm{k} \cdot \bm{r}_{j}} \hat{b}_{j}
\end{equation}

\noindent where $V$ is the quantization volume set to be the in-trap volume of the gas. The momentum density then writes:

\begin{equation}
    \rho(\bm{k})=\left\langle\hat{a}^{\dagger}_{\bm{k}} \hat{a}_{\bm{k}}\right\rangle=\frac{1}{V} \sum_{j, i} e^{-i \bm{k} \cdot\left(\bm{r}_{j}-\bm{r}_{i}\right)}\left\langle\hat{b}_{i}^{\dagger} \hat{b}_{j}\right\rangle
    \label{eq:momentum_distribution}
\end{equation}

\noindent If the particle are non-interacting, the ballistic relation gives $\bm{k} = m \bm{r}/\hbar t_{\rm{TOF}}$. From equation \ref{eq:rho_tof}, we obtain:

\begin{equation}
    \rho(\bm{k}) = \frac{\rho_{\rm{TOF}}(\bm{r}=\hbar t\bm{k}/m,t)}{V  \left(\frac{m}{\hbar t} \right)^3 |w_0(\bm{k})|^2}
\end{equation}

In conclusion, under the conditions that there are no interactions during the TOF and $t_{\rm{TOF}} \gg t_{\rm{FF}}$ to be in the far-field regime, the TOF distribution maps the in-trap momentum distribution. 

\subsubsection{Momentum distribution across the Mott transition}

As we have just seen, the momentum distribution is strongly dependent from the coherence properties of the system and in turn of the lattice depth.

\begin{itemize}
    \item In the superfluid phase $U/J \to 0$, the system is coherent $G^{(1)}(i,j) = \bar{n}$. From equation \ref{eq:momentum_distribution}, we get that the momentum distribution consists of sharp analog diffraction peaks located at $\bm{k} = j k_d \bm{e}_i$ with $j \in \Z$ and $\bm{e}_i$ the unitary vector in direction $i=x,y,z$. In terms of $\rho_{\rm{TOF}}(\bm{r},t)$, the amplitude of the different peaks is set by the Fourier transform of the Wannier function. 
    
    \item In the Mott insulator phase $U/J \to \infty$ the system is totally incoherent $G^{(1)}(i,j) = \delta_{i,j}$. The momentum distribution is then constant, meaning that $\rho_{\rm{TOF}}(\bm{r},t)$ simply reflects the Fourier transform of the Wannier function, \ie a Gaussian-like function.
    
    \item For intermediate values of $U/J$, the visibility of the interference pattern progressively decreases as $U/J$ increases. In addition, as $V_0$ increases, the Wannier function is more and more localized so that the width of its Fourier transform increases. As a result, the population of the diffracted peaks increases with the lattice depth.
\end{itemize}

The experimental quantity $\rho_{\rm{TOF}}(\bm{r},t)$ is therefore a powerful tool to characterize the phase of the system across the Mott transition. This is illustrated on Fig-\ref{fig:mott_greiner} of the first experimental observation of the Mott transition with cold atoms \cite{greiner2002quantum}, on which we can clearly see the visibility decreasing with $V_0$.

Importantly, a characteristic of the Superfluid to Mott insulator transition is that even though the system is incoherent in the insulating phase, the coherence can be restored by ramping down the lattice depth to the superfluid phase. The standard procedure to characterize the presence of the Mott transition is then to set $V_0$ to be in the superfluid region, ramp up it up to see that the interference pattern disappear, and ramp it down to find back the interference pattern. This allows to certify that the loss of coherence is indeed an effect of the competition between $U$ and $J$ and not an experimental defect, such as unwanted heating of the cloud. 

\begin{figure}
    \centering
    \includegraphics[width=0.9\textwidth]{Fig/Chapter2/mott_greiner.png}
    \caption{Absorption images of Rubidium atoms taken $15 \ \rm{ms}$ after the atoms are released from a 3D cubic lattice. Note that the far-field regime condition is not fulfilled here. \textbf{a)} s=0. \textbf{b)} s=3. \textbf{c)} s=7. \textbf{d)} s=10. \textbf{e)} s=13. \textbf{f)} s=14. \textbf{g)} s=16. \textbf{h)} 20. Taken from \cite{greiner2002quantum}.}
    \label{fig:mott_greiner}
\end{figure}


\subsection{Mean-field interactions}

The experimental technique described in the last paragraph holds if there are no interactions between the particles during the TOF. If it were otherwise, the interactions would affect the TOF dynamics of the gas, preventing us to use the ballistic relation and to map the TOF distribution $\rho_{\rm{TOF}}(\bm{r},t)$ to the in-trap momentum $\rho(\bm{k})$. As interactions cannot be effectively turned off by means of a Feshbach resonance, experimentally inaccessible for $^4 \He$, it is crucial to determine whether interactions during the TOF can be neglected or not and under which conditions.

Describing the interactions between each of the individual particles would be impossible, even for numerical methods for which the calculation time would be prohibitive. To circumvent this issue, the problem can be simplified using the \textbf{mean-field approximation}. The idea is to approximate the effect of the interactions of every atoms on a single one as an \textbf{average} single effect. This considerably simplifies the many-body problem, reducing it to an effective one-body problem.

To quantify the effect of the interactions treated at the mean-field level, we introduce the interaction energy $U_{\rm{int}} \sim gn$ with $g$ the strength of the interactions and $n$ the atomic density. To determine whether the interactions are affecting the expansion of the gas released from the lattice, $U_{\rm{int}}$ must be compared to the zero point energy of the ground-state of the approximate harmonic oscillator associated to a single lattice site, $\hbar \omega_L$. In typical experimental conditions, $U_{\rm{int}}/h \approx 10^3 \ \rm{Hz} \ll \omega_L/2 \pi \approx 10^5-10^6 \ \rm{Hz}$, meaning that the initial expansion is driven by the zero point energy of the lattice and not the released interaction energy. After a small expansion time, the Wannier functions of the different sites overlap as their width become of the order of the lattice spacing $d$ and might then interact. However, on this time scale, the atomic density is reduced by a factor $(x_0/d)^3 \approx 10^2$ (for $s=10$), meaning that this interaction effect should also negligible \cite{gerbier2008expansion}, provided that the initial density is not too high. This point will be discussed in further details in light of experimental data in Chapter \ref{sec:chapter_3}.

As developed in \cite{kupferschmidt2010role}, the interactions can also induce a dephasing between the different sites. As a matter of fact, the time evolution of the phase of the wave-function associated to a lattice site depends on its initial energy, and therefore of $U_{\rm{int}}$. If the lattice sites have different lattice fillings, which typically is the case in the superfluid phase where the on-site atom number fluctuations are large, the different interfering wave-functions can be dephased from one another, reducing the visibility of the interference pattern. This effect is however also negligible \cite{gerbier2008expansion,kupferschmidt2010role} in the case of our 3D lattice where $a_s \ll x_0$, provided that the filling is not too high.

\subsection{Beyond mean-field interactions}

While the mean-field approximation is efficient to obtain a first understanding on how the interactions might affect the TOF, it is inherently limited as it does not consider interaction effects between several particles. One clearly identifiable beyond mean-field effect happening during the TOF is the presence of scattering halos between the diffraction peak \cite{greiner2001exploring}. These scattering halos signal the presence of $s$-wave collisions between the atoms of the different diffraction peaks, \ie with significantly different velocities, happening during the first moments of the TOF as they separate. This effect is analog to the scattering halos observed between two colliding condensates \cite{khakimov2016ghost,perrin2007observation,zin2006elastic}.

We have conducted a thorough experimental study of the $s$-wave two-body collisions during the TOF to determine whether they would affect our measurement of the momentum distribution \cite{tenart2020two}. This study will be detailed in Chapter \ref{sec:chapter_3}.

\section{Extension of the Bogoliubov theory to lattice gases}

So far, we have focused on the description of the lattice Hamiltonian under the scope of Wannier functions, culminating in the Bose-Hubbard Hamiltonian describing the physics of the Mott transition. We now wish to go back to the central point developed in Chapter \ref{sec:chapter_1}, namely the \kmk correlations in the quantum depletion predicted by the Bogoliubov theory of the weakly-interacting, homogeneous Bose gas. We concluded by saying that using an optical lattice would be a fitting solution to efficiently increase the interactions as a means to reach the low temperature regime dominated by the interactions $\mu \gg \kB T$ for which we expect the pair correlation signal to be experimentally detectable. This however requires that we extend the method of the Bogoliubov theory to the case of the lattice gas to identify the condition under which the \kmk pairs should be observable.

\subsection{Effect of the lattice amplitude}

As developed in \ref{sec:bogo_approx}, the central point of the Bogoliubov approximation for the weakly-interacting homogeneous Bose gas resides in the fact that the interactions are \textbf{weak}. This means that the system can be described as a BEC from which only a \textbf{small fraction} of the atoms, the depletion, are removed by the effect of interactions.

For the lattice gas, the condensed atoms correspond to the sharp diffraction peaks of the momentum distribution, while the depleted atoms in high quasi-momentum states correspond to the diffuse background between the diffraction peaks that increases as $U/J$ and therefore the strength of interactions increases, as illustrated by Fig.-\ref{fig:mott_greiner}. We thus need to be in the shallow lattice regime to use the Bogoliubov approximation, \ie at low values of the lattice depth such as $U/J \ll (U/J)_c$. This corresponds to the superfluid phase where the condensed fraction is close to 1 and the fraction of depleted atoms is small.


\subsection{Dispersion relation and effective mass}

For the remainder of this section, we will assume that we are in the shallow lattice regime so that the Bogoliubov approximation can be used. To begin, we remind the Hamiltonian for weakly interacting atoms without the lattice potential, as first introduced in Chapter \ref{sec:chapter_1}:

\begin{equation}
    \hat{H}=\sum_{\bm{k}}\frac{\hbar^2 k^2}{2m} \hat{a}^{\dagger}_{\bm{k}}  \hat{a}_{\bm{k}} +  \frac{g}{2V} \sum_{\bm{k}_1,\bm{k}_2,\bm{k}_3} \hat{a}^{\dagger}_{\bm{k_1}+\bm{k_3}} \hat{a}^{\dagger}_{\bm{k_2}-\bm{k_3}} \hat{a}_{\bm{k_1}} \hat{a}_{\bm{k_2}} 
\end{equation}

We now add the lattice potential but still neglect the external harmonic trapping potential and write the new Hamiltonian in the Bloch wave basis. In fact, we have already seen the expression of the non-interacting term in the Bloch wave basis in equation \ref{eq:H_bloch}. As we have done throughout this chapter, we consider that only the lowest energy band $n=0$ is populated. The full Hamiltonian writes \cite{dalibard2013cages}:

\begin{equation}
    \hat{H}=\sum_{\bm{q}} E_{0}(\bm{q}) \hat{c}_{\bm{q}}^{\dagger} \hat{c}_{\bm{q}}+\frac{g}{2} \sum_{\bm{q}_{1}, \bm{q}_{2}, \bm{q}_{1}^{\prime}, \bm{q}_{2}^{\prime}} C\left(\bm{q}_{1}, \bm{q}_{2}, \bm{q}_{1}^{\prime}, \bm{q}_{2}^{\prime}\right) c_{\bm{q}_{1}^{\prime}}^{\dagger} c_{\bm{q}_{2}^{\prime}}^{\dagger} c_{\bm{q}_{2}} c_{\bm{q}_{1}}
    \label{eq:lattice_bogo_hamiltonian}
\end{equation}

\noindent with $\hat{c}_{\bm{q}}$ the operator destroying a particle in the Bloch wave $\psi_{0,q}$ and 


\begin{equation}
    C\left(\bm{q}_{1}, \bm{q}_{2}, \bm{q}_{1}^{\prime}, \bm{q}_{2}^{\prime}\right)=\int_{0}^{V} \psi_{0, \bm{q}_{1}^{\prime}}^{*}(\bm{r}) \psi_{0, \bm{q}_{2}^{\prime}}^{*}(\bm{r}) \psi_{0, \bm{q}_{1}}(\bm{r}) \psi_{0, \bm{\bm{q}}_{2}}(\bm{r}) \mathrm{d} \bm{r}
\end{equation}

The non-interacting term $\sum_{\bm{q}} E_{0}(\bm{q}) \hat{c}_{\bm{q}}^{\dagger} \hat{c}_{\bm{q}}$ is quite similar to its equivalent in the homogeneous case $\sum_{\bm{k}}\frac{\hbar^2 k^2}{2m} \hat{a}^{\dagger}_{\bm{k}}  \hat{a}_{\bm{k}}$. In fact, as we can see on Fig-\ref{fig:dispersion_relation_harmonic}, the function $E_{0}(\bm{q})$ can be well approximated by a parabolic function at low values of $q$. We then rewrite $E_{0}(\bm{q})$ as:

\begin{equation}
    E_{0}(\bm{q}) \approx \frac{\hbar^2 q^2}{2 m^*}, \ \text{with } \frac{1}{m^*} = \frac{1}{\hbar^2} \frac{\mathrm{d}^2 E_0 }{\mathrm{d}q^2}
\end{equation}

\noindent where we have introduced the notion of effective mass $m^*$ defined from the curvature of the Bloch energy band, actually very useful to study the dynamics of particles in a lattice potential \cite{kramer2002macroscopic,dalibard2013cages}. Under this form, the non-interacting term of the lattice Hamiltonian is then of the exact same form as for the homogeneous Hamiltonian, the effect of the lattice being contained in the new effective mass $m^*$.

\begin{figure}
    \centering
    \includegraphics[width=0.5\textwidth]{Fig/Chapter2/dispersion_relation_harmonic.png}
    \caption{Harmonic approximation of the dispersion relation of the first energy band $E_0$ for $s=5$.}
    \label{fig:dispersion_relation_harmonic}
\end{figure}

\subsection{The rescaled interaction strength}

\label{sec:rescaled_interaction}

We turn to calculating the interaction term in equation \ref{eq:lattice_bogo_hamiltonian}. First, like in the homogeneous case, the conservation of momentum gives the relation:

\begin{equation}
    \bm{q_1} + \bm{q_2} = \bm{q}'_1 + \bm{q}'_2
\end{equation}

\noindent allowing us to reduce the sum on $\bm{q}_1,\bm{q}_2, \bm{q}'_1, \bm{q}'_2$ to a sum on three quasi-momenta $\bm{q}_1,\bm{q}_2$ and $\bm{q}_3$. Using the Bloch function form $\psi_{n,q} (x)= e^{iqx} u_{n,q} (x)$ and assuming that only the states at the bottom of the band are populated, we can finally approximate the Hamiltonian \ref{eq:lattice_bogo_hamiltonian} to \cite{dalibard2013cages}:

\begin{equation}
    \hat{H} \approx \sum_{q} \frac{\hbar^{2} q^{2}}{2 m^{*}} \hat{c}_{q}^{\dagger} \hat{c}_{q}+\frac{g^{\prime}}{2 V} \sum_{\bm{q_{1}}, \bm{q_{2}}, \bm{q_{3}}} \hat{c}^{\dagger}_{\bm{q_1}+\bm{q_3}} \hat{c}^{\dagger}_{\bm{q_2}-\bm{q_3}} \hat{c}_{\bm{q_1}} \hat{c}_{\bm{q_2}} 
\end{equation}

\noindent with $g'$ the rescaled interaction strength defined as:

\begin{equation}
    g' = g \left(d \int_0^d |u_{0,0} (x)|^4 \mathrm{d}x \right)^3
\end{equation}

\noindent As long as $V_0 > 0,$ we have $g' > g$ signaling that the lattice indeed increases the strength of the interactions. Importantly, $g'$ increases with $V_0$ as the atoms become increasingly localized in smaller region of spaces, increasing the strength of the interactions.

In conclusion, we obtain an Hamiltonian of the same form than the homogeneous case, with two notable differences:

\begin{itemize}
    \item We have replaced the mass $m$ by the effective mass $m^*$ in the dispersion relation.
    \item The interaction strength $g$ has been replaced by the rescaled and higher interaction strength $g'$.
\end{itemize}

We have then achieved the objective set at the beginning of this chapter, namely obtain a system with \kmk paired quantum depleted atoms as described by the Bogoliubov theory of the weakly-interacting Bose gas, but with increased interactions so that we should be able to reach the low temperature regime dominated by interactions $\kB T \ll \mu$ for which the \kmk correlation signal should be detectable. 

Importantly, the predictions of the Bogoliubov theory detailled in Chapter \ref{sec:chapter_1} should not be taken at face value for the much more complicated system of the inhomogeneous lattice gas. In fact, there have not been clear experimental studies testing the validity of the Bogoliubov approach for this kind of system. It will then be of great interest to compare the predictions of the simple homogeneous case to the experimental data as a means to detect whether the Bogoliubov approach fails or not and under which conditions.

\chapter{Single-atom resolved momentum measurement of lattice Bose gas}

blah blah blah

\section{Metastable Helium}

Metastable Helium, noted $\mathrm{He}^*$, is kind of an odd atom in the ensemble of species that we know how to bring to quantum degeneracy. Its most important feature, which is actually the reason why we chose this atom to measure correlation functions in momentum space, is the existence of the metastable state $2 \ ^3 S_1$. This excited state is called metastable for its very long lifetime of the order of $8,000 \ \mathrm{s}$, far larger than what is required for experiments. Very interestingly, as Helium is a noble gas, the amount of energy required to excite Helium into its metastable state is quite large, $19.8 \ \rm{eV}$. This large energy is sufficient for a metastable Helium atom to extract an electron from an electronic surface. This opens the way for \textbf{electronic detection} techniques, in opposition to the much more widely spread optical detection techniques, that allows for \textbf{single atom detection} as we will see in this chapter. In addition, the energy level structure (see Fig.-\ref{fig:niveaux}) is well adapted to laser cooling with a transition in the near-infrared around $\lambda_0 \simeq 1083 \ \mathrm{nm}$ for which reliable laser sources are available. Metastable Helium was actually amongst the first atoms to be brought to quantum degeneracy with the first BEC of $\He$ being obtained in 2001 simultaneously at the Institut d'Optique and Laboratoire Kastler Brossel in France. Helium also has the advantage to have a stable, albeit very rare and expensive fermionic isotope $^3 \He$ that has also been brought to quantum degeneracy at the Amsterdam LaserLab in 2006.

In spite of all these advantages, $\He$ comes with a few experimental difficulties that explain why they are actually quite few $\He$ experiments over the world. First, Helium is a very light atom and this comes with some very practical difficulties like the need for pre-cooling with liquid nitrogen and quite long Zeeman slowers. Second, $\He$ is subject to Penning collisions that bring back an atom to the ground state to ionize the other:

\begin{equation}
    \He + \He \rightarrow \mathrm{He} + \mathrm{He}^+ + \mathrm{e}^{-}
\end{equation}

Such a reaction thus results in the loss of two atoms and must be avoided. \NOTE{FINIR}

\begin{figure}
    \centering
    \includegraphics[width=0.9\textwidth]{Fig/Chapter3/niveaux.png}
    \caption{Energy levels of the Helium atom. The metastable state is the triplet state $2 \ ^3S_1$ that we will call the ground state of the metastable Helium atom. Laser cooling is performed on the optical transition $2 \ ^3S_1 \rightarrow 2 \ ^3 P$ of wavelength $\lambda_0 \simeq 1083 \ \mathrm{nm}$. More specifically, we address the transition to  $2 \ ^3 P_2$ or $2 \ ^3 P_1$ depending on the cooling scheme (respectively Doppler and sub-Doppler), as well as the transition to  $2 \ ^3 P_0$ to perform two-photon Raman transfer as we will detail later on.}
    \label{fig:niveaux}
\end{figure}

In the following, we will detail the different experimental steps used to bring a gas of metastable Helium to quantum degeneracy.

\section{Bose-Einstein Condensation of metastable helium}

\subsection{The source}

As the energy difference between the ground-state and the metastable state is very large, it is impossible to excite the atoms optically.

\begin{figure}
    \centering
    \includegraphics[width=0.9\textwidth]{Fig/Chapter3/source.png}
    \caption{Picture of the source apparatus.}
    \label{fig:my_label}
\end{figure}

\subsection{3D optical lattice}

\subsection{Electronic detection: The Micro Channel Plate Detector}

\section{Two-photon Raman transfer}

\subsection{Principle of the two-photon Raman transfer}

\subsection{Experimental implementation}

\section{Characterisation of two-body collisions in the time-of-flight dynamics}

\subsection{Classical model}

\subsection{Evolution with total atom number}

\subsection{Evolution with lattice depth}

\subsection{Conclusion}

\section{Adabiatic preparation in the vicinity of the Mott transition}

\subsection{Thermometry method}

\subsection{Fischer information and Cramér-Rao bound}

\subsection{Entropy measurement}


\chapter{Experimental observation of k/-k correlations in the depletion of a weakly-interacting Bose gas}

The Bogoliubov interacting Bose-gas has been the subject of a large variety of experimental studies (CITATIONS). However, these experiments have mainly focused on measuring the Bogoliubov spectrum of excitations. More than 60 years after the seminal Lee, Huang and Yang paper \cite{lee1957}, an experimental study of the correlations in the many-body ground-state is yet to be done. As we have seen thus far, our experimental setup is perfectly suited for such an investigation. 

We will detail in this chapter the details of our experiment aiming at detecting \kmk pairs in the depletion of a weakly-interacting Bose gas and our main results. COMPLETER AVEC PLAN?

\section{Numerical procedure to measure two-body correlations}

Our goal is to compute the normalized second-order correlation function:

\begin{equation}
    g^{(2)} (\bm{k},\bm{k'}) = \frac{\mean{\hat{a}^{\dagger}(\bm{k}) \hat{a}^{\dagger}(\bm{k'}) \hat{a}(\bm{k}) \hat{a}(\bm{k'})}}{\mean{\hat{a}^{\dagger}(\bm{k}) \hat{a}(\bm{k}) } \mean{\hat{a}^{\dagger}(\bm{k'}) \hat{a}(\bm{k'}) }}
\end{equation}

We recognize that $\mean{\hat{a}^{\dagger}(\bm{k}) \hat{a}(\bm{k})}$ is the momentum density in mode $\bm{k}$ that we will write $\rho(\bm{k})$ in the following. As seen on the previous chapter, the $\gtwo$ function will take values different from 1 in two cases:

\begin{itemize}
    \item For $\bm{k'} \simeq \bm{k}$, the \textbf{normal} correlations corresponding to the Hanbury Brown and Twiss effect also known as bosonic bunching.
    \item For $\bm{k'} \simeq -\bm{k}$, the \textbf{anomalous} correlations corresponding to the \kmk pairs in the quantum depletion.
\end{itemize}

In practice, plotting the $\gtwo$ function defined as such does not make much sense. On the one hand, the function here is 6D and thus hard to plot in an intelligible way. On the other hand, obtaining a sufficient signal to noise ratio to make correlation measurements for single values of $\bm{k}$ and $\bm{k'}$ is impossible. The idea is then to integrate the $\gtwo$ over all momenta $\bm{k}$ and introduce a new parameter $\delta \bm{k}$ to write:

\begin{equation}
    g_{N,A}^{(2)} (\delta {\bm k})=\frac{\int_{\Omega_{k}} \langle \hat{a}^{\dagger}({\bm k}) \hat{a}^{\dagger}(\delta {\bm k} \pm {\bm k}) \hat{a}({\bm k}) \hat{a}(\delta {\bm k} \pm {\bm k}) \rangle \mathrm{d}{\bm k}}{\int_{\Omega_{k}} \rho({\bm k}) \rho(\delta {\bm k} \pm {\bm k}) \mathrm{d}\bm{k}},
    \label{Eq:g2}
\end{equation}

With this definition, we see that for $\delta \bm{k}=0$, we are either looking at \textbf{normal} \kk correlations (subscript N) when chosing the plus sign, or \textbf{anomalous} \kmk correlations (subscript A) with the minus sign. We reduced the 6D function to a 3D function of $\delta \bm{k}$ which equals $\bm{0}$ when the correlation condition $\bm{k'} = \pm \bm{k}$ is fulfilled. This gives us a natural way to evaluate $\gtwo (\delta \bm{k})$ with the experimental data: we compute the values of the parameter $\delta \bm{k}$ for every detected atom pairs in an experimental run by calculating their momentum difference or sum, for normal and anomalous correlations respectively. By computing the histogram of these values and averaging over many experimental runs, we evaluate the numerator of equation \ref{Eq:g2}.

\subsection{Description of the algorithm}

The algorithm described here is similar to the one used in our previous work \cite{carcy2019momentum,cayla2020} and detailed in \cite{cayla_these,carcy_these}. This previous version was mainly designed for the observation of bosonic bunching. I adapted the algorithm to make it suitable for the calculation of \kmk correlations as well, as we will discuss now.

\subsubsection{Numerator calculation}

The first step is to compute the numerator of equation \ref{Eq:g2} that we denote $G^{(2)}(\delta \bm{k})$. We note $N_{runs}$ the number of experimental runs and $N_{i}$ the number of atoms in the $i$-th shot. The procedure is as follows:

% \begin{algorithm}
% \caption{$G^{(2)}$ calculation}
%     \begin{algorithmic}{}
%         \FOR{$i=1:N_{runs}$}
%             \FOR{$j=1:N_{i}$}
%                 \STATE Compute $\vec{k}_i+\vec{k}_j$
%                 \STATE Increment histogram $G^{(2)}$ corresponding pixel
%             \ENDFOR
%         \ENDFOR
%     \end{algorithmic}{}
% \end{algorithm}

\begin{algorithm}
 \caption{$G^{(2)}$ calculation}
    \begin{algorithmic}
         \For{$i=1:N_{runs}$}
            \For{$j=1:N_{i}$}
               \For{$p=1:N_{i}$}
                    \State Compute $\delta \bm{k} = \bm{k}_j \pm \bm{k}_p$
                    \State Increment 3D histogram $G^{(2)}$ corresponding pixel
                \EndFor
            \EndFor
        \EndFor
\end{algorithmic}

\end{algorithm}

We end up with a 3D histogram where each voxel is associated to a value of $\delta \bm{k} = (\delta k_x,\delta k_y, \delta k_z)$ and records how many atom pairs have this specific momentum sum or difference, depending on the kind of the correlations we want to probe.

The major difference with the previous version of the algorithm is that we record here the full 3D histogram of calculated $\delta \bm{k}$ on every pair of atoms. The procedure was made originally made simpler by calculating three one-dimensional histograms, one for each direction space. For instance, the $x$ direction histogram was obtained by selecting one atom and calculating $\delta k_x$ only for atoms with $k_y$ and $k_z$ close to the one of the considered atom (\textcolor{red}{NUL}). This method obviously saves computing time and RAM space, but is not suited to computing \kmk correlations.

At this point, we record in the central voxels associated to $\delta \bm{k} \simeq \bm{0}$ what we call \textbf{true coincidences}, namely two atoms detected conjointly as a result of \kmk pairing or bosonic bunching. However, we also record \textbf{accidental coincidences} that do not represent correlations but result from the distribution of the atoms. We then need a normalization process to get rid of the contribution of accidental coincidences. 

\subsubsection{Denominator calculation}

We now want to compute the denominator of equation \ref{Eq:g2}, representing the effect of accidental coincidences. To perform this calculation, we would like to have a sample of atoms with the same momentum density than our experimental data but uncorrelated. This can be done by merging all experimental shots, not correlated with one another. We then apply the procedure we have just described to this data set. However, some of the correlations happening in single shots will remain in this large file. In the end, the total number of correlations in the numerator is $\sum_i N_i^2$ whereas the number of coincidences in the merged file is $(\sum_i N_i)^2$. If we note $\bar{N}$ the mean number of atoms per shot, we see that :

\begin{equation}
      \frac{\sum_i N_i^2}{(\sum_i N_i)^2}=\frac{N_{\rm{runs}} \bar{N}^2}{N_{\rm{runs}}^2 \bar{N}^2}=\frac{1}{N_{\rm{runs}}}
\end{equation}{}

Therefore, with enough shots, the contribution of residual coincidences in the denominator is negligible. 

In the end, the integrated $g^{(2)}$ function can obtained by dividing the calculated numerator by the denominator and multiplying by the normalization factor $\frac{(\sum_i N_i)^2}{\sum_i N_i^2}$ taking into account the number of coincidences of the numerator and denominator. Usually, we only take a fraction of all atoms for the denominator calculation to avoid large computation time. 

\subsection{Benchmarking of the algorithm with two-body collision spheres}

Before using the algorithm to look for a supposedly small \kmk pairing signal in the depletion of a weakly-interacting Bose gas, it was crucial to test it on a data set with a large number of \kmk pairs to certify that it was working properly. Luckily, we could re-use the data taken for measuring two-body collisions during the time-of flight described in Chapter 2. The simplest case is to use only one of the lattice beam to induce 1D diffraction and observe only two collision spheres. FINIR


\section{Accessing the BEC depletion}

In order to detect \kmk pairs in the depletion, it is absolutely crucial to remove from the analysis all atoms belonging to the BEC and its diffracted copies showing no correlations as it is a pure coherent state. These atoms indeed largely outnumber the depleted atoms and will therefore entirely drown out the \kmk correlation signal of the quantum depletion, would they not be removed. 

For each record data set, we begin the analysis by removing all atoms outside the volume $\Omega_k$ that we design to exclude momentum regions with condensed atoms. 



\section{k/-k correlation signal}

\section{Effect of temperature}

\section{Study of the amplitude of the correlation peaks}

\subsection{Transverse integration effects}

\subsection{Scaling with momentum density}

\subsection{Discussion on the detected number of atom pairs}

\section{Study of the width of the correlation peaks}

\section{Relative number squeezing}

\chapter{Towards measuring Tan's contact in 1D gases}

\section{Tan's contact}

\subsection{Definition from the large momentum tails}

To understand what Tan's contact is, we consider two atoms with contact interactions in the ultracold regime in 1D. The two-body wave-function then only depends from the inter-particle distance $r$ and the scattering length $a_s$ \cite{viverit2004momentum}:

\begin{equation}
    \psi(r) = - \frac{r}{a_s} \, e^{-r/a_s}
\end{equation}

\noindent The Fourier transform of this expression is rather easy to compute and writes:

\begin{equation}
    \tilde{\psi}(k) = \int_0^\infty \psi(r) \, e^{2 \pi i k r} \, dr \propto \frac{1}{(i 2 \pi k - 1/a_s)^2}
\end{equation}

\noindent from which we obtain the momentum distribution:

\begin{equation}
     n(k) = |\tilde{\psi}(k) |^2 \propto \frac{1}{2(2 \pi k a_s)^4 - 2(2 \pi k a_s)^2 +1}
\end{equation}

\noindent Interestingly, if we look at the asymptotic behavior at high $k$ we find that:

\begin{equation}
    n(k) \underrel{k \to \infty} \to \frac{1}{k^4}
    \label{eq:kmf_scaling}
\end{equation}

\noindent Importantly, this signature of contact interactions in the momentum distribution holds for higher dimensions, independently of temperature, interaction strength or quantum statistics making it a \textbf{universal relation}. From equation \ref{eq:kmf_scaling}, we define the Tan's Contact $C$ as:

\begin{equation}
    C = \lim_{k \to \infty} k^4 n(k)
\end{equation}

\subsection{Connection to thermodynamic quantities}

While the $\kmf$ scaling is universal, the value of $C$ depends on the physical characteristics of the system such as the number of particles, temperature, dimension etc. and thus contain meaningful information that would otherwise be hard to measure with standard experimental techniques. This was first theorized by Shina Tan in 2008 \cite{tan2008large} who showed that $C$ is a thermodynamic quantity revealing how the total energy of a two component Fermi gas changes when adiabatically tuning the inverse scattering length $a_s$:

\begin{equation}
    -\frac{d E}{d(1 / a_s)}=\frac{h^{2} C}{2 \pi m}
\end{equation}

\noindent This result is known as Tan's adiabatic sweep theorem and can be adapted to the 1D bosonic case \cite{barth2011tan} to obtain:

\begin{equation}
    C=\left.\frac{4 m}{\hbar^{2}} \frac{\partial \Omega}{\partial a_{1 D}}\right|_{T, \mu}
\end{equation}

\noindent with $\Omega$ the grand potential. This result can also be rewritten to include the interaction energy of the system $\mean{H_{\rm{int}}}$ that we have encountered quite a lot throughout this thesis and is usually hard to measure separately from the total energy:

\begin{equation}
    C=\frac{2 g m^{2}}{\hbar^{4}}\left\langle H_{\text {int }}\right\rangle
    \label{eq:C_with_int}
\end{equation}

\subsection{Characterization of 1D Lieb-Liniger regimes}

Another significant motivation to measure Tan's contact is to characterize the different regimes of Lieb-Liniger 1D systems of interacting bosons as a function of temperature and the strength of the interactions. Tan's contact is indeed particularly suited to study The Lieb-Liniger model as it revolves around the approximation that the interactions between the atoms are repulsive, contact interactions:

\begin{equation}
    H = \sum_j \left[ - \frac{ \hbar^2}{2m} \frac{\partial^2}{\partial^2 {x_j}^2} + V(x_j) \right] + g \sum_{j<l} \delta(x_j - x_l) 
\end{equation}

\noindent with $g$ the strength of the interactions as we have seen on a few occasions in this thesis and $V$ an external trapping potential. 


These different regimes have been widely investigated \cite{petrov2000regimes} and are illustrated on the state diagram of Fig.\ref{fig:1D_diagram}. The lower right part of the diagram corresponds to the strongly interacting or Tonks-Girardeau regime where the repulsive interactions are so strong that they mimic the Pauli exclusion principle for fermions, the gas is then said to fermionize. As the strength of the interactions decreases, the gas progressively goes to a weakly-interacting quasi-condensate phase characterized by suppressed density fluctuations but fluctuating phase, contrary to the true condensate. The nearly ideal gas region refers to the region where the effect of interactions are negligible compared to temperature. \NOTE{voir definitions gamma et t}.

The main difficulty to experimentally characterize those regimes resides in the fact that most quantities show a smooth and monotonic behavior when crossing the transition points between the different regimes. This motivated theoretical studies of the dependency of the Tan's contact with the strength of interactions and temperature to determine whether $C$ constitutes a good probe or not. Previous works have conducted such studies for homogeneous bosons at finite temperature \cite{kheruntsyan2003pair,kormos2009expectation}, trapped bosons at zero temperature \cite{minguzzi2002high,olshanii2003short} or for the trapped finite temperature Tonks-Girardeau regime \cite{vignolo2013universal}. These works were recently completed by \cite{yao2018tan}, characterizing trapped Lieb-Liniger bosons for arbitrary values of the temperature and the interaction parameter.

One of the main experimental difficulties for measuring Tan's contact comes from the fact that the high momentum $\kmf$ tails correspond to very low density values that are hard to detect with classic optical imaging techniques. This problem is however solved with the $\He$ detector thanks to his large dynamic momentum range. Interestingly, our experimental apparatus can be adapted to study 1D physics by transforming our 3D optical lattice into a 2D one as we will see in \ref{sec:1D_exp}, making it the perfect candidate to verify the predictions of \cite{yao2018tan}.

\begin{figure}
    \centering
    \includegraphics[width=0.8\textwidth]{Fig/Chapter5/quant_degeneracy_1D_trapped.PNG}
    \caption{State diagram of trapped 1D Bose gases with repulsive interactions as a function of the reduced temperature $t$ and the interaction parameter $\gamma$. The dashed diagonal line separates the classical and quantum degenerate regimes, while the solid lines indicate smooth crossovers between the different regimes.}
    \label{fig:1D_diagram}
\end{figure}


\section{Theoretical study}

\label{sec:1D_theory}

Before going into the experimental details, we start our study by summarizing the main results of \cite{yao2018tan}.

\subsection{Two-parameter scaling}

At first glance, Tan's contact should depend from 4 parameters:

\begin{itemize}
    \item The total number of particles $N$.
    \item The temperature $T$.
    \item The trapping frequency $\omega_{\rm{1D}}$.
    \item The coupling constant $g$.
\end{itemize}

\noindent The first result of \cite{yao2018tan} is to show that $C$ actually depends from only two parameters, the first one being the reduced interaction strength:

\begin{equation}
    \xi_{\gamma}=-a_{\rm{ho}}/a_{\rm{1D}}\sqrt{N}
\end{equation}

\noindent with $a_{\rm{ho}}=\sqrt{\hbar/m \omega_{\rm{1D}}}$ the harmonic oscillator length and $a_{\rm{1D}}=a_{\rm{oh}}^2/a_s$ the 1D scattering length. The second one is the reduced temperature

\begin{equation}
    \xi_T=-a_{\rm{1D}}/\lambda_T
\end{equation}

\noindent with $\lambda_T=\sqrt{2\pi \hbar^2/m k_b T}$. The contact can then be written as a function of $\xi_{\gamma}$ and $\xi_{T}$:

\begin{equation}
    C= \frac{N^{5/2}}{a_{\rm{ho}}^3} f(\xi_{\gamma},\xi_{T})
\end{equation}

The goal is then to determine the variations of $f(\xi_{\gamma},\xi_{T})$. To do so, the authors of \cite{yao2018tan} follow two complementary approach. The first one consists in using the Bethe Ansatz which is the exact solution of the Yang-Yang equations \cite{yang1969thermodynamics} of the 1D homogeneous gas. The results are adapted to the trapped case by using the Local Density Approximation (LDA) as we did in \ref{sec:ch2_trapping_effects}. The validity of this approach is checked by comparing its prediction to {\it ab-initio} QCM calculations as shown on Fig-\ref{fig:C_theo}.

\begin{figure}
    \centering
    \includegraphics[width=0.9\textwidth]{Fig/Chapter5/BS_LDA_vs_Bethe.PNG}
    \caption{Reduced contact $a_{ho}^3 C_3 / N^{5/2}$ as a function of $\xi_T$ and $\xi_{\gamma}$ as predicted from the LDA approach (solid lines) and QMC calculations (points). The different symbols correspond to various parameters for the QMC calculations (see \cite{yao2018tan} for further details). (a) Reduced contact versus $\xi_{\gamma}$ at fixed temperatures corresponding to $\xi_T = 0.0085$ (blue), 0.28 (green), and 18.8 (red). (b) (b) Reduced contact versus $\xi_T$  at fixed interaction strengths corresponding to $\xi_\gamma = 10^{-2}$ (blue), $1.58 \times 10^{-1}$ (green), and 4.47 (red). The black dashed, red dotted, and red dash-dotted lines correspond to \NOTE{todo} respectively. }
    \label{fig:C_theo}
\end{figure}

\subsection{Maximum contact versus temperature}

A striking and unexpected feature of Fig.-\ref{fig:C_theo} panel (b) is that the contact shows a non-monotonous dependency with $\xi_T$ with a maximum, contrary to the monotonous increase in the Tonks-Girardeau regime predicted by \cite{vignolo2013universal}. While the maximum exists for any value of $\xi_{\gamma}$, the effect is more pronounced in the strongly-interacting regime. 

\subsubsection{Strongly-interacting regime}

In this regime, the contact can be determined analytically via a virial expansion \cite{vignolo2013universal} and writes:

\begin{equation}
    C = \frac{2 N^{5/2}}{\pi a_{ho}^3} \frac{\xi_\gamma}{\xi_T} \left( \sqrt{2} - \frac{e^{1/2 \pi \xi_T^2}}{\xi_T} \textrm{Erfc}(1/\sqrt{2 \pi} \xi_T) \right)
    \label{eq:C_strong_int}
\end{equation}

In the asymptotic regime of low-temperature limit $\xi_\gamma^{-1} \leq \xi_T \leq 1$, this expression simplifies to:

\begin{equation}
    C = 2 \sqrt{2} \, \frac{N^{5/2}}{a_{ho}^3} \, \xi_\gamma \, \xi_T 
    \label{eq:strong_int_lowT}
\end{equation}

\noindent In the opposite regime of high-temperature $\xi_\gamma^{-1}, \, 1 \leq \sqrt{\xi_T}$, we rather get:

\begin{equation}
    C \simeq 2 \sqrt{2} \, \frac{N^{5/2}}{ \pi a_{ho}^3} \, \frac{\xi_\gamma}{\xi_T} 
    \label{eq:BS_strong_int_highT}
\end{equation}

\noindent We thus clearly see the non-monotonic behavior of the contact. These 3 expressions are plotted in dashed line on Fig-\ref{fig:C_theo}. We see that the full analytical expression of \ref{eq:C_strong_int} (black dashed line) well matches the LDA predictions, except for low-temperatures for which the virial expansion is not suited.

The existence of a maximum value of the contact can be understood by the competition between the effect of temperature and interactions. While interaction dominates, the gas is fermionized and the contact increases with temperature \cite{vignolo2013universal}, whereas it decreases as thermal fluctuations take over and fermionization disappears. The location of the maximum of the contact thus provides a way to characterize the crossover to fermionization. 

\subsubsection{Weakly-interacting regime}

In the weakly-interacting regime, the interactions are not strong enough to fermionize the gas. In the low-temperature regime $1, \, \xi_T \leq \xi_\gamma^{-1}$, the gas forms a quasi-condensate and the contact is obtained from equation \ref{eq:C_with_int} with the mean-field expression of $H_{\text {int }}$ and writes:

\begin{equation}
    C = \eta \frac{N^{5/2}}{a_{ho}^3} \xi_\gamma^{5/3} 
    \label{eq:weak_int_lowT}
\end{equation}

\noindent with $\eta = 4 \times 3^{2/3}/5$. We see that $C$ does not depend from temperature here. At high temperatures $\xi_\gamma^{-1} \leq \xi_T \leq \xi_\gamma^{-2}$, interactions become negligible so that the gas is nearly ideal and the contact writes:

\begin{equation}
    C = \left( 16 \sqrt{\pi} \, \frac{N^{5/2}}{a_{ho}^3} \, \xi_\gamma^5 \, \xi_T^3 \right) G(\alpha) \, ,
    \label{eq:weak_int_highT}
\end{equation}

\noindent with $G(\alpha)$ decreasing at least in $\lambda_T^4$ (see \cite{yao2018tan} for the explicit expression), making $C$ decrease with temperature. Once again, identifying the temperature at which $C$ starts to decay allows to characterize the crossover between the quasi-condensate regime and the nearly ideal Bose gas regime.

\section{Experimental realisation of 1D gases with the optical lattice}

\label{sec:1D_exp}

Now that we have seen what Tan's contact is and how it could be used to characterize the regimes of Lieb-Liniger 1D gases, we will show how our experimental apparatus can be adapted to study 1D physics with the objective of testing experimentally the predictions of \cite{yao2018tan}. The main idea to obtain an  experimental 1D system is to ``freeze'' the degrees of freedom of the atoms in two directions of space. To do so, the easiest solution is to use a harmonic trapping potential with trapping frequencies $\omega_{\perp}$ high enough so that the energy difference $\Delta E =\hbar \omega_{\perp}$ between the ground-state and the first excited state is much larger than the typical energy of the atoms $\Delta E \gg \kB T, \mu$. Such high trapping frequencies are accessible in our experiment thanks to the optical lattice. Instead of using the 3 pairs of countra-propagating beam as we did so far, we use only 2 to produce a 2D lattice. Interestingly, the total laser power is divided amongst 2 pair of beams instead of 3, meaning that we can reach much higher values of the lattice depth, typically up to $s=30$. In the direction where there is no lattice, the trapping potential results from the Gaussian shape of the beams and has a trapping frequency $\omega_{\rm{1D}} =2 \pi \times 140 \sqrt{s} = 2 \pi  \times 713 \ \rm{Hz}$ for $s=26$. In the other 2 directions, the trapping frequency is however very large as a result of the lattice interference pattern $\omega_{\perp} \simeq 200 \ \rm{kHz}$, which is much larger than the energy of the atoms $\kB T, \mu \simeq 25 \ \rm{kHz}$ with typical experimental parameters.

\begin{figure}
    \centering
    \includegraphics[width=0.9\textwidth]{Fig/Chapter5/1D_config.png}
    \caption{Configuration of the optical lattice to produce 1D tubes. In the transverse direction, the lattice interference pattern creates a confining potential that can be approximated to a harmonic potential near the center of the site. The trapping frequency is high enough so that the degree of freedom of the atoms in these directions is ``frozen''. On the other hand, the lattice is absent in the longitudinal direction and the trapping frequency only results from the Gaussian shape of the beams. This is the 1D direction.}
    \label{fig:my_label}
\end{figure}


Using the optical lattice in this configuration then allows us to emulated 1D physics. The main drawback of this method is that we end up with an array of 1D gases rather than a single one, complicating the comparison with theory.


\subsection{Characterization of the 1D tubes}

\subsubsection{Number of atoms}

The major difficulty comes from the fact the atom number varies from one 1D tube to another. To determine the atom number distribution, we first need to determine the density profile of the cloud in the 2D lattice.

To do so, we first remind that under the Thomas-Fermi approximation (\NOTE{ref}), the density profile of a BEC in a 3D harmonic potential writes:

\begin{equation}
     n(\bm{r}) = \frac{\mu}{g} \, \left[ 1 - \left( \frac{x}{R_x} \right)^2 - \left( \frac{y}{R_y} \right)^2 - \left( \frac{z}{R_z} \right)^2 \right]
\end{equation}

\noindent where $R_i = \sqrt{\frac{2 \mu}{m \omega_i^2}}$ is the Thomas-Fermi radius in direction $i$. Under the mean-field approximation, the chemical potential is:

\begin{equation}
     \mu = \frac{\hbar \bar{\omega}}{2} \left(  15 N_{\rm{tot}} \frac{a_s}{a_{\rm{ho}}}\right)^{2/5}
\end{equation}

\noindent with $\bar{\omega}=\omega_x \omega_y \omega_z/3$ the average trapping frequency.

Similarly to the method developed in \ref{sec:rescaled_interaction}, we rescale $\mu$ to account for the presence of the 2D lattice with:

\begin{equation}
    \tilde{\mu} = \frac{\hbar \bar{\omega}}{2} \left(  15 N_{\rm{tot}} \frac{\tilde{a}_s}{a_{\rm{ho}}}\right)^{2/5}
\end{equation}

\noindent where $\tilde{a}_s= a_s \left(d \int_0^d |u_{0,0} (x)|^4 \mathrm{d}x \right)^2$ is the rescaled interaction strength, with the notable difference that we are using here a power 2 instead of power 3 in \ref{sec:rescaled_interaction} as we use here a 2D lattice. \NOTE{vérifier} We finally obtain the new Thomas-Fermi radius in the transverse directions

\begin{equation}
    R_{\rm{TF}} = \frac{1}{d} \sqrt{\frac{2 \tilde{\mu}}{m \omega_{\perp}^2}}
\end{equation}

\noindent that we express in units of lattice spacing $d$ for convenience. The number of atoms in the tube indexed $j,l$ then writes:

\begin{equation}\label{eq:N_per_tube}
    N_{j,l} = N_{00} \left( 1 - \frac{j^2 + l^2}{R_{\rm{TF}}^2} \right) .
\end{equation}

\noindent where $N_{00}$ is the number of atoms in the central tube. We deduce $N_{00}$ from the total atom number $\NBEC$ with the normalization condition $\NBEC = \sum_{j,l} N_{j,l}$ giving:

\begin{equation}
    N_{00} = \frac{5}{2 \pi} \frac{\NBEC}{R_{\rm{TF}^2}}
\end{equation}

\begin{figure}
    \centering
    \includegraphics[width=0.6\textwidth]{Fig/Chapter5/atomic_distrib_2Dlatt.PNG}
    \caption{Atom number distribution in a 2D lattice of amplitude $s=26$ for $\NBEC=30 \times 10^3$.}
    \label{fig:my_label}
\end{figure}

\begin{figure}
    \centering
    \includegraphics[width=0.45\textwidth]{Fig/Chapter5/tubes_occupes.png}
    \caption{Schematic of the array of 1D tubes. The large blue circle denotes the parabolic density profile of the BEC that determines which of the lattice sites contain atoms (blue dots).}
    \label{fig:my_label}
\end{figure}

\subsubsection{Density and interaction parameter}

Knowing the number of atom in each tube, we now look to determine the density and interaction parameter $\gamma$ in each of the tubes. To do so, we first introduce the effective 1D interaction strength $g_{\rm{1D}} \simeq 2 \hbar \omega_{\perp} a_s$ \cite{olshanii1998atomic} that depends from the transverse trapping frequency and the 3D scattering length. With this definition, we can write the 1D chemical potential and 1D density for the different tubes, both functions of the number of atoms in the tube:

\begin{equation}
    \mu_{1D}^{j,l} = \left( \frac{3}{4 \sqrt{2}} \, N_{j,l}\,  g_{1D} \, \omega_{1D}\,  \sqrt{m} \right)^{2/3}
\end{equation}

\begin{equation}\label{eq:n_1D}
    \rho_{1D}^{j,l}(x) = \frac{\mu_{1D}}{g_{1D}} - \frac{1}{2} \, m \, \omega_{1D}^2 \, x^2
\end{equation}

\noindent In practice, the second term of equation \ref{eq:n_1D} can be neglected because of the small size of the 1D gases $\sim \mu \rm{m}$ and the weak confinement $\omega_{1D} \approx 2 \pi \times 700 \rm{kHz}$ so that $\rho_{1D}^{j,l}(x)$ is constant and well approximated by its value at the center of the trap $\rho_{1D}^{j,l}(0)$. We then simply write $\rho_{1D}^{j,l}$.

Finally, we can write the interaction parameter $\gamma$ corresponding to the ratio between the interaction and the kinetic energy:

\begin{equation}
    \gamma_{j,l} = \frac{m \, g_{1D}}{\hbar n^{j,l}_{1D}} 
\end{equation}

\subsubsection{Weighted average}

\noindent The values of $\gamma_{j,l}$ \NOTE{à voir} for each of the 1D tubes are however not very meaningful in practice as the distribution that we measure results from the contribution of every lattice tubes. It is therefore more convenient to define a single averaged value of $\bar{\gamma}$ to approximately describe the entire ensemble of 1D gases. One could first simply think of using a simple average:

\begin{equation}
    \bar{\gamma} = \frac{1}{N_{\rm{tubes}}} \sum_{j,l} \gamma_{j,l}
\end{equation}

\noindent This kind of averaging is however too strong of an approximation as it assumes that each of the tubes contribute equally to the total measured distribution which is wrong as the contribution of the tubes with more atoms will be more significant. We then chose to weight the contribution of each of the tubes in the average by its fraction of the total atom number:

\begin{equation}
    \bar{\gamma} = \sum_{j,l} \frac{N_{j,l}}{\NBEC} \gamma_{j,l}
\end{equation}

\noindent Note that this kind of averaging can be done for all relevant quantities that vary from one 1D tube to another.



\subsection{Independence of the tubes}

In order to properly observe 1D physics, it is crucial that all the 1D tubes are independent from one another, \ie no coherence subsists in the transverse directions. This is true when the typical timescale \NOTE{faire cette section}

\section{Detection of large momentum components}

While the great sensitivity of the $\He$ detector is perfectly suited to detect the very low density $\kmf$ momentum tails, the size of the MCPs limits the range of detector. From the work \cite{xu2015universal}, we obtain that the $\kmf$ decay should start around $k_0 \sim 1.6 \times \rho_{\rm{1D}}(0)$ \NOTE{blah blah chiffres}.

\subsection{Magnetic gradient and displacement procedure}

One solution to this issue is to give the entire cloud a momentum kick in the first instants of the TOF to artificially change the momentum range of the $\He$ detector. With our experimental setup, the easiest way to do so is to create a magnetic gradient to apply a magnetic force on the atoms during a time $t_{\rm{grad}}$ before transferring them to the $m_j=0$ sub-state. 

This technique however brings some experimental complications as the population transfer cannot be done immediately after turning off the trap. As a matter of fact, the atoms starts moving during the time $t_{\rm{grad}}$ and will therefore be at different positions when the transfer is performed. The problem comes from the fact that there is a slight inhomogeneity in the bias field along direction $x$ used to set the energy difference between the sub-states $m_j=0$ and $m_j=1$. This means that the resonance condition for a Raman or RF transfer depends on the initial momentum of the atoms, with the consequence that we cannot properly transfer the whole cloud to $m_j=0$ with a simple single frequency Rabi pulse.

To solve this issue, \NOTE{expliquer rapidement sweep}. We understand however that it is increasingly difficult to fulfill the adiabatic condition for all momentum classes if the spread in resonance frequencies is too high. We therefore need to devise a displacement sequence as short as possible to minimize the distance travelled by the atoms before the transfer, as well as to make sure that the magnetic gradient is properly turned off not to further increase the field inhomogeneity.

The procedure is represented on Fig.-\ref{fig:displacement_sequence}. Right after the lattice is turned off, we increase the current in the MOT coils to produce the magnetic gradient. The MOT coils are indeed the coils with which we can produce the stronger gradient, reducing the time during which we need to apply it to reach the proper momentum shift. However, the current in the MOT coils typically needs around $10 \ \rm{ms}$ to reach the highest possible values, which is already too long. We then set the command voltage $V_{\rm{command}}$ to be close to the highest possible value, let the current increase for $t_1 = 1 \ \rm{ms}$ and then set the command to $0$ and let the current decay for $t_2-t_1=13 \ \rm{ms}$ \NOTE{check numbers} until it is fully turned off. After that, we finally perform the population transfer and let the atoms fall unto the MCP. The momentum displacement of the cloud can be set by changing the command voltage $V_{\rm{command}}$.

\begin{figure}
    \centering
    \includegraphics[width=0.8\textwidth]{Fig/Chapter5/displacement_sequence.png}
    \caption{Experimental sequence to shift the entire momentum distribution so that the $\kmf$ tails fall unto the $\He$ detector.}
    \label{fig:displacement_sequence}
\end{figure}

\subsection{Benchmarking with 3D lattice gases momentum distribution}

To test that our method does not induce any distortion of the momentum distribution, we benchmark it with 3D lattice gas momentum distribution slightly above the Mott critical point so that the momentum distribution has a wide background but still sharp diffraction peaks. This allows to check for distortion at high momentum values, while making more precise measurements with the narrow diffraction peaks \NOTE{vraiment pas ouf}.





\section{Experimental study}

\subsection{Analysis of the transverse shape}

The first thing that we need to check is the transverse shape of the 3D distribution to know whether we can fully decouple what is happening in the 1D direction from what is happening in the other two transverse directions. The transverse momentum distribution is supposed to be a Gaussian distribution whose width depends on the transverse trapping frequency. If the atoms are in the harmonic oscillator ground state of the transverse direction, the RMS width of the distribution in momentum space is $\Delta k_{\rm{theo}}=\sqrt{\frac{m \omega_{\perp}}{2 \hbar}}$. \NOTE{factor 2?}
\paragraph{} On Fig.-\ref{fig:1D_transverse}, we plot the transverse distribution along gravity at different positions along the 1D direction and normalize it to 1. We observe that we get the same RMS size for every $k_{\rm{1D}}$ at which the cut is done meaning that the 1D direction is fully decoupled from what is happening in the transverse direction. We extract its RMS width \fcolorbox{red}{white}{$\Delta k_{\rm{exp}}=5.96(2) \mu \rm{m}^{-1}$}. This data set was taken with $s=26$, meaning that $\omega_{\rm{site}}=1.33 \times 10^6$ Hz \NOTE{errorbar?}, giving \fcolorbox{red}{white}{$\Delta k_{\rm{theo}}=6.4 \mu \rm{m}^{-1}$} so a reasonable agreement between the two values. \NOTE{notations}

\begin{figure}
    \centering
    \includegraphics[width=0.9\textwidth]{Fig/Chapter5/1D_transverse_effect.png}
    \caption{Caption}
    \label{fig:1D_transverse}
\end{figure}

\subsection{Calculation of the momentum density}


\label{sec:1D_calculation_momentum_density}

In order to compare the experimental values of the Tan's contact to theory, it is crucial to obtain the absolute value of the 1D density $n_{\rm{1D}}(k)$ from the experimental data. To do so, we exploit the fact that the transverse distribution shape is the same along the 1D direction as we have just seen. Under the effect of the fast transverse expansion, some atoms fall beyond the MCP and are therefore not detected. However, knowing the transverse profile, we can do as if everything was only happening in one direction and integrate over the transverse profile. The procedure is the following:

\begin{itemize}
    \item We plot the transverse distribution (in the vertical direction where it is not cut out by the finite size of the $\He$ detector) and extract its RMS width $\sigma$. \NOTE{notations}
    \item For one pixel of size $\Delta k_{1D} \times \Delta k_{\perp}^2$, we have (keeping in mind normalization condition with the factor $2\pi$) \NOTE{vérifier ça}:
    

    \begin{align}
        n_{1D}(k) &= 2\pi \times \frac{N_{\rm{vox}}(k)}{\eta \Delta k_{1D} \Delta k_{\perp}^2} \left(\int n_{\perp}(k) dk \right)^2 \\
        % &=2\pi \times \frac{N_{\rm{pix}}(k)}{\eta \delta k_{1D} \delta k_{\perp}^2} 2 \pi \sigma 
    \end{align}
    
    \noindent $\eta$ being the detection efficiency and $N_{\rm{pix}}$ the number of atoms in the voxel at a given $k$. From this, we obtain the expression of $n_{\rm{1D}}(k)$ that depends only on known experimental values:
    
    \begin{equation}
        n_{1D}(k) = 4\pi^2 \times \frac{N_{\rm{pix}}(k)}{\eta \delta k_{1D} \delta k_{\perp}^2} \sigma 
    \end{equation}

    

\end{itemize}

\subsection{Transverse integration effects}

As in \ref{sec:transverse_integration}, the transverse size of the voxels defines a transverse integration $\Delta k_{\perp}$ that needs to be sufficiently large \NOTE{number} to ensure a proper signal to noise ratio. As illustrated on Fig.\ref{fig:1D_integration}, the transverse integration however effectively reduces the momentum range in the 1D direction because of the circular shape of the detector that cuts out a part of the integration volume. The transverse integration must then be kept as low as possible and the distorted edges of the distribution ignored in the analysis. 

\begin{figure}
    \centering
    \includegraphics[width=0.7\textwidth]{Fig/Chapter5/1D_transverse_integration.png}
    \caption{Gravity integrated 2D image of the distribution of the 1D lattice gas illustrating the effect of the transverse integration. The red shaded area indicated the region where the geometry of the detector affects the measurement of $n_{\rm{1D}} (k)$.}
    \label{fig:1D_integration}
\end{figure}

\subsection{Measurement of the temperature}

\label{sec:1D_temperature}

As we want to study the dependency of the Tan's contact with temperature as well make comparison with QMC calculations, we first need to extract the temperature from the experimental data. This is easier than for 3D Bose-Hubbard gases as the width of the momentum distribution gives information about the temperature of the gas. At low values of $k$, the shape of the 1D momentum density is Lorentzian as illustrated on Fig.\ref{fig:1D_temperature}. \NOTE{refs} The temperature of the gas can be extracted from the width of the Lorenztian shape using 

\begin{equation}
    n(k)=\frac{2n/\Delta k }{1+(k/\Delta k)^2}
\end{equation}

\noindent with $n$ the maximum density ({\color{blue}[CHECK]}) and:

\begin{equation}
    \Delta k=\frac{m k_b T}{\hbar^2 \rho_{1D}(0)} \alpha_{\rm{fit}}
\end{equation}

\noindent with $\rho_{1D}(0)$ the spatial density at the center of the tube and $\alpha_{\rm{fit}}$ a coefficient to take into account the effect of the trapping potential. This coefficient varies slightly with the interaction parameter $\gamma$ and has been calibrated through QMC calculations. We use a $\rho_{1D}(0)$ which correspond to the weighted averaged $\rho_{1D}(0)$ over the tube distribution. {\color{blue}(good choice of $\rho_{1D}(0)$)?}.

\begin{figure}
    \centering
    \includegraphics[width=0.8\textwidth]{Fig/Chapter5/1D_temperature_lorentz.png}
    \caption{Normalized 1D momentum distribution $n_{\rm{1D}} (k)$ for lattice holding times $t_{\rm{hold}}=5 \ \rm{ms}$ and $t_{\rm{hold}}=500 \ \rm{ms}$. The Lorentzian fit well matches the data at low $k$. The increase in the width of the distribution signals the increase in temperature induced by the increased holding time in the lattice.}
    \label{fig:1D_temperature}
\end{figure}

\NOTE{a détailler}

\subsection{Interaction parameter}

The other relevant parameter affecting the value of Tan's contact besides temperature is the interaction parameter $\gamma$. It writes:

\begin{equation}
    \gamma = \frac{m g_{1D}}{\hbar^2 \rho_{1D}(0)}
    \label{eq:gamma}
\end{equation}

\noindent and can be calculated from the number of atoms in a tube, itself deduced from the total atom number with the algorithm presented in \NOTE{ref}. Table \ref{tab:gamma_vs_N} shows its mininmum, maximum, and weighted average value over the ensemble of 1D gases for data sets with different atom numbers.


\begin{table}[h!]
\centering
{\rowcolors{2}{white}{MainColor!12}
    \begin{tabular}{c|c|c|c}
        {\color{MainColor} N} &  {\color{MainColor}$\gamma_{\rm{min}}$} & {\color{MainColor}$\gamma_{\rm{max}}$} & {\color{MainColor}$\mean{\gamma}$}  \\
        \hline
        3.10 $\times 10^4$  & 0.259 & 2.17 & 0.416 \\
        1.10 $\times 10^5$ & 0.156 & 1.63 & 0.252 \\
        2.26 $\times 10^5$ & 0.117 & 1.44 & 0.190 \\
    \end{tabular}}
\caption{Variations of the interaction parameter $\gamma$ among the 1D tubes for different total atom numbers.}
\label{tab:gamma_vs_N}
\end{table}

\subsection{Experimental procedure and first extracted values of the Tan's contact}

The procedure to measure Tan's contact for a given data set is as follows:

\begin{itemize}
    \item We prepare the parameters of the experiment to reach the desired values of temperature and atom number. The latter is calibrated via absorption imaging while the former is set roughly by changing the holding time in the lattice and checking that the width of the 1D distribution increases. 
    \item We start by taking $\sim 100$ experimental shots with no gradient to measure the low $k$ distribution from which we can extract the temperature as explained in \ref{sec:1D_temperature}. We do not need to take a large number of shots as the signal is quite high and we do not require a very high signal to noise ratio to obtain the temperature.
    \item We set the gradient to shift the momentum distribution to access the momentum region where the the $\kmf$ tails are supposed to be present as explained in \NOTE{ref}. Usually, the displacement is not too high so that the momentum range overlaps the natural momentum range of the $\He$ detector where no gradient is used, allowing to check that the displaced data matches nicely the non displaced data in the region of the overlap. 
    \item After computing the 1D distribution $n_{\rm{1D}} (k)$ with the method detailed in \ref{sec:1D_calculation_momentum_density}, we plot the quantity $n_{\rm{1D}} (k) \times k^4$. The presence of $\kmf$ tails is signaled by a flat zone that we can fit with a constant function to extract the bare value of the contact $C$ as illustrated on Fig.-\ref{fig:1D_plots}.

    
\end{itemize}  
    
\begin{figure}
    \centering
    \includegraphics[width=0.95\textwidth]{Fig/Chapter5/1D_plots.png}
    \caption{Plots of $n_{\rm{1D}}$ for $s=26$, $N=3.1(3) \times 10^4$ and $t_{\rm{hold}}=5 \ \rm{ms}$. (a) Linear scale plot of the normalized $n_{\rm{1D}}$ for two momentum ranges. (b) Same data in log scale. The red line indicates a $k^{-4}$ fit. (c) $k^4 n_{\rm{1D}} (k)$ at high momentum. The red shaded area indicates the flat zone of the $\kmf$ tail.}
    \label{fig:1D_plots}
\end{figure}

In order to compare conveniently the experimental data to the theoretical work of \cite{yao2018tan}, we introduce two dimensionless quantities. The first one is the reduced temperature:

\begin{equation}
    \xi_T=-a_{\rm{1D}}/\lambda_T
\end{equation}

\noindent with $a_{\rm{1D}}$ the 1D scattering length and $\lambda_T=\sqrt{2\pi \hbar^2/m k_b T}$. The second one is the reduced the interaction strength:

\begin{equation}
    \xi_{\gamma}=-a_{\rm{ho}}/a_{\rm{1D}}\sqrt{N}
\end{equation}

\noindent with $a_{\rm{ho}}=\sqrt{\hbar/m \omega}$ the harmonic oscillator length. The calculations presented in \cite{yao2018tan} show that 

\begin{equation}
    C= \frac{N^{5/2}}{a_{\rm{ho}}^3} f(\xi_{\gamma},\xi_{T})
\end{equation}

\noindent We can then define a rescaled contact

\begin{equation}
    \tilde{C}=C \frac{a_{\rm{ho}}^3}{N^{5/2}}
\end{equation}

\noindent which depends only on $f(\xi_{\gamma},\xi_{T})$, $\xi_{\gamma}$ and $\xi_{T}$ then being the good parameters to characterize the contact. The ultimate goal of the experiment would then be to characterize the evolution of $f(\xi_{\gamma},\xi_{T})$ and compare it to the predictions of \cite{yao2018tan}. Our quantity of interest will then be the rescaled contact $\tilde{C}$.

\subsubsection{Effect of temperature}

We plot on Fig.-\ref{fig:C_tilde_vs_T} the experimental rescaled contact $\tilde{C}$ as a function of $\xi_T$ for a fixed atom number $N=1.1 \times 10^5$ corresponding to $\xi_{\gamma}=0.113$. The error bars corresponds to the standard deviation over the data points averaged to obtain the value of the contact. \NOTE{terminer avec partie théorique}

\begin{figure}
    \centering
    \includegraphics[width=0.6\textwidth]{Fig/Chapter5/C_tilde_vs_T.png}
    \caption{Rescaled contact $\tilde{C}$ as a function $\xi_T$. \NOTE{finir la caption avec partie théorique}}
    \label{fig:C_tilde_vs_T}
\end{figure}

\subsubsection{Effect of the total atom number}

\subsection{Comparison with QMC calculations}

\section{Discussion of the preliminary results}

\chapter*{Conclusion}

\label{chap:conclusion}
\addcontentsline{toc}{chapter}{\nameref{chap:conclusion}}


\fancyhead[LO]{\sffamily\bfseries Conclusion} % Print the nearest section name on the left side of odd pages
\fancyhead[RE]{\sffamily\bfseries Conclusion} % Print the current chapter name on the right side of even pages

The central result of this thesis is the observation of \kmk pairs in the quantum depletion of a weakly-interacting lattice Bose gas, a result that oriented most of the work I did during my PhD. This measurement was the next step in our task of fully characterizing the correlations across the superfluid-to-Mott insulator transition after the first two works \cite{carcy2019momentum,cayla2020} conducted by the former PhD students Hugo Cayla and Cécile Carcy that focused on the local correlations, respectively deep in the Mott regime and in the superfluid region. 

The first work that I did during my PhD was to study the two-body collisions halos \cite{tenart2020two} as a means to complete the previous benchmarking work \cite{cayla2018single} to certify that the measured atomic distribution faithfully represents the in-trap momentum distribution of the gas with a \textbf{single particle resolution}. We devised a simple theoretical model predicting the number of atoms in the collision halos that we validated experimentally by measuring this number for large number of atoms loaded in the lattice. Extrapolating the predictions of the simple model to the low atom numbers usually used in our experiments, we found that two-body collisions are indeed negligible. This work was then completed by the study of the adiabatic preparation of the gas in the vicinity of the Mott transition \cite{carcy2021} mainly lead by Cécile Carcy in collaboration with the theoretician Tommaso Roscilde and finished around the middle of my PhD, concluding the series of experiment aiming to prove that our experiment properly simulates the Bose-Hubbard model.

As the observation of the \kmk pairs had already been attempted by our team without success, we decided to improve our apparatus by implementing a two-photon Raman transfer to improve the detection efficiency and in turn our chances of seeing the \kmk correlation signal. Building the Raman transfer optical setup and testing it was the second main project of my PhD. 

It was more or less at this time that the Covid-19 pandemic hit, forcing us to leave the lab and stay at home. I used this time away from the experiment to develop the algorithm to compute the anomalous correlation function $\gtwo_A$ to look for the \kmk correlations, that I then tested and troubleshot at first with simulated data and in a second time with the data from the earlier project on the collision halos in which simple classical \kmk correlations can be observed. 

We started the measurement campaign for the \kmk correlations a few weeks after the end of first lockdown and were able to observe first experimental signals. We then performed the experiment again at high temperatures and observed that the \kmk correlation signal was lost contrary to \kk correlations, convincing us that the observed \kmk correlations were indeed linked to $T=0$ quantum coherences. We then took several data sets with different total atom number to see if we could observe similar effects than in the Bogoliubov theory of the weakly-interacting Bose gas. We were able to measure widths of the \kmk correlation peak consistent with the estimations of \cite{butera2020} and observe the $1/\bar{\rho}}$ scaling of the amplitude of the \kmk correlation peak. In addition, we clearly observed $\gtwo_A (\bm{0}) > \gtwo_N (\bm{0})$, violating the Cauchy-Schwarz inequality, once again signaling the quantum nature of the correlation signal, and finally measured relative number squeezing between modes $\bm{k}$ and $-\bm{k}$. These last two measurements constitute a first step towards showing the presence of entanglement in the many-body ground state of our system.

In a nutshell, we were able to report the first observation of \kmk correlations in an \textbf{at-equilibrium} system, resulting from the interplay between quantum fluctuations and interactions, confirming the 60 years old prediction of Bogoliubov and Lee-Huang-Yang. This result is also of great importance for our future experiments as it shows that our experiment is capable of detecting non-local \kmk correlations, hinting at a possible detection of \kmk correlations in Cooper pairs and opening the way to future measurements aiming to observe more complex correlation pattern, notably close the Quantum Critical Point of the Mott transition as we will discuss on the outlooks.

In parallel, I also spent a significant amount of time working on the project of measuring Tan's contact in 1D gases, following our collaboration with Hepeng Yao and his supervisor Laurent--Sanchez Palencia from Centre de Physique Théorique at Ecole Polytechnique. We managed to find a solution to the momentum range problem of the detector by using a magnetic gradient to shift the entire momentum distribution and access the high momentum region, and were able to observe a $\kmf$ decay on various data sets. While the qualitative evolution of the contact with temperature and interaction strength is consistent with theory, there is large discrepancy with the QMC calculations that remains to this day unexplained and should be the subject of future experiments.

\section*{Outlooks}

\NOTE{voir avec chapitre 4 corrigé}

Our measurements of \kmk correlations have voluntarily left constant the lattice depth. An immediate way of pushing these measurements further that we have already started working on is to repeat them while progressively increasing the lattice depth. As $U/J$ increases and with it the strength of the interactions, we should reach a point at which the Bogoliubov approximation is not valid anymore. It would then be interesting to see how this effect translates to the \kmk correlation signal. We notably expect that more complex correlation pattern should appear as the strength of the interactions increases, possibly involving more than 2 particles. These kind of complex correlations are expected to be particularly important at the Quantum Critical Point of the superfluid-to-Mott insulator transition. A short-mid range prospect would then be to develop new data analysis techniques to measure higher order correlation functions, test them in simple cases like by measuring bosononic bunching with more than 2 particles, and finally use them in experimental data progressively closer to the Quantum Critical Point. As obtaining a good enough signal to noise ratio to measure a $n$-th order correlation function gets increasingly difficult as $n$ increases, we would need to take large amount of data at the Quantum Critical Point. This prospect is particularly exciting as no theory predicts what should happen in terms of momentum space correlations at the Quantum Critical Point, meaning that our experiment could really enter the realm of Quantum Simulators. 

In addition, our observation of the violation of the Cauchy-Schwarz inequality would be enough to show the presence of entanglement if we were able to measure the correlator  $\langle a^{\dagger}({\bm k}) a({-\bm k}) \rangle$ and show that it is equal to 0 as in Bogoliubov theory. This would require to add new experimental tools to our apparatus like Bragg spectroscopy to do so.

Another short range objective would be to keep investigating the discrepancy between the experimental data and the QMC calculations for the measurement of Tan's contact, notably by taking additional data using the newly added two-photon Raman transfer instead of a RF transfer as we did. We hope that the increased detection efficiency would help us being less sensitive to the possible effects of the $m_j=0$ impurities while reducing the number of non-transferred $m_j=1$ atoms that may have weird trajectories that would lead them to fall on the detector and perturb our measurement. This measurements might then help us to identify eventual problems in the experiment or in the way that we compare our data to the theory.

Finally, a more long term prospect would be to improve the experimental setup to bring the fermionic isotope of Helium, $^3\He$, to quantum degeneracy. This would open the way to study a whole new kind of physics with the great momentum space resolution of our detector. It would be particularly interesting to study the physics of the BEC-BCS transition and directly measure \kmk correlations in a Cooper pair of the BCS phase. To do so, we would first need to identify a usable Feshbach resonance to create the Cooper pairs as there have currently not been a proper investigation of the existence of Feshbach resonance in $^3\He$. 

\renewcommand{\thefigure}{1}
\begin{figure}[h!]
    \centering
    \includegraphics[width=0.95\textwidth]{Fig/Conclusion/before_after.png}
    \caption[The new experiment room]{The new experiment room. (a) Before moving in, with only a few optical tables left by the previous occupants of the room. (b) After moving in.}
    \label{fig:before_after}
\end{figure}

Actually, the lab room in which all the experiments of this thesis were conducted was starting to get packed, and adding the new equipment to cool down a new atomic species would have barely left enough space in the room for a PhD student. This last year, we took a first (but big!) step towards the installation of $^3\He$ in our experiment by moving the entire apparatus to a new bigger room a few steps down the corridor (see Fig.-\ref{fig:before_after}), giving us plenty of additional space. At the moment that I am writing this manuscript, we managed to get the experiment back to its working state and were able to produce BECs and should be able to resume taking data soon. I would like to end this manuscript with one last figure (\ref{fig:SC_moving}) which is of my favorite picture of my time as a PhD student, showing the Helium Lattice team in the perilous process of moving the science chamber to the new room, in which I hope it will serve to make many beautiful experiments in the years to come.



\renewcommand{\thefigure}{2}

\begin{figure}
    \centering
    \includegraphics[width=0.8\textwidth]{Fig/Conclusion/SC_moving.jpg}
    \caption{The Helium Lattice team with the Science Chamber, moving from the old room to the new one.}
    \label{fig:SC_moving}
\end{figure}






\cleardoublepage
\phantomsection
\addcontentsline{toc}{chapter}{\listfigurename}


\fancyhead[LO]{\sffamily\bfseries List of Figures} % Print the nearest section name on the left side of odd pages
\fancyhead[RE]{\sffamily\bfseries List of Figures} % Print the current chapter name on the right side of even pages
\listoffigures

\cleardoublepage
\phantomsection
\addcontentsline{toc}{chapter}{\listtablename}
\listoftables

\cleardoublepage
\phantomsection
\chapter*{Publications}
\label{chap:publications}
\addcontentsline{toc}{chapter}{\nameref{chap:publications}}

\begin{itemize}
    \item C. Carcy, H. Cayla, A. Tenart, A. Aspect, M. Mancini, and D. Clément. \textbf{Momentum-space atom correlations in a mott insulator}. Physical Review X, 9(4):041028, 2019. Used as reference \cite{carcy2019momentum} in this manuscript.
    \vspace{0.5cm}
    \item A. Tenart, C. Carcy, H. Cayla, T. Bourdel, M. Mancini, and D. Clément. \textbf{Two-body collisions in the time-of-flight dynamics of lattice bose superfluids}. Physical Review Research, 2(1):013017, 2020. Used as reference \cite{tenart2020two} in this manuscript.
    \vspace{0.5cm}
    \item H. Cayla, S. Butera, C. Carcy, A. Tenart, G. Hercé, M. Mancini, A. Aspect, I. Carusotto, and D. Clément. \textbf{ Hanbury-brown and twiss bunching of phonons and of the quantum depletion in a strongly-interacting bose gas}. Physical Review Letters, 125:165301, 2020. Used as reference \cite{cayla2020} in this manuscript.
    \vspace{0.5cm}
    \item C. Carcy, G. Hercé, A. Tenart, T. Roscilde, and D. Clément. \textbf{Certifying the adiabatic preparation of ultracold lattice bosons in the vicinity of the mott transition}. Physical Review Letters, 126(4):045301, 2021. Used as reference \cite{carcy2021} in this manuscript.
    \vspace{0.5cm}
    \item G. Hercé, C. Carcy, A. Tenart, J.-P. Bureik, A. Dareau, D. Clément, and T. Roscilde. \textbf{Studying the low-entropy mott transition of bosons in a three-dimensional optical lattice by measuring the full momentum-space density}. Physical Review A, 104, 2021. Used as reference \cite{herce2021studying} in this manuscript.
    \vspace{0.5cm}
    \item A. Tenart, G. Hercé, J.-P. Bureik, A. Dareau, and D. Clément. \textbf{Observation of pairs of atoms at opposite momenta in an equilibrium interacting bose gas}. arXiv preprint arXiv:2105.05664, 2021. Accepted for publication in Nature Physics. Used as reference \cite{tenart2021observation} in this manuscript. 
\end{itemize}




\setlength{\parskip}{\parskipinitial}
%\usechapterimagefalse



\addcontentsline{toc}{chapter}{Bibliography}
\fancyhead[LO,RE]{\sffamily\normalsize\bfseries Bibliography}

\bibliography{sample}



\end{document}
