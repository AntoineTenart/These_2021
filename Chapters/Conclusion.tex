\chapter*{Conclusion}

\label{chap:conclusion}
\addcontentsline{toc}{chapter}{\nameref{chap:conclusion}}


\fancyhead[LO]{\sffamily\bfseries Conclusion} % Print the nearest section name on the left side of odd pages
\fancyhead[RE]{\sffamily\bfseries Conclusion} % Print the current chapter name on the right side of even pages

The central result of this thesis is the observation of \kmk pairs in the quantum depletion of a weakly-interacting lattice Bose gas, a result that oriented most of the work I did during my PhD. This measurement was the next step in our task of fully characterizing the correlations across the superfluid-to-Mott insulator transition after the first two works \cite{carcy2019momentum,cayla2020} conducted by the former PhD students Hugo Cayla and Cécile Carcy that focused on the local correlations, respectively deep in the Mott regime and in the superfluid region. 

The first work that I did during my PhD was to study the two-body collisions halos \cite{tenart2020two} as a means to complete the previous benchmarking work \cite{cayla2018single} to certify that the measured atomic distribution faithfully represents the in-trap momentum distribution of the gas with a \textbf{single particle resolution}. We devised a simple theoretical model predicting the number of atoms in the collision halos that we validated experimentally by measuring this number for large number of atoms loaded in the lattice. Extrapolating the predictions of the simple model to the low atom numbers usually used in our experiments, we found that two-body collisions are indeed negligible. This work was then completed by the study of the adiabatic preparation of the gas in the vicinity of the Mott transition \cite{carcy2021} mainly lead by Cécile Carcy in collaboration with the theoretician Tommaso Roscilde and finished around the middle of my PhD, concluding the series of experiment aiming to prove that our experiment properly simulates the Bose-Hubbard model.

As the observation of the \kmk pairs had already been attempted by our team without success, we decided to improve our apparatus by implementing a two-photon Raman transfer to improve the detection efficiency and in turn our chances of seeing the \kmk correlation signal. Building the Raman transfer optical setup and testing it was the second main project of my PhD. 

It was more or less at this time that the Covid-19 pandemic hit, forcing us to leave the lab and stay at home. I used this time away from the experiment to develop the algorithm to compute the anomalous correlation function $\gtwo_A$ to look for the \kmk correlations, that I then tested and troubleshot at first with simulated data and in a second time with the data from the earlier project on the collision halos in which simple classical \kmk correlations can be observed. 

We started the measurement campaign for the \kmk correlations a few weeks after the end of first lockdown and were able to observe first experimental signals. We then performed the experiment again at high temperatures and observed that the \kmk correlation signal was lost contrary to \kk correlations, convincing us that the observed \kmk correlations were indeed linked to $T=0$ quantum coherences. We then took several data sets with different total atom number to see if we could observe similar effects than in the Bogoliubov theory of the weakly-interacting Bose gas. We were able to measure widths of the \kmk correlation peak consistent with the estimations of \cite{butera2020} and observe the $1/\bar{\rho}}$ scaling of the amplitude of the \kmk correlation peak. In addition, we clearly observed $\gtwo_A (\bm{0}) > \gtwo_N (\bm{0})$, violating the Cauchy-Schwarz inequality, once again signaling the quantum nature of the correlation signal, and finally measured relative number squeezing between modes $\bm{k}$ and $-\bm{k}$. These last two measurements constitute a first step towards showing the presence of entanglement in the many-body ground state of our system.

In a nutshell, we were able to report the first observation of \kmk correlations in an \textbf{at-equilibrium} system, resulting from the interplay between quantum fluctuations and interactions, confirming the 60 years old prediction of Bogoliubov and Lee-Huang-Yang. This result is also of great importance for our future experiments as it shows that our experiment is capable of detecting non-local \kmk correlations, hinting at a possible detection of \kmk correlations in Cooper pairs and opening the way to future measurements aiming to observe more complex correlation pattern, notably close the Quantum Critical Point of the Mott transition as we will discuss on the outlooks.

In parallel, I also spent a significant amount of time working on the project of measuring Tan's contact in 1D gases, following our collaboration with Hepeng Yao and his supervisor Laurent--Sanchez Palencia from Centre de Physique Théorique at Ecole Polytechnique. We managed to find a solution to the momentum range problem of the detector by using a magnetic gradient to shift the entire momentum distribution and access the high momentum region, and were able to observe a $\kmf$ decay on various data sets. While the qualitative evolution of the contact with temperature and interaction strength is consistent with theory, there is large discrepancy with the QMC calculations that remains to this day unexplained and should be the subject of future experiments.

\section*{Outlooks}

\NOTE{voir avec chapitre 4 corrigé}

Our measurements of \kmk correlations have voluntarily left constant the lattice depth. An immediate way of pushing these measurements further that we have already started working on is to repeat them while progressively increasing the lattice depth. As $U/J$ increases and with it the strength of the interactions, we should reach a point at which the Bogoliubov approximation is not valid anymore. It would then be interesting to see how this effect translates to the \kmk correlation signal. We notably expect that more complex correlation pattern should appear as the strength of the interactions increases, possibly involving more than 2 particles. These kind of complex correlations are expected to be particularly important at the Quantum Critical Point of the superfluid-to-Mott insulator transition. A short-mid range prospect would then be to develop new data analysis techniques to measure higher order correlation functions, test them in simple cases like by measuring bosononic bunching with more than 2 particles, and finally use them in experimental data progressively closer to the Quantum Critical Point. As obtaining a good enough signal to noise ratio to measure a $n$-th order correlation function gets increasingly difficult as $n$ increases, we would need to take large amount of data at the Quantum Critical Point. This prospect is particularly exciting as no theory predicts what should happen in terms of momentum space correlations at the Quantum Critical Point, meaning that our experiment could really enter the realm of Quantum Simulators. 

In addition, our observation of the violation of the Cauchy-Schwarz inequality would be enough to show the presence of entanglement if we were able to measure the correlator  $\langle a^{\dagger}({\bm k}) a({-\bm k}) \rangle$ and show that it is equal to 0 as in Bogoliubov theory. This would require to add new experimental tools to our apparatus like Bragg spectroscopy to do so.

Another short range objective would be to keep investigating the discrepancy between the experimental data and the QMC calculations for the measurement of Tan's contact, notably by taking additional data using the newly added two-photon Raman transfer instead of a RF transfer as we did. We hope that the increased detection efficiency would help us being less sensitive to the possible effects of the $m_j=0$ impurities while reducing the number of non-transferred $m_j=1$ atoms that may have weird trajectories that would lead them to fall on the detector and perturb our measurement. This measurements might then help us to identify eventual problems in the experiment or in the way that we compare our data to the theory.

Finally, a more long term prospect would be to improve the experimental setup to bring the fermionic isotope of Helium, $^3\He$, to quantum degeneracy. This would open the way to study a whole new kind of physics with the great momentum space resolution of our detector. It would be particularly interesting to study the physics of the BEC-BCS transition and directly measure \kmk correlations in a Cooper pair of the BCS phase. To do so, we would first need to identify a usable Feshbach resonance to create the Cooper pairs as there have currently not been a proper investigation of the existence of Feshbach resonance in $^3\He$. 

\renewcommand{\thefigure}{1}
\begin{figure}[h!]
    \centering
    \includegraphics[width=0.95\textwidth]{Fig/Conclusion/before_after.png}
    \caption[The new experiment room]{The new experiment room. (a) Before moving in, with only a few optical tables left by the previous occupants of the room. (b) After moving in.}
    \label{fig:before_after}
\end{figure}

Actually, the lab room in which all the experiments of this thesis were conducted was starting to get packed, and adding the new equipment to cool down a new atomic species would have barely left enough space in the room for a PhD student. This last year, we took a first (but big!) step towards the installation of $^3\He$ in our experiment by moving the entire apparatus to a new bigger room a few steps down the corridor (see Fig.-\ref{fig:before_after}), giving us plenty of additional space. At the moment that I am writing this manuscript, we managed to get the experiment back to its working state and were able to produce BECs and should be able to resume taking data soon. I would like to end this manuscript with one last figure (\ref{fig:SC_moving}) which is of my favorite picture of my time as a PhD student, showing the Helium Lattice team in the perilous process of moving the science chamber to the new room, in which I hope it will serve to make many beautiful experiments in the years to come.



\renewcommand{\thefigure}{2}

\begin{figure}
    \centering
    \includegraphics[width=0.8\textwidth]{Fig/Conclusion/SC_moving.jpg}
    \caption{The Helium Lattice team with the Science Chamber, moving from the old room to the new one.}
    \label{fig:SC_moving}
\end{figure}



