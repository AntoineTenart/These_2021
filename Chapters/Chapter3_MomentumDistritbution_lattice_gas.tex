\chapter{Single-atom resolved momentum measurement of lattice Bose gas}

blah blah blah

\section{Metastable Helium}

Metastable Helium, noted $\mathrm{He}^*$, is not a very widely used atom in the ultracold atoms community as it comes with a few experimental difficulties that we will discuss in this chapter. However, this atom has one huge advantage that differentiates it from the other condensable atomic species: its metastable state. Indeed, the Helium atom can be excited in the state $2^3 S_1$ which has a very large lifetime of the order of $8,000 \ \mathrm{s}$, far larger than what is required for experiments. In this metastable state, the Helium atom has a quite large energy of $19.8 \ \mathrm{eV}$ that can be exploited to detect the atom \textbf{electronically}, contrary to the wide majority of ultracold atoms experiment that relies on optical detection techniques. This comes with the great advantage that this electronic detection technique is suited to detect \textbf{single atoms}, making metastable Helium the ideal candidate for correlation functions measurement as we have discussed in the first chapter of this thesis. In addition, the energy level structure is well adapted to laser cooling with a transition in the near-infrared around $\lambda \simeq 1083 \ \mathrm{nm}$ for which reliable laser sources are available.

\section{The experimental setup}

\subsection{Bose-Einstein condensation of mestatable Helium}

\subsection{3D optical lattice}

\subsection{Electronic detection: The Micro Channel Plate Detector}

\section{Two-photon Raman transfer}

\subsection{Principle of the two-photon Raman transfer}

\subsection{Experimental implementation}

\section{Characterisation of two-body collisions in the time-of-flight dynamics}

\subsection{Classical model}

\subsection{Evolution with total atom number}

\subsection{Evolution with lattice depth}

\subsection{Conclusion}

\section{Adabiatic preparation in the vicinity of the Mott transition}

\subsection{Thermometry method}

\subsection{Fischer information and Cramér-Rao bound}

\subsection{Entropy measurement}
