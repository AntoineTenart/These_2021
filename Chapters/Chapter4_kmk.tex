\chapter{Experimental observation of k/-k correlations in the depletion of a weakly-interacting Bose gas}

\label{sec:chapter_4}

The Bogoliubov interacting Bose-gas has been the subject of a large variety of experimental studies (\cite{fontaine2018,miller1962,ozeri2005,stepanov2019}). However, these experiments have mainly focused on measuring the Bogoliubov spectrum of excitations. More than 60 years after the seminal Lee, Huang and Yang paper \cite{lee1957}, an experimental study of the correlations in the many-body ground-state is yet to be done. As we have seen through the different chapters of this thesis, our experimental setup is perfectly suited for such an investigation. 

We will present in this chapter the main result of this thesis, namely the observation of the \kmk pairs of the quantum depletion. We will detail the numerical procedure and the analysis method used to extract the correlation signals and present the experimental results. With the theoretical developments of Chapter \ref{sec:chapter_1}, we will show how we can link the measured anomalous correlation signal to the \kmk pairs of the quantum depletion by studying the effect of temperature, the amplitude and widths of the correlation peak, as well as the fluctuations of the atom number difference between modes $\bm{k}$ and $-\bm{k}$. In addition, we will show how our measurement constitutes a first step towards showing the presence of momentum-space entanglement in many-body ground states.

\section{Numerical procedure to measure two-body correlations}

\label{sec:numerical_calculation}

As we have seen in the previous chapter, we are capable for each experimental run of reconstructing the 3D momentum coordinates of every atom detected by the MCP. We must now devise a numerical procedure to extract the two-body correlation signals from this raw data. Our goal is then to compute the general normalized second-order correlation function:

\begin{equation}
    g^{(2)} (\bm{k},\bm{k'}) = \frac{\mean{\hat{a}^{\dagger}(\bm{k}) \hat{a}^{\dagger}(\bm{k'}) \hat{a}(\bm{k}) \hat{a}(\bm{k'})}}{\mean{\hat{a}^{\dagger}(\bm{k}) \hat{a}(\bm{k}) } \mean{\hat{a}^{\dagger}(\bm{k'}) \hat{a}(\bm{k'}) }}
\end{equation}

\noindent We recognize that $\mean{\hat{a}^{\dagger}(\bm{k}) \hat{a}(\bm{k})}$ is the momentum density in mode $\bm{k}$ that we will write $\rho(\bm{k})$ in the following. We remind to begin the result of Chapter \ref{sec:chapter_1} which is that the $\gtwo$ function will take values different from 1 in two cases:

\begin{itemize}
    \item For $\bm{k'} \simeq \bm{k}$, the \textbf{normal} correlations corresponding to the Hanbury Brown and Twiss effect also known as bosonic bunching.
    \item For $\bm{k'} \simeq -\bm{k}$, the \textbf{anomalous} correlations corresponding to the \kmk pairs in the quantum depletion.
\end{itemize}

In practice, plotting the $\gtwo$ function defined as such does not make much sense. On the one hand, the function here is 6D and thus hard to plot in an intelligible way. On the other hand, obtaining a sufficient signal to noise ratio to make correlation measurements for single values of $\bm{k}$ and $\bm{k'}$ is impossible. The idea is then to integrate the $\gtwo$ function over all momenta $\bm{k}$ and introduce a new parameter $\delta \bm{k}$ to write:

\begin{equation}
    g_{N,A}^{(2)} (\delta {\bm k})=\frac{\int_{\Omega_{k}} \langle \hat{a}^{\dagger}({\bm k}) \hat{a}^{\dagger}(\delta {\bm k} \pm {\bm k}) \hat{a}({\bm k}) \hat{a}(\delta {\bm k} \pm {\bm k}) \rangle \mathrm{d}{\bm k}}{\int_{\Omega_{k}} \rho({\bm k}) \rho(\delta {\bm k} \pm {\bm k}) \mathrm{d}\bm{k}}
    \label{Eq:g2}
\end{equation}

\noindent With this definition, we see that for $\delta \bm{k}=\bm{0}$, we are either looking at \textbf{normal} \kk correlations (subscript N) when chosing the plus sign, or \textbf{anomalous} \kmk correlations (subscript A) with the minus sign. Note that we will often use the subscript $(N,A)$ for all definitions concerning both normal and anomalous correlation functions. We reduced the 6D function to a 3D function of the parameter $\delta \bm{k}$ which equals $\bm{0}$ when the correlation condition $\bm{k'} = \pm \bm{k}$ is fulfilled. This gives us a natural way to evaluate $\gtwo (\delta \bm{k})$ with the experimental data: we compute the values of the parameter $\delta \bm{k}$ for every detected atom pairs in an experimental run by calculating their momentum difference or sum for normal and anomalous correlations respectively. By computing the histogram of these values and averaging over many experimental runs, we evaluate the numerator of equation \ref{Eq:g2}.

\subsection{Description of the algorithm}

The algorithm described here is similar to the one used in our previous works \cite{carcy2019momentum,cayla2020} and detailed in \cite{cayla_these,carcy_these}. This previous version was mainly designed for the observation of bosonic bunching. I adapted the algorithm to make it suitable for the calculation of \kmk correlations as well, as we will discuss now.

\subsubsection{Numerator calculation}

The first step is to compute the numerator of equation \ref{Eq:g2} that we denote $G^{(2)}(\delta \bm{k})$. We note $N_{runs}$ the number of experimental runs and $N_{i}$ the number of atoms in the $i$-th shot. The procedure is as follows:

% \begin{algorithm}
% \caption{$G^{(2)}$ calculation}
%     \begin{algorithmic}{}
%         \FOR{$i=1:N_{runs}$}
%             \FOR{$j=1:N_{i}$}
%                 \STATE Compute $\vec{k}_i+\vec{k}_j$
%                 \STATE Increment histogram $G^{(2)}$ corresponding pixel
%             \ENDFOR
%         \ENDFOR
%     \end{algorithmic}{}
% \end{algorithm}

\newpage

\begin{algorithm}[h!]
 \caption{$G^{(2)}$ calculation}
    \begin{algorithmic}
         \For{$i=1:N_{runs}$}
            \For{$j=1:N_{i}$}
               \For{$p=1:N_{i}$}
                    \State Compute $\delta \bm{k} = \bm{k}_j \pm \bm{k}_p$
                    \State Increment 3D histogram $G^{(2)}$ corresponding voxel
                \EndFor
            \EndFor
        \EndFor
\end{algorithmic}

\end{algorithm}

\noindent We end up with a 3D histogram where each voxel is associated to a value of $\delta \bm{k} = (\delta k_x,\delta k_y, \delta k_z)$ and records how many atom pairs have this specific momentum sum or difference, depending on the kind of the correlations we want to probe.

The major difference with the previous version of the algorithm is that we record here the full 3D histogram of calculated $\delta \bm{k}$ on every pair of atoms. The procedure was originally made simpler by calculating three one-dimensional histograms, one for each direction of space. Each of these histograms represents a one-dimensional cut in the general 3D histogram $G^{(2)}(\delta \bm{k})$. For instance, the $x$ direction histogram was obtained by selecting one atom labelled $i$ and calculating $\delta k_x$ only for atoms $j$ close enough in momentum space to find a \kk correlations, \ie with $|k_y^{(i)}-k_y^{(j)}| \leq \Delta k_{\perp}$ and $|k_z^{(i)}-k_z^{(j)}| \leq \Delta k_{\perp}$, where $\Delta k_{\perp}$ defines a transverse integration (see later). This method obviously saves computing time and RAM space, but is not suited to look for \kmk correlations.

At this point, we record in the central voxels associated to $\delta \bm{k} \simeq \bm{0}$ what we call \textbf{true coincidences}, namely two atoms detected conjointly as a result of \kmk pairing or bosonic bunching. However, we also record \textbf{accidental coincidences} that do not represent correlations but result from the momentum distribution of the atoms. We then need a normalization process to get rid of the contribution of accidental coincidences. 

\subsubsection{Denominator calculation}

\label{sec:algo}

We now want to compute the denominator of equation \ref{Eq:g2}, representing the effect of accidental coincidences. To perform this calculation, we would like to have a sample of atoms with the same momentum density than our experimental data but uncorrelated. This can be done by merging all experimental shots, not correlated with one another. We then apply the procedure we have just described to this data set. However, some of the correlations happening in single shots will remain in this large file. In the end, the total number of correlations in the numerator is $\sum_i N_i^2$ whereas the number of coincidences in the merged file is $(\sum_i N_i)^2$. With $\bar{N}$ the mean number of atoms per shot, we see that:

\begin{equation}
      \frac{\sum_i N_i^2}{(\sum_i N_i)^2}=\frac{N_{\rm{runs}} \bar{N}^2}{N_{\rm{runs}}^2 \bar{N}^2}=\frac{1}{N_{\rm{runs}}}
\end{equation}{}

\noindent Therefore, with enough shots, the contribution of residual coincidences is negligible and the normalization procedure valid. 

In the end, the integrated $g^{(2)}$ function can obtained by dividing the numerator histogram by the denominator and multiplying by the normalization factor $\frac{(\sum_i N_i)^2}{\sum_i N_i^2}$ taking into account the number of coincidences of the numerator and denominator. It is possible to take a fraction of all atoms for the denominator calculation to avoid large computation time. This is particularly handy to have quick first results before launching longer calculations for a nicer signal-to-noise ratio.

\subsection{Accessing the BEC depletion}
\label{sec:accessing_depletion}

In order to detect \kmk pairs in the depletion, it is absolutely crucial to remove from the analysis all atoms belonging to the BEC and its diffracted copies as explained in Chapter \ref{sec:chapter_1}. For each recorded data set and before running the algorithm, we remove all atoms outside the volume $\Omega_k$ that we design to exclude momentum regions with condensed atoms as illustrated on Fig-\ref{fig:omega_k}. We set $\Omega_k$ to have a cubic symmetry that matches the symmetry of the momentum distribution in a cubic lattice. We remove all atoms with $|k_i| < k_{\rm{min}}$ and $|k_i| > k_{\rm{max}}$ where $k_i$ is the momentum projection along an axis $i=x,y,z$. We use $k_{\rm{min}}=0.15 \, k_d$, corresponding to $\sim 6$ times the RMS width of the BEC peaks, in order to ensure that all condensed atoms have been removed. The high limit is set to $k_{\rm{max}}=0.85 \, k_d$ to exclude higher order peaks and is slightly smaller than the momentum range probed by the MCP.

\begin{figure}
    \centering
    \includegraphics[width=0.8\textwidth]{Fig/Chapter4/densite.png}
    \caption{1D cut of the momentum density illustrating the integration volume $\Omega_k$. The central peak corresponds to the BEC and the lateral peaks at $\pm \ k_d$ to diffraction peaks induced by the presence of the optical lattice. The green area shows the volume $\Omega_k$ containing the depleted atoms selected for the correlation measurement. While barely visible in linear scale, they can be seen in the log scale plot shown in inset.}
    \label{fig:omega_k}
\end{figure}

\subsection{Saturation of the detector and reconstruction error}

As we have seen in Chapter \ref{sec:chapter_3} (\NOTE{mettre référence précises}), the algorithm reconstructing the positions at which the atoms fall on the MCP can give wrong results when the atomic flux is too high, which is typically the case with the very dense BEC. We saw that if we consider two atoms falling in the same pixel at times $t^A$ and  $t^B$ so that those two times are very close, the 8 times $(t_{x 1}^{A}, t_{x 2}^{A}, t_{y 1}^{A}, t_{y 2}^{A}, t_{x 1}^{B}, t_{x 2}^{B}, t_{y 1}^{B}, t_{y 2}^{B})$ can get mixed up and result in a mistake in the reconstruction process \cite{cayla_these}. The wrong reconstructed positions can then either be:

\begin{equation}
\begin{aligned}
&X 1=\frac{t_{x 1}^{A}-t_{x 2}^{B}}{2}=X+\frac{t^{A}-t^{B}}{2 v_{x}} \quad Y 1=\frac{t_{y 1}^{B}-t_{y 2}^{A}}{2}=Y+\frac{t^{B}-t^{A}}{2 v_{x}} \quad T 1=\frac{t^{A}+t^{B}}{2}\\
&\text { and }\\
&X 2=\frac{t_{x 1}^{B}-t_{x 2}^{A}}{2}=X+\frac{t^{B}-t^{A}}{2 v_{x}} \quad Y 2=\frac{t_{y 1}^{A}-t_{y 2}^{B}}{2}=Y+\frac{t^{A}-t^{B}}{2 v_{x}} \quad T 2=\frac{t^{A}+t^{B}}{2}
\end{aligned}
\end{equation}

\noindent or 

\begin{equation}
\begin{aligned}
&X 1=\frac{t_{x 1}^{A}-t_{x 2}^{B}}{2}=X+\frac{t^{A}-t^{B}}{2 v_{x}} \quad Y 1=\frac{t_{y 1}^{A}-t_{y_{2}}^{B}}{2}=Y+\frac{t^{A}-t^{B}}{2 v_{x}} \quad T 1=\frac{t^{A}+t^{B}}{2}\\
&\text { and }\\
&X 2=\frac{t_{x 1}^{B}-t_{x 2}^{A}}{2}=X+\frac{t^{B}-t^{A}}{2 v_{x}} \quad Y 2=\frac{t_{y 1}^{B}-t_{y 2}^{A}}{2}=Y+\frac{t^{B}-t^{A}}{2 v_{x}} \quad T 2=\frac{t^{A}+t^{B}}{2}
\end{aligned}
\end{equation}

\noindent We see from these equations that if we take $X \simeq 0$ and $Y \simeq 0$, $X1 \simeq -X2$ and $Y1 \simeq -Y2$. If the times $T1=T2$ correspond to $k_z=0$, the wrongly reconstructed pair of atoms looks exactly like a \kmk pair! Importantly, the saturation occurs mainly for BEC atoms because of the high density, and the BEC corresponds exactly to momentum values close to $k=0$, \ie  $X \simeq 0$ and $Y \simeq 0$. This is then a problem as we can artificially create \kmk pairs because of reconstruction issues.

To circumvent this issue, we remove from the analysis the momentum region where $|k_z| < 0.05 \ k_d$ which corresponds to the region where the wrongly reconstructed BEC atoms are, at the expense of losing some possible ``true'' pairs located in this region.

\NOTE{voir comment équilibrer cette partie avec le chapitre 3 a posteriori}


\subsection{Benchmarking of the algorithm with two-body collision spheres}

\label{sec:benchmark_algo}

Before using the algorithm to look for a supposedly small \kmk pairing signal in the depletion of a weakly-interacting Bose gas, it was crucial to test it on a data set with a large number of \kmk pairs to certify that it was working properly. Luckily, we could re-use the data taken for measuring two-body collisions during the time-of flight described in Chapter \ref{sec:chapter_3}. In addition to the 3D diffraction data that we already presented, we also took data with a single lattice beam to induce 1D diffraction and obtain only two collision spheres (see Fig.\ref{fig:1D_spheres}). The advantage is that we have a large number of atoms in the collision spheres, making the analysis easier. 

\begin{figure}
    \centering
    \includegraphics[width=0.7\textwidth]{Fig/Chapter4/1d_spheres.png}
    \caption{1D diffraction and associated collision spheres. Note that while the condensed peaks are shown here, they are removed before calculating the correlations.}
    \label{fig:1D_spheres}
\end{figure}

As shown in Chapter \ref{sec:chapter_3}, the collision spheres are populated by atoms pairs undergoing an elastic collision. After the collision, the two atoms of the pair are \kmk correlated. Contrary to the \kmk correlations of the quantum depletion, we de not expect a correlation peak at $\delta \bm{k}=\bm{0}$. If we consider for instance collisions between the 0th and 1st orders of diffraction along $z$, the overall momentum before and after the collision is $k_d \bm{z}$ ($\bm{z}$ is the unitary vector of the $z$ axis), so that the sum of the momenta of the two correlated atoms after the collision must be $k_d \bm{z}$. We thus expect a correlation peak for each sphere, one at $\delta k = (0,0,+1 \ k_d)$  and one at $\delta k = (0,0,-1 \ k_d)$.

We run the algorithm on the experimental data and obtain the results shown in Fig.-\ref{fig:kmk_kapitza}. We observe two correlations peaks at the expected locations! We can now extract the width and amplitude of the peaks with a Gaussian fit and see if they match our expectations. We find RMS widths of $\sigma_A^{(-1)}=2.3(9) \times 10^{-2} \ k_d$ and $\sigma_A^{(+1)}=2.7(8) \times 10^{-2} \ k_d$, the error bars being given by the uncertainty on the fit coefficients. This is consistent with the measured widths of the scattering halos presented in \NOTE{METTRE SECTION} for a few hundreds of thousands loaded atoms. 
The analysis of the amplitude is rather easy in this configuration as we know that every atom on the sphere has a momentum correlated partner. We can then now how much correlated pairs we are suppose to detect and compare it to the experiment. To count the number of detected pairs, we use the following procedure:


\begin{enumerate}
    \item The voxel size is increased to $\Delta k_{\parallel} = 0.3 \ k_d$ so that the central voxel contains the entirety of the correlation peak to count every true correlations.
    
    \item We count the number of coincidences $N_{\rm{numerator}}$ in the central voxel of the numerator of equation \ref{Eq:g2}. 
 
   
    \item  We count the number of coincidences $N_{\rm{denominator}}$ in the central voxel of the denominator of equation \ref{Eq:g2} to evaluate the number of accidental coincidences. 
    
    
    \item As the numerator contains both true and accidental coincidences, we evalute the number of true coincidences by subtracting the number of accidental coincidences and taking into account the normalization factor:
    \begin{equation}
          N_{\rm{pairs}}=N_{\rm{numerator}} - N_{\rm{denominator}} \times \frac{\sum_i N_i^2}{(\sum_i N_i)^2}
    \end{equation}
  
\end{enumerate}

We find a number of detected pairs of \NOTE{redo the analysis}. To evaluate the expected number of pairs for comparison, we must account for the detection efficiency $\alpha_{\rm{MCP}}$. The probability to detect both atoms in a pair is $\alpha_{\rm{MCP}}^2$, we then detect $\alpha_{\rm{MCP}}^2 N_{\rm{tot}}$ atoms that have a matching correlated atom, $N_{\rm{tot}}$ being the total number of atoms in the considered portion of sphere. Since the number of detected atoms $N_{\rm{detected}}$ that we easily access is $\alpha_{\rm{MCP}} N_{\rm{tot}}$, we can simply multiply $N_{\rm{detected}}$ by $\alpha_{\rm{MCP}}$ to get the number of expected pairs. We find \NOTE{redo the analysis} in very good agreement with the experimental value. On a side note, this measurement cross-checks the calibration of the detector efficiency described in Chapter \ref{sec:chapter_3}.

% \begin{itemize}
%     \item We first evaluate the number of accidental coincidences. To do so, We select the voxel corresponding to the maximum value of a correlation peak. We record the number of counts in the \textbf{normalization} histogram (denominator of equation \ref{Eq:g2}) in this voxel $N^{\rm{center}}_{\rm{ac-norm}}$. We re-scale it by the proper factor to compare it to the numerator values:
    
%     \begin{equation}
%         N_{\rm{ac}}^{\rm{center}}=N^{\rm{center}}_{\rm{ac-norm}} \frac{\sum N_i^2}{N_{\rm{atoms-norm}}^2}
%     \end{equation}
    
%     where $N_{\rm{atoms-norm}}$ is the number of atoms chosen for the calculation of the denominator.
    
%     \item We then evaluate the total number of true coincidences $N_{\rm{true}}$. To do so, we must account for the detection efficiency $\alpha_{\rm{MCP}}$. The probability to detect both atoms in a pair is $\alpha_{\rm{MCP}}^2$, we then detect $\alpha_{\rm{MCP}}^2 N_{\rm{tot}}$ atoms that have a matching correlated atom, $N_{\rm{tot}}$ being the total number of atoms in the considered portion of sphere. Since the number of detected atoms $N_{\rm{detected}}$ that we easily access is $\alpha_{\rm{MCP}} N_{\rm{tot}}$, we can simply multiply $N_{\rm{detected}}$ by $\alpha_{\rm{MCP}}$ to get the number of detected correlated atoms, giving us the number of true coincidences.
    
%     \item As the size of the voxel is smaller than the correlation peak, we only record a fraction of the coincidences in the central pixel of the peak. In Fig-\ref{fig:kmk_kapitza}, we deliberately chose the transverse size of the voxels to be larger than the correlation peak. In these conditions, the number of true coincidences in the central pixel writes:
    
%     \begin{equation}
%          N_{\rm{true}}^{\rm{center}}=N_{\rm{true}} \times \rm{erf} (\frac{\Delta k_{\parallel}}{2 \sqrt{2} \sigma_{A}})
%     \end{equation}
  
%   where $\Delta k_{\parallel}$ is the longitudinal size of the voxel. 
   
%   \item We measure $N^{\rm{center}}_{\rm{ac}}=1.44 \times 10^5$ and $N^{\rm{center}}_{\rm{true}}=5.89 \times 10^3$. As the numerator contains both true and accidental coincidences counts, the amplitude of the correlation peak writes:
   
%   \begin{equation}
%       \rm{Amplitude} = \frac{N^{\rm{center}}_{\rm{ac}} + N^{\rm{center}}_{\rm{true}}}{N^{\rm{center}}_{\rm{ac}}} = 1 + 0.04
%   \end{equation}
   
%   which is in very good adequation with the results of the algorithm.
    
% \end{itemize}



\begin{figure}
    \centering
    \includegraphics[width=0.7\textwidth]{Fig/Chapter4/kmk_kapitza.png}
    \caption{1D cut of the anomalous correlation function $\gtwo_A$ along the $z$ axis. We observe two correlation peaks, one for each correlation sphere. The peaks are not exactly at $-1 \ k_d$ and $1 \ k_d$ due to a slight mis-recentering of the data. The longitudinal size of the voxels is $\delta k_{\parallel}= 2 \times 10^{-2}$ and the transverse integration $\delta k_{\perp}= 9 \times 10^{-2}$}
    \label{fig:kmk_kapitza}
\end{figure}




\section{Observation of the pair correlation signal}

\label{sec:post_selec}

As we have the algorithm necessary to extract the correlation signal and checked that it is working properly, we now look to observe the \kmk pairs of the quantum depletion.

We prepare a BEC with a target number of $\NBEC=5 \times 10^3$ atoms in an optical lattice of amplitude $V=7.75 \ E_r$. With this lattice amplitude, we are in the superfluid domain of the phase diagram in which we expect the \kmk correlations. In order to have sufficient statistics, we repeat the experiment $\sim 4,000$ times. In practice, we cannot prepare BECs with the exact same number of atoms at each shot. We then need to post-select the data and remove runs with a detected atom number falling too far from the target number. We allow for $30 \%$ fluctuations around the target number and end up conserving around $\sim 2,000$ runs on which we run the algorithm.

\begin{figure}
    \centering
    \includegraphics[width=0.65\textwidth]{Fig/Chapter4/correlations_kmk_errorbars.png}
    \caption{1D cuts through the anomalous correlation function $g_{A}^{(2)}$ along the axis of the 3D optical lattice. The transverse integration is $\Delta k_{\perp}=3 \times 10^{-2} \ k_d$ and the longitudinal voxel size is $\Delta k_{\parallel}=1.2 \times 10^{-2} \ k_d$. The data is fitted by Gaussian functions (solid lines). The nice correlation peaks signal the presence of \kmk pairs. The error bars are obtained from the inverse square root of the number of counts in the voxels.}
    \label{fig:kmk_signal}
\end{figure}

We have plotted on Fig-\ref{fig:kmk_signal} 1D cuts through the calculated $g_{A}^{(2)}$ function on which we can see clear correlation peaks standing out from the noise! This is the kind of signal we were aiming to obtain and constitutes the central result of this thesis. Before analyzing the features of this correlation signal in more details, we conduct a first series of experimental checks. First, we extend the range of $\delta \bm{k}$ on which we plot the $g_{A}^{(2)}$ function to find correlation peaks at $\delta k = \pm k_d$ as shown on Fig.-\ref{fig:periodicity}. This is something that we expected from Bloch theorem (see \NOTE{reference section}) as the lattice is a periodic potential.

\begin{figure}
    \centering
    \includegraphics[width=0.7\textwidth]{Fig/Chapter4/periodicity.png}
    \caption{1D cut of the anomalous correlation function $\gtwo_A$. Because of the lattice periodic potential, we observe additional correlation peaks at $\pm k_d$.}
    \label{fig:periodicity}
\end{figure}

In a second time, we also check that there are no correlations in the coherent BEC state. We do so by selecting atoms with $|k_i|<0.04 \ k_d$ with $i=x,y,z$. We show on Fig.-\ref{fig:BEC_correlations} the calculated normal and anomalous correlation functions. As expected, both correlation functions are flat, expect for a small modulation of the order of $1 \%$ that is an artifact of the normalization procedure caused by shot-to-shot fluctuations of the width of the BEC (see \cite{cayla_these}).

\begin{figure}
    \centering
    \includegraphics[width=0.65\textwidth]{Fig/Chapter4/correlations_BEC.png}
    \caption{Normal and anomalous correlation functions in the BEC. The correlation functions are flat, except for a small-amplitude modulation that is due to normalisation issues induced by shot-to-shot fluctuations of the BEC width in momentum-space.}
    \label{fig:BEC_correlations}
\end{figure}

Now that we have observed an anomalous correlation signal, we must ask whether its origin is indeed the one we expect, namely \kmk pairing in the quantum depletion as a result of the interplay between interactions and quantum fluctuations. For starters, one of the key specificity of the \kmk pairs of the quantum depletion is that they exist in an \textbf{at-equilibrium} system. This is in strong contrast with \textbf{out-of-equilibrium} configurations where non linearities efficiently drive resonant processes. In these cases, both momentum and energy are conserved and no quantum treatment is required to explain the pairing process. In fact, \kmk pairing in such systems has already been observed on various experimental platforms such as:

\begin{itemize}
    \item Parametric down conversion in quantum optics \cite{burnham1970}.
    \item Dissociation of diatomic molecules in atomic physics \cite{greiner2005}.
    \item Elastic collisions in high energy physics \cite{arnison1982} or with ultracold atoms \cite{perrin2007observation} as we have seen earlier (see Fig.-\ref{fig:kmk_kapitza}).
\end{itemize}

The main difference with the \kmk correlation signal obtained in the collision spheres is that we observe here a peak located at $\delta \bm{k} = \bm{0}$. This signals that the total momentum of the atom pair is 0, and as our system consists of only an at-rest BEC, the pairing process could not have resulted from an out-of-equilibrium effect. This is the novelty of our experimental observation.

While these first observations seems to point towards the fact that we are indeed seeing the \kmk pairing of the quantum depletion, we can look to further characterize the correlation signal and determine whether our observations match what is expected from Bogoliubov theory to prove this point. In a first time, we will take a look at the effect of temperature that is supposed to destroy the \kmk correlation signal supposedly linked to $T=0$ quantum coherences as explained in Chapter \ref{sec:chapter_1}.

\section{Effect of temperature}

The temperature can be increased in a rather simple manner by holding the atoms at the final amplitude of the lattice for a longer duration, the gas being continuously heated over time (attributed to imperfections such as spontaneous emission or mechanical vibrations). We repeat the experiment with a holding time of $500 \ \rm{ms}$ corresponding to hundreds of tunneling times $225 \times h/J$. The increase in temperature can be seen by looking at the momentum density profile as shown in the panel b of Fig-\ref{fig:kmk_temperature}. As expected, the thermal depletion has increased, increasing the momentum density in the depletion region by a factor $\sim 4$. Note however that we did not increase the temperature too much to keep a significant condensed fraction of the order of $f_c = 29 \%$ not to entirely remove the quantum depletion. This increase of the thermal depletion implies a decrease of $\gtwo_A (\bm{0})$ of at least a factor $4^2$, bringing it under the experimental noise, as we see on Fig-\ref{fig:kmk_temperature} panel a.

We also repeated the experiment for an intermediate temperature obtained with a holding time of $200 \ \rm{ms}$ corresponding to $90 \times h/J$ tunneling time. The condensed fraction is then $f_c=55 \%$ and we observe, as expected, a peak of intermediate amplitude as shown on Fig-\ref{fig:kmk_3_temp}.

\begin{figure}
    \centering
    \includegraphics[width=\textwidth]{Fig/Chapter4/kmk_temperature_error_bars.pdf}
    \caption{Atom-atom correlations in weakly-interacting BECs at two different temperatures. The data for the low-temperature BEC ($f_{c}=84\%$), {\it resp.} for the heated BEC ($f_{c}=29\%$), are depicted in blue, {\it resp.} in red. 
    {\bf (a)}. Anomalous correlations $g_{A}^{(2)}(\delta k)$ at opposite momenta. The ${\bm k}$/$-{\bm k}$ peak disappears as the temperature increases.
    {\bf (b)}. 1D cut of the density $\rho(k)$ in semilog scale. The depletion density increases with temperature.
    {\bf (c)}. Normal correlations $g_{N}^{(2)}(\delta k)$ for the same datasets and $\Omega_k$. The peak amplitude shows no significant change as the temperature increases. Note that the transverse integration $\Delta k_{\perp}=1.5 \times 10^{-2} \, k_d$ used here reduces the amplitude of the peaks.}
    \label{fig:kmk_temperature}
\end{figure}

We thus see that the \kmk correlation signal is extremely sensitive to temperature as predicted, hinting to the fact that it is indeed related to a $T=0$ ground-state effect. It is also quite illuminating to compare the \kmk correlations to the bosonic bunching \kk correlations. As explained in Chapter \ref{sec:chapter_1}, the bosonic bunching effect is the consequence of chaotic statistics, a property shared by the thermal and quantum depletion. Therefore, changing the temperature and thus the balance between thermal and quantum depletion has no effect on bosonic bunching. This is what we observe experimentally as shown on Fig-\ref{fig:kmk_temperature} panel c.

In conclusion, we have on the same experimental data two very different behaviours with temperature that illustrate nicely the natures of the correlation signals. On the one hand, \kk correlations unaffected by temperature reveal the chaotic statistics of the system, on the other hand, \kmk correlations reveal the many-body ground state quantum coherences. These observations constitutes a rather convincing argument that we are indeed observing a \kmk correlation signal caused by the quantum depletion and not some other effect that we could have overlooked.

\begin{figure}
    \centering
    \includegraphics{Fig/Chapter4/kmk_3_temp_errorbars.pdf}
    \caption{Anomalous correlation function for data sets with different temperatures and condensed fractions. The amplitude of the $\bm{k}$/$-\bm{k}$ correlation signal is progressively lost as the temperature rises and the condensed fraction diminishes. Inset: amplitude of the correlation peak $g^{(2}_{A}({\bm 0})$ as a function of the condensed fraction $f_c$.}
    \label{fig:kmk_3_temp}
\end{figure}

We are then at this point rather convinced that we observe the \kmk correlations of the quantum depletion, as well as the expected bosonic bunching. We can now look to study in quantitative details these correlation signals. To start on this topic, the attentive reader would have noticed than even though we expect perfect bosonic bunching $\gtwo_N (\bm{0})=2$, we rather observe $\gtwo_N (\bm{0}) \simeq 1.8$. This is due to transverse integration effects of the 3D $\gtwo$ function that need to be accounted for to extract the proper information from the amplitude of the correlation peaks.

\section{Study of the amplitude of the correlation peaks}

\subsection{Transverse integration effects}

The ``true'' theoretical, normalized and integrated two-body correlation function is well modelled in first approximation by a 3D Gaussian function:

\begin{equation}
    \gtwo (\delta \bm{k}) = 1 + \eta_0 \prod_{i=x,y,z} \exp \left(\frac{-\delta k_i^2}{2 \sigma_i}\right)
    \label{eq:g2_theo}
\end{equation}

\noindent where we introduce $\eta_0$ the amplitude of the correlation peak and $\sigma_i$ the RMS width of the correlation peak in direction $i=x,y,z$. Note that we will detail here a general calculation that applies for both normal and anomalous correlation.

As mentioned in \ref{sec:numerical_calculation}, the values of the $\gtwo$ function are recorded in a 3D histogram made of cubic voxels whose dimensions will be denoted $\Delta k_{\parallel}$. To extract a 1D cut along the $x$ direction for instance, we could simply take the line of voxels verifying $\delta k_y=\delta k_z =0$. However, this is often not sufficient to have a proper signal-to-noise ratio. We therefore rather average the values of several voxels lines associated to $\delta \bm{k}$ values verifying $\abs{\delta k_y, \delta k_z} \leq \Delta k_{\perp}$ as illustrated on Fig-\ref{fig:cube_integration}.

\begin{figure}
    \centering
    \includegraphics[width=0.45\textwidth]{Fig/Chapter4/cube_integration.png}
    \caption{Illustration of the transverse integration. Every voxel contains the value of the $\gtwo$ function for a given value of $\delta \bm{k}=(\delta k_x, \delta k_y, \delta k_z$). The figure illustrates the procedure to take a 1D cut in the $x$ direction: we average over several pixel lines to increase the signal-to-noise ratio. This is the transverse integration $\Delta k_{\perp}$ as defined on the schematic.}
    \label{fig:cube_integration}
\end{figure}

We can now re-write equation \ref{eq:g2_theo} to show what is the function that is actually plotted when taking a 1D cut along the $x$ direction after the transverse integration process:

\begin{equation}
    g^{(2)} \left(\delta k_{x}\right)=\frac{1}{\left(2 \Delta k_{\perp}\right)^{2}} \iint_{-\Delta k_{\perp}}^{\Delta k_{\perp}} \left( 1+ \eta_0 \prod_{i=x, y, z} \exp \left(\frac{-\delta k_{i}^{2}}{2 \sigma_i^{2}}\right) \mathrm{d} \delta k_y \mathrm{d} \delta k_z \right)
\end{equation}

This expression can be analytically evaluated and writes:

\begin{equation}
    g^{(2)} \left(\delta k_{x}\right)=1+\eta_0 \frac{2 \pi \sigma_y \sigma_z}{\left(2 \Delta k_{\perp}\right)^{2}} \exp \left(\frac{-\delta k_{x}^{2}}{2 \sigma_{z}^{2}}\right) \operatorname{erf}\left(\frac{\Delta k_{\perp}}{\sqrt{2} \sigma_{y}}\right) \operatorname{erf}\left(\frac{\Delta k_{\perp}}{\sqrt{2} \sigma_{z}}\right)
\end{equation}

Note that we have here neglected the small longitudinal integration induced by the size of the voxel, which is 3 times smaller than the RMS width of the correlation peaks. In addition, we assume that the correlation peaks are isotropic as the lattice potential and then the spatial size of the gas are isotropic, giving $\sigma_x=\sigma_y=\sigma_z=\sigma$. We thus see that when measuring any correlation peak amplitude with a Gaussian fit on a 1D cut of the $\gtwo$ function, we get a reduced amplitude $\eta$ that writes:

\begin{equation}
    \eta (\Delta k_{\perp})= \eta_{0} \times \frac{2 \pi \sigma^2}{(2\Delta k_{\perp})^2}\left[\rm{erf} \left(\frac{\Delta k_{\perp}}{\sqrt{2}\sigma} \right)\right]^2
    \label{eq:fit_integration}
\end{equation}

The idea is then to measure $\eta$ for several values of $\Delta k_{\perp}$ and fit the data with equation \ref{eq:fit_integration} with $\eta_0$ and $\sigma$ as free parameters. For a single value of the transverse integration, $\eta$ is obtained by averaging the 3 amplitudes fitted on the 1D cuts along the 3 directions of space. The uncertainty on the parameters of the fit defines the error bars on the amplitude.

% This method is illustrated on Fig-\ref{fig:integration_kk} for the bosonic bunching measurement for two temperatures as discussed in the previous section, as well as on Fig-\ref{fig:eta_vs_int_kmk} for different anomalous correlation signals.


\subsection{Normal correlations}

Coming back to the normal correlations plot of Fig-\ref{fig:kmk_temperature}, we now apply the method we have just described to correct the effects of the transverse integration with the results shown on Fig-\ref{fig:integration_kk}. From the fitted value of $\eta_{0,N}$, we obtain $g^{(2)}_N(\bm{0})=2.05(6)$ and $g^{(2)}_N(\bm{0})=2.09(5)$ for the high and low condensed fraction data sets respectively. These numbers are consistent with $\gtwo_N (\bm{0})=2$ expected for perfect bosonic bunching, confirming the prediction of the Bogoliubov theory that the statistics of the system are chaotic. Note that our team conducted a thorough study of \kk correlations in the depletion of a lattice gas in \cite{cayla2020}. 

\begin{figure}
    \centering
    \includegraphics[width=0.65\textwidth]{Fig/Chapter4/eta_vs_int_kk.png}
    \caption{Fitted amplitude of the normal correlation peak $\eta_N$ as a function of the transverse integration $\Delta k_{\perp}$. The data is fitted with model defined in equation \ref{eq:fit_integration} with $\eta_0$ and $\sigma$ as free parameters. This method allows us to extract the amplitude at zero transverse integration $\eta_{0,N}$.}
    \label{fig:integration_kk}
\end{figure}

\subsection{Anomalous correlations}

We now turn to analyzing the amplitude of the anomalous correlation peaks. As we have seen with the calculations developed in section \ref{sec:amp_parametric}, we can draw a clear analogy between the Bogoliubov weakly-interacting Bose gas and the non-degenerate parametric amplifier in Quantum Optics. We showed that the amplitude of the anomalous correlation peak is expected to scale linearly with the inverse of what is called the average mode occupancy, {\it i.e.} the average number of photons per mode. In atomic physics, the analog to the average mode occupancy is the momentum density, {\it i.e.} the average number of atoms in a certain mode $\bm{k}$.

As explained in \ref{sec:accessing_depletion}, we measure integrated correlation functions over the momentum volume $\Omega_k$. The parameter setting the amplitude we observe is then the average momentum density $\bar{\rho}_{\Omega_k}$ defined as:

\begin{equation}
    \bar{\rho}_{\Omega_k}=\int_{\Omega_{k}} \rho({\bm k}) {\rm d}{\bm k}
\end{equation}

We have several possibilities to change the value of $\bar{\rho}_{\Omega_k}$:

\begin{itemize}
    \item Change the experimental parameters to load a different target total atom number $\NBEC$ in the optical lattice.
    \item ``Artificiall'' change the total atom number $\NBEC$ by changing the target atom number when doing the data post-selection (see \ref{sec:post_selec}).
    \item Change the bounds of the integration volume $\Omega_k$.
\end{itemize}

We prepare 4 data sets using a combination of these 3 possibilities. We perform the experiment with a target loaded atom number $\NBEC=[2.5,5,10] \times 10^3$, and extract an additional set with $\NBEC=3.5 \times 10^3$ from the data intended for $\NBEC = 5 \times 10^3$ by changing the post-selection criterion. We also reduce $\Omega_k$ to momenta between $k_{\rm{min}} = 0.3 \ k_d$ and $k_{\rm{max}} = 0.7 \ k_d$, {\it i.e.} to a region where the depletion is lower. We thus reduce $\bar{\rho}_{\Omega_k}$ to observe higher amplitude values in hope of observing a clear violation of the Cauchy-Schwarz inequality (see \ref{sec:cs_inequality}).

The results are plotted on Fig.-\ref{fig:amplitude_vs_rhob} alongside the normal correlations amplitude for comparison. As expected, we observe a nice linear scaling of the anomalous amplitude with the inverse average density. On the other side, changing $1/\rhob$ does not change the chaotic nature of the system statistics, we thus observe $\gtwo_N (\bm{0})=2$ independently of the value of $\rhob$. Once again, the amplitudes were corrected of transverse integration effects as shown on Fig.-\ref{fig:eta_vs_int_kmk}. Note that the normal correlations amplitude is not shown for the point at the lowest average density as the amplitude does not increase at lower densities, resulting in a decrease of the signal-to-noise ratio.

\begin{figure}
    \centering
    \includegraphics[width=0.65\textwidth]{Fig/Chapter4/amplitude_kmk.png}
    \caption{Amplitude of the correlation peaks versus the inverse average density $\rhob$. We observe a linear scaling for anomalous correlations, while normal correlations stay constant and compatible with $\gtwo_N (\bm{0})=2$.}
    \label{fig:amplitude_vs_rhob}
\end{figure}

\begin{figure}
    \centering
    \includegraphics[width=0.65\textwidth]{Fig/Chapter4/eta_vs_int_kmk.png}
    \caption{Fitted amplitude of the normal correlation peak $\eta_N$ as a function of the transverse integration $\Delta k_{\perp}$.}
    \label{fig:eta_vs_int_kmk}
\end{figure}

\subsubsection{Discussion on the detected number of atom pairs}

After looking at the scaling of the amplitude, we take a look at the absolute value of the amplitude from which we extract the number of detected \kmk pairs, following the procedure previously described in \ref{sec:benchmark_algo}. Note that we cannot directly compare the experimental values to equation \ref{eq:amp_g2_A} that holds only if the depletion is entirely quantum.

% \begin{enumerate}
%     \item The voxel size is increased to $\Delta k_{\parallel} = 0.3 \ k_d$ so that the central voxel contains the entirety of the correlation peak to avoid having to consider integration effects.
%     \item We count the number of coincidences $N_{\rm{numerator}}$ in the central voxel of the numerator of equation \ref{Eq:g2}. 
%     \item  We count the number of coincidences $N_{\rm{denominator}}$ in the central voxel of the denominator to evaluate the number of accidental coincidences.
%     \item We evaluate the number of atoms in a \kmk pair with $N_{\rm{pairs}}=N_{\rm{numerator}} - N_{\rm{denominator}} \times \frac{\sum_i N_i^2}{(\sum_i N_i)^2}$
% \end{enumerate}

For the data set $\NBEC=5 \times 10^3$, $f_c=84 \%$, we find that we detect on average $0.5$ pairs per shot, for roughly $N_{\Omega_k}=100$ atoms detected in $\Omega_k$. We push the analysis a little further to calculate the fraction of quantum depleted atoms in the overall depletion condensate that we will write $\alpha_Q$. We first assume that all atoms in the quantum depletion form a \kmk pair and that atoms forming a \kmk pair necessarily belong to the quantum depletion. We then have:

\begin{equation}
    N_{\rm{pairs}}= N_{\Omega_k} \alpha_{\rm{MCP}} \alpha_Q
    \label{eq:Npairs}
\end{equation}

\noindent This equation can be understood as follows. We initially have a given number of depleted atoms in $\Omega_k$ that we detect only a fraction of because of the MCP detection efficiency. This number is $N_{\Omega_k}$. In all these atoms, only the fraction of quantum depleted ones $\alpha_Q$ will be \kmk paired. In addition, we miss some pairs when detecting only one atom of the pair because of the detection efficiency, hence the addition of $\alpha_{\rm{MCP}}$ in the formula.

We use equation \ref{eq:Npairs} to evaluate $\alpha_Q$ and find $\alpha_Q \simeq 2 \%$. Using a $T=0$ Gutzwiller approach (see Chapter \ref{sec:chapter_2} (\NOTE{verifier quand le chapitre 2 est fait})), we estimate the quantum depletion to be $5\%$ and infer that the thermal depletion must then be $\simeq 10\%$ as the overall condensed fraction is $f_c=84 \%$. This would mean that we should rather have $\alpha_Q \simeq 33 \%$, so an order of magnitude larger than our experimental measurement. Finding a clear explanation for this discrepancy remains an open question and would require an extensive theoretical work far beyond the scope of this thesis. We can however suggests a few possible explanations:

\begin{itemize}
    \item While we estimate the total quantum depletion, we cannot count all the pairs as some of them are located in the region of the BEC removed from the analysis. At the moment, we do not have the theoretical tools necessary to determine how much pairs are removed that way.
    \item Some of the pairs that we detect are located close to the edge of the first Brillouin where the relation dispersion becomes flat. It is then not so clear whether Bogoliubov's theory predictions for an homogeneous gas should hold, as the Bogoliubov dispersion relation is an increasing function of the momentum.
    % \item The balance between thermal and quantum depletion depends from the momentum $k$. It is then not so clear of what this number should be averaged over $\Omega_k$. In addition, a significant amount of \kmk pairs can be located in the low $k$ region that we remove from the analysis, thus reducing the number of detected pairs.
    % \item Dispersion relation of the lattice \NOTE{COMPLETER}
    % \item Is is also not so clear how the pairs are projected into the different Brillouin zones when performing the time-of-flight measurement. This could also makes us miss some pairs (\NOTE{BOF}).
\end{itemize}

In conclusion, the observed linear scaling is consistent with what is predicted by Bogoliubov theory and gives us another argument showing that the observed anomalous correlation signal is linked to the quantum depletion. The absolute value of the amplitude cannot however be understood in the framework of the Bogoliubov theory that does not account for all the specificities of our experiment, such as the presence of the lattice. In fact, evaluating this number is currently beyond what is possible to calculate, making our experiment an example of quantum simulation.


\section{Towards measuring entanglement}

As we underlined in the first chapter of this thesis, it would be of great interest to characterize how entanglement emerges in at-equilibrium many-body systems such as the one we are working with here. Although we do not have all the experimental tools to claim that we are indeed seeing entanglement in our experiment yet, we nevertheless observe clear signatures of quantum phenomena that hints towards it as we will discuss now.



\subsection{Relative number squeezing}

As detailled in paragraph \ref{sec:correlated_pop} of Chapter \ref{sec:chapter_1}, if we were in the perfect situation $T=0$ were the depletion is fully quantum, the populations in modes $\bm{k}$ and $-\bm{k}$ would be totally correlated so that the quantity $N(\bm{k})-N(-\bm{k})$ is non fluctuating and always equal to 0. This is obviously not the case in our finite temperature experiment where a significant fraction of the depletion is thermal and thus uncorrelated. However, we expect the \kmk correlations to reduce the fluctuations of $N(\bm{k})-N(-\bm{k})$. This is what we call \textbf{relative number squeezing}, similar to --but not to be confused with-- regular squeezing \cite{walls1983squeezed} that denotes the reduction of the fluctuations of an operator ({\it e.g} momentum) at the expense of increased fluctuations for the conjugate operator ({\it e.g} position).

% A slightly higher level proof of quantum behavior than the violation of the Cauchy-Schwarz inequality is observing \textbf{squeezing}. Before going into our experimental details and observations, let us briefly remind what is squeezing is.

% In quantum mechanics, the maximum precision with which one can hope to measure two conjugate quantities such as position and momentum is bound by the very famous Heisenberg inequality:

% \begin{equation}
%     \Delta x \Delta p \geq \frac{\hbar}{2}
% \end{equation}

% In most cases, notably the quantum harmonic oscillator ground state $\ket{0}$ and coherent states $\ket{\alpha}$, the minimum uncertainties values saturating the Hesienberg inequality are the same for the two conjugate quantities $\Delta x_{g} = \Delta p_g$. The specificity of squeezed states is that the uncertainty one of the quantities has been reduced below the lower limit of the ground state, at the detriment of the uncertainty of the conjugate quantity. If we were to represent the uncertainty in quadrature space, we would get a perfect circle for a coherent state and a "squeezed" circle, {\it i.e.} an ellipse, for a squeezed state, hence the name. \NOTE{BIBLIO/EXEMPLES}, \NOTE{VOIR AVEC CHAPITRE 1 OU METTRE CA}

% A well-known example of how to obtain squeezing is the non degenerate parametric amplifier in non linear quantum optics. \NOTE{A faire quand le chapitre 1 est fait}

 The idea is to measure the statistics of the difference of atom numbers in modes $\bm{k}$ and $-\bm{k}$ that we will refer to as $N(\bm{k})$ and $N(-\bm{k})$. For one of the modes, the fluctuations of the number of atoms are set by the \textbf{shot noise} and follow a Poisson law. The particularity of the Poisson law for a random variable is that the variance of the variable is equal to its mean, giving for instance for mode $\bm{k}$:

\begin{equation}
    \sigma_{N(\bm{k})}^2 =\mean{N(\bm{k})^2}-\mean{N(\bm{k})}^2 = \mean{N(\bm{k})}
\end{equation}

What can we tell of the statistics of the number difference in two modes $\bm{k}$ and $\bm{k}'$, $N(\bm{k}) - N(\bm{k}')$? If the populations in the two modes are totally uncorrelated, the difference of two independent Poissonian random variables is Poissonian as well and we then get:

\begin{equation}
    \sigma_{N(\bm{k})-N(\bm{k}')} = \sqrt{\mean{N(\bm{k})-N(\bm{k}')}}
\end{equation}

If we now chose $\bm{k}'=-\bm{k}$, we expect the \kmk correlations present in the depletion to reduce the fluctuations of the number difference, yielding what is called a \textbf{sub-Poissonian} law. Just like the \kmk correlation signal is lost with temperature, we expect that the reduction of the number difference fluctuations becomes smaller and smaller as temperature increases, increasing the fraction of thermally, uncorrelated, depleted atoms. 

Our goal is then to measure $\sigma_{N(\bm{k})-N(-\bm{k})}$ and see whether it is smaller than the expected value for a Poisson law. To this end, we define a convenient quantity that we call the squeezing parameter $\xi$: 

\begin{equation}
    \xi_{\bm{k},\bm{k'}}^2=\frac{\langle (N_{\bm{k}} - N_{\bm{k'}})^2 \rangle - \langle N_{\bm{k}} - N_{\bm{k'}} \rangle^2}{\langle N_{\bm{k}} \rangle + \langle N_{\bm{k'}} \rangle}
\end{equation}

This squeezing parameter to the square is simply the standard deviation of the number difference normalized by the expected standard deviation for uncorrelated variables. Therefore, we expect $\xi^2_{\bm{k},\bm{k}'}=1$ if there are no correlations between modes $\bm{k}$ and $\bm{k'}$, and $\xi^2_{\bm{k},\bm{k}'} < 1$ if the modes populations are correlated.

To evaluate $\xi^2$, we record for each experimental shots the number of detected atoms in cubic boxes paving the entire integration volume $\Omega_k$. The boxes need to be quite large in order to have enough atoms in each boxes. We chose a size of $(0.3 \ k_d)^3$. The average numbers of detected atoms per shot are $\sim 100$, $\sim 240$ and $\sim 360$ for the data sets with $f_c=84\%$, $f_c=55\%$ and $f_c=29\%$ respectively. Having access to the atom number for each of the $\sim 2000$ runs at various values of $\bm{k}$, we compute the squeezing parameter between different pairs of boxes as illustrated on Fig.-\ref{fig:squeezing}, either correlated or uncorrelated for reference. The measured values of $\xi^2$ are averaged over all possible pairs of boxes for a given configuration. The uncertainty on $\xi^2$ is evaluated statistically on all the $\xi^2$ values obtained on the 62 different boxes pairs and defined as the standard error.

\begin{figure}
    \centering
    \includegraphics[width=0.7\textwidth]{Fig/Chapter4/squeezing.png}
    \caption{Relative number squeezing measurement. The number of detected atoms in cubic boxes are recorded for each experimental run. We compute the squeezing parameter $\xi$ between correlated (orange) boxes as illustrated on the right, or uncorrelated (blue) boxes on the left. The central black box corresponds to the BEC, removed from the analysis.}
    \label{fig:squeezing}
\end{figure}

The results are summarized in Table \ref{tab:squeezing}. For the low-temperature, high condensed fraction data, we are indeed able to observe a small relative number squeezing within the errorbars. For uncorrelated modes, the squeezing parameter is very slightly above 1, highlighting that the correlations indeed reduce the fluctuations of the number difference. For higher temperatures (lower condensed fraction), no relative number squeezing is observed as the correlations are drowned out. In addition, we see that the squeezing parameter increases as we increase the temperature. This is due to global atom number fluctuations of the order of $15\%$ in our experiment. At low temperature, there are few detected atoms and the shot noise is thus large and dominating the total atom number fluctuations. On the opposite, at higher temperatures, the number of depleted atoms increases, reducing the shot noise. The contribution of total atom number fluctuations are then not negligible anymore and increase the fluctuations of the number difference, higher than what is expected for a Poisson law, explaining why we observe $\xi^2 > 1$.

\begin{table}[]
    \arrayrulecolor{black}
    \centering
    \begin{tabular}{|c|c|c|}
        \hline
        $f_c$ & $\xi^2_{\bm{k},-\bm{k}}$ & $\xi^2_{\bm{k},\bm{k}}$ \\
        \hline
        0.84 & 0.992(3) & 1.004(3) \\
        \hline
        0.55 & 1.017(4) & 1.017(3) \\
        \hline
        0.29 & 1.040(5) & 1.045(5) \\
        \hline
    \end{tabular}
    \caption{Experimental values of the squeezing parameter for correlated and uncorrelated modes and different condensed fractions.}
    \label{tab:squeezing}
\end{table}

\subsection{Experimental violation of the Cauchy-Schwarz inequality}

We have previously shown in \ref{sec:cs_inequality} how the Cauchy-Schwarz inequality writes with creation and annihilation operators and translated it in terms of correlation functions. We remind that we obtained:

\begin{equation}
    \gtwo_A (\bm{k},-\bm{k}) \leq \gtwo_N (\bm{k},\bm{k})
\end{equation}

We thus have a clear violation of the Cauchy-Schwarz inequality with our experimental data on 3 different data points, with a maximum violation of $4.27(8) > 2$ (Fig.-\ref{fig:amplitude_vs_rhob}). This adds up to the list of quantum behaviour signatures in the anomalous correlation signal.

As discussed in \ref{sec:cs_inequality}, violating the Cauchy-Schwarz inequality equals fulfilling the Busch-Parentani criterion to prove entanglement, provided that certain conditions are fulfilled. The first one is that the statistics of the system must be chaotic. This is something that we have experimentally verified by measuring $\gtwo_N (\bm{0}) = 2$. The second condition is to have $\langle a^{\dagger}({\bm k}) a({-\bm k}) \rangle=0$. While this is true in Bogoliubov theory and would be reasonable to assume in our experiment, this is not something that have experimentally measured and that therefore forbids us to claim that we observe entanglement in momentum space. While we do not have at the moment the experimental tools necessary to measure $\langle a^{\dagger}({\bm k}) a({-\bm k}) \rangle$, proving the presence of entanglement in momentum space in many-body ground states would be a significant result that motivates the improvement of our experimental setup in the near future.




% The very famous Cauchy-Schwarz inequality has seen countless applications in mathematics and physics. What will most interest us here is its formulation in the framework in probability theory. In classical physics, with two fluctuating quantities $I_1$ and $I_2$, the Cauchy-Schwarz inequality writes:

% \begin{equation}
%     \mean{I_1 I_2} \leq \sqrt{\mean{I_1^2} \mean{I_2^2}}
% \end{equation}

% This inequality can be rewritten with creation/annihilation operators to work with two-body correlation functions. For simplicity sake, let's consider only two modes 1 and 2. We introduce the notation $G^{(2)}_{i,j} = \mean{\hat{a}(i)^{\dagger} \hat{a}(j)^{\dagger} \hat{a}(i) \hat{a}(j)}$. The Cauchy-Schwarz inequality becomes \cite{kheruntsyan2012violation,walls2008}:

% \begin{equation}
%     G^{(2)}_{1,2} \leq \sqrt{G^{(2)}_{1,1} G^{(2)}_{2,2} }
% \end{equation}

% If we now chose mode $1 \rightarrow \bm{k}$, mode $2 \rightarrow -\bm{k}$ and consider a symmetrical case with $\mean{\hat{a}^{\dagger}(\bm{k}) \hat{a}(\bm{k})}=\mean{\hat{a}^{\dagger}(-\bm{k}) \hat{a}(-\bm{k})}$ (consistent at $T=0$ where the depletion is fully quantum and exclusively populated in pairs), we obtain $G^{(2)}_{\bm{k},\bm{k}}=G^{(2)}_{-\bm{k},-\bm{k}}$ and finally:

% \begin{equation}
%     \gtwo_A (\bm{k},\bm{-k}) \leq \gtwo_N (\bm{k},\bm{k})
% \end{equation}

% Therefore, the Cauchy-Schwarz inequality states that the cross-correlation amplitude between opposite momenta cannot exceed the amplitude of the auto-correlation with a classical model. As shown on Fig-.\ref{fig:amplitude_vs_rhob}, the experimental amplitude of the anomalous correlations goes significantly higher than for normal correlations, thus clearly violating the Cauchy-Schwarz inequality for classical quantities: we are thus observing quantum correlations. As developed in \cite{kheruntsyan2012violation}, violating the Cauchy-Schwarz inequality is one of the easiest way to prove the presence of non-classical correlations and represents a first, necessary step towards obtaining more significant results such as proving the presence of entanglement.

\section{Study of the width of the correlation peaks}

The last aspect of the \kmk correlation signal that remains to be studied so far is the width of the correlation peak. Once again, this quantity contains meaningful information about the many-body ground state. The key aspect is the same used by Hanbury Brown and Twiss in their seminal paper to measure the size of Sirius through the measurement of the second order correlation function, namely the width of the correlation peak is inversely proportional to the spatial size of the source. 

%This argument validates our previous assumption that the correlation peak is isotropic, as the lattice trapping potential is isotropic so that the spatial size of the BEC is the same in the three directions of space.

As we have done throughout this chapter, we will compare the predictions for both anomalous and normal correlations. We then need to determine what are the sizes of the components responsible for both correlation signals. On the one hand, the anomalous correlations are exclusively caused by quantum depleted atoms. The quantum depletion is non-existent outside of the BEC so its spatial is the one of the BEC. Therefore, we expect the width of the anomalous correlation peak $\sigma_A$ to be inversely proportional to the size of the BEC $L_{\rm{BEC}}$. On the other hand, the normal correlations are caused by quantum depleted atoms but also and importantly thermally depleted atoms whose spatial size extends beyond the BEC because of the increased kinetic energy. This tells us that the width of the normal correlations peak $\sigma_N$ should be smaller than for anomalous correlations.

\begin{figure}
    \centering
    \includegraphics[width=0.65\textwidth]{Fig/Chapter4/widths.png}
    \caption{RMS widths of the anomalous (green) and normal (blue) correlation peaks. The green area represents the expected with from the BEC width $\sigma_{\rm{BEC}}$ (see main text).}
    \label{fig:width}
\end{figure}

We plot on Fig.-\ref{fig:width} the experimental correlation peaks widths for different total atom numbers. The horizontal error bars are obtained in the same way than for Fig.-\ref{fig:amplitude_vs_rhob} , while the vertical error bars correspond to the standard deviation of the mean over the three directions of the momentum space. As expected, we observe that for all atom numbers, $\sigma_A > \sigma_N$. In addition, one would expect to see both widths decrease with the total atom number as the size of the system grows. This is more or less what the experimental data suggests but the error bar are not small enough to make a proper conclusion on this point. Note that in the Thomas-Fermi regime, the size of the BEC changes slowly with the total atom number $L_{\rm{BEC}} \propto \NBEC^{1/5}$, translating in a change of $\sim 30\%$ of the width of the correlation peaks from the data with $N=2.5 \times 10^3$ to $N=10 \times 10^3$ that is hard to resolve within our error bars.

We now look at the quantitative value of $\sigma_A$ that can be numerically evaluated. The calculations have been performed by S. Butera and I. Carusotto from the BEC center in Trento, Italy with the Bogoliubov theory for a trapped 1D system \cite{butera2020}. They evaluate $\sigma_{A,\rm{theo}} = 0.94 \sigma_{\rm{BEC}}$, with $\sigma_{\rm{BEC}}$ the RMS width in momentum space of the condensate. This relation is not modified in presence of a lattice as the size $L_{\rm{BEC}}$ does not change when the ratio $\mu/\hbar \omega$ is fixed. This is explained by the equality $m \omega^2 = m^* \omega^* ^2$ with $m^*$ the effective mass in the lattice and $\omega^*$ the corresponding effective frequency as defined in Chapter \ref{sec:chapter_2} \NOTE{référence}.

We therefore need to measure $\sigma_{\rm{BEC}}$ to exploit this result. To do so, we need to be careful of the saturation of the detector. Indeed, the BEC is very dense resulting in a high flux of atoms saturating the detector. If we plot 1D cuts of the momentum density, the BEC momentum profile is then flattened at the top and fitting with a Gaussian function over-evaluates the momentum width of the condensate. To circumvent this issue, we adapt the parameters of the Raman transfer (\NOTE{RAJOUTER SECTION}) to reduce drastically the detection efficiency and thus the flux of atoms to avoid saturation of the detector and ensure proper fitting.

During our first analysis, we noticed that the anomalous correlation peak was larger than the BEC, contrary to what we would have expected. We also noticed that the momentum width of the BEC was larger than what we obtained applying the Gutzwiller ansatz with our calibrated atom numbers. We attribute this to an imperfection in our experiment, the shot-to-shot fluctuations of the center-of-mass of the atomic distribution. When averaging over many experimental runs, these fluctuations enlarges artificially the measured width of the BEC, as well as the width of the anomalous correlation peak. These fluctuations are nevertheless easy to characterize by comparing the experimental momentum width of the BEC to the one predicted by the Gutzwiller variational approach.

When accounting for center-of-mass fluctuations, the measured momentum density results from the convolution with the distribution of center-of-mass displacements and has a RMS width:

\begin{equation}
    \sigma_{\rm{BEC}}=\sqrt{\sigma_{\rm{BEC},0}^2+\Delta k_{\rm{com}}^2}
\end{equation}

\noindent where $\sigma_{\rm{BEC},0}$ is the ``true'' BEC momentum width. For instance, the Gutzwiller variational approach gives  $\sigma_{\rm{BEC},0} \simeq 1.7 \times 10^{-2} \ k_d$ for a total atom number $\NBEC = 5 \times 10^5$ and we measure $\sigma_{\rm{BEC}}=2.0(1) \times 10^{-2} \ k_d$. From this we deduce $\Delta k_{\rm{com}}=1.1(1) \times 10^{-2} \ k_d$. We repeat the procedure to evaluate $\Delta k_{\rm{com}}$ for all of the data sets of Fig.-\ref{fig:width}.

For a \kmk pair of atoms, a center-of-mass displacement ${\rm{d}} \bm{k}$ induces a momentum difference $\delta \bm{k}=2 {\rm{d}} \bm{k}$. The effect of the fluctuations are thus twice larger for anomalous correlations peak width than for the BEC momentum width:

\begin{equation}
    \sigma_A = \sqrt{\sigma_{A,0}^2+4\Delta k_{\rm{com}}^2}
\end{equation}

\noindent Combining this with the numerical evaluation and the measured values of $\Delta k_{\rm{com}}$, we obtain a corrected estimate of $\sigma_A$ that is represented as the green area in Fig.-\ref{fig:width}. The width of the area represent the uncertainty given by the uncertainty on the measurement of $\sigma_{\rm{BEC}}$ and the uncertainty on the determination of $L_{\rm BEC}$ caused by fluctuations of the total atom number. We find that our experimental data matches nicely the numerical calculations of \cite{butera2020}.

We note that the fluctuations of the center-of-mass have however no effect on the normal correlation signal. Within a given shot, the center-of-mass fluctuations simply manifest as a global displacement of all the atoms of this shot by a quantity that we note $\bm{k}_{\mathrm{COM}}$. As a result, the momentum difference $\delta \bm{k} = \bm{k}_1 - \bm{k}_2$ between two atoms of this shot is not affected by the COM fluctuations, $(\bm{k}_1+\bm{k}_{\mathrm{COM}}) - (\bm{k}_2+\bm{k}_{\mathrm{COM}})=\bm{k}_1 - \bm{k}_2$. In turn, the normal correlation function $g^{(2)}_{N}(\delta \bm{k})$ remains unchanged. When accounting for the effect of center-of-mass fluctuations, we then rather observe $\sigma_A \geq \sigma_N$ as not all anomalous widths of Fig-\ref{fig:width} are distinguishable from the normal widths at the same total atom number within error bars.

\section{Conclusion}

We have reached the end of our investigation of two-body correlation in the depletion of weakly interacting Bose gases. To sum things up, let us remind the main steps of the work conducted during this thesis and its main results.

\begin{itemize}
    \item We decided to try to detect experimentally the \kmk pairs of the quantum depletion predicted by the Bogoliubov theory of the weakly-interacting Bose gas, as this phenomenon is one of the conceptually simplest non-trivial, many-body, quantum correlation effect. By conducting this study, we hope to better understand the physics of many-body ground states and how correlations and entanglement emerge through the combined effects of interactions and quantum fluctuations.
    \item We based our experimental procedure on an experimental setup producing $\He$ BECs  implementing a 3D single-atom resolved electronic detection technique. We decided to use an optical lattice to be in the low-temperature regime dominated by interactions to ensure a sufficient level of quantum depletion so that the \kmk pairs can be properly detected.
    \item We completed previous benchmarking measurements \cite{cayla2018single} by measuring two-body time-of-flight collisions with large number of atoms in the lattice. From this, we were able to conclude that two-body collisions are negligible for the typical number of atoms used in our correlation measurement experiments. This validates the ballistic relation linking the momentum of the in-trap to their detected position after the TOF.
    \item We measured the temperature of the gas at different amplitudes of the lattice potential to certify the adiabatic preparation of the system.
    \item We implemented a two-photon Raman transfer scheme to replace the previous RF transfer, improving the detection efficiency by a factor 2.
    \item We adapted an existing algorithm to make it suited to extract the \kmk correlation signal from the experimental data and benchmarked it with collision spheres data.
    \item We were able to successfully observe anomalous and normal correlation signals in various data sets with different total atom numbers. We validated that the observed anomalous correlation signal is linked to the \kmk pairs of the quantum depletion by observing that:
    \begin{itemize}
        \item The signal is lost with temperature, contrary to the normal correlation signal that reveal the chaotic statistics of the system, unaffected by temperature.
        \item $\gtwo_A(\bm{0})$ scales linearly with the inverse average momentum density $\bar{\rho}_{\Omega_k}$, while $\gtwo_N(\bm{0})$ remains constant.
        \item We get relative number squeezing between correlated modes.
        \item We violate Cauchy-Schwarz inequality by observing $\gtwo_A(\bm{0}) > \gtwo_N(\bm{0})$ by more than a factor 2.
        \item The width of the anomalous correlation peak is larger than for normal correlations $\sigma_A > \sigma_N$ and matches the numerical calculations of \cite{butera2020} when accounting for center-of-mass fluctuations.
    \end{itemize}
\end{itemize}

For the last chapter of this thesis, we will shift our attention to another topic and exploit another feature of the single atom resolution of our experiment. While we have used it so far to look for correlations between individual particles, a strong advantage that we have overlooked is the possibility to detect very low densities signals inaccessible with optical measurements. An example of such a signal is the $k^{-4}$ tails in the momentum density of 1D gases, also known as Tan's contact \cite{tan2008large}. In the next chapter, we will briefly give the essential elements of the theory of 1D gases and detail our preliminary experimental efforts towards measuring Tan's contact.

